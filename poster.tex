\documentclass[14pt]{scrartcl}
\usepackage{url}

\newenvironment{citemize}{
\begin{list}{$\bullet$}{\setlength{\itemsep}{0pt} \setlength{\rightmargin}{0pt} \setlength{\leftmargin}{0.5\labelwidth} \setlength{\topsep}{0pt}}
}{\end{list}}

\def\B{{\mathbb B}}
\def\C{{\mathbb C}}
\def\D{{\mathbb D}}
\def\Fp{{\mathbb F}_p}
\def\Fell{{\mathbb F}_{\ell}}
\def\F{{\mathbb F}}
\def\H{{\mathbb H}}
\def\M{{\mathbb M}}
\def\N{{\mathcal N}}
\def\O{{\mathcal O}}
\def\0{{\mathbb 0}}
\def\P{{{\mathbb P}}}
\def\Q{{\mathbb Q}}
\def\R{{\mathbb R}}
\def\T{{\mathbb T}}
\def\Z{{\mathbb Z}}

\newcommand{\sol}{_{a^p,b^p,c^p}}
\newcommand{\bound}{\partial}
\newcommand{\la}[1]{\mathfrak{#1}}
\newcommand{\im}{\operatorname{Im}}
\newcommand{\ann}{\text{Ann} \hspace{0.1em} }
\newcommand{\rank}{\text{rank} \hspace{0.1em} }
\newcommand{\coker}[1]{\text{coker}\hspace{0.1em}{#1}}
\newcommand{\sgn}{\text{sgn}}
\newcommand{\lcm}{\text{lcm}}
\newcommand{\re}{\text{Re}  \hspace{0.1em} }
\newcommand{\ext}[1]{\text{Ext}(#1)}
\newcommand{\Hom}[1]{\text{Hom}(#1)}
\newcommand{\End}[1]{\text{End(#1)}}
\newcommand{\bs}{\setminus}
\newcommand{\rpp}[1]{\mathbb{R}\text{P}^{#1}}
\newcommand{\cpp}[1]{\mathbb{C}\text{P}^{#1}}
\newcommand{\tr}{\text{tr}\hspace{0.1em} }
\newcommand{\inner}[1]{\langle {#1}\rangle}
\newcommand{\tensor}{\otimes}
\newcommand{\Cl}{\text{Cl}}
\renewcommand{\sp}[1]{\text{Sp}_{#1}}
\newcommand{\GL}{\text{GL}}
\newcommand{\PGL}{\text{PGL}}
\newcommand{\SL}{\text{SL}}
\newcommand{\so}[1]{\text{SO}_{#1}}
\newcommand{\SO}{\text{SO}}
\newcommand{\pso}[1]{\text{PSO}_{#1}}
\renewcommand{\o}[1]{\text{O}_{#1}}
\renewcommand{\sp}[1]{\text{Sp}_{#1}}
\newcommand{\psp}[1]{\text{PSp}_{#1}}
\newcommand{\Span}{\rm Span}
\newcommand{\Frob}{\rm Frob}
\newcommand{\tor}{\rm tor}
\newcommand{\rad}{{\rm{rad}}}
\newcommand{\denom}{\rm denom}
\renewcommand{\bar}{\overline}
\newcommand{\notdiv}{\nmid}
\newcommand{\pfrac}[2]{\left( \frac{#1}{#2} \right)}
\newcommand{\bfrac}[2]{\left| \frac{#1}{#2} \right|}
\newcommand{\Ell}{\rm{Ell}}
\newcommand{\AV}{\rm{AV}}
\newcommand{\Gal}{\text{Gal}}
\newcommand{\ord}{{\rm{ord}}}

\newcommand{\kron}[2]{\bigl(\frac{#1}{#2}\bigr)}
\newcommand{\leg}[2]{\Biggl(\frac{#1}{#2}\Biggr)}

\DeclareSymbolFont{bbold}{U}{bbold}{m}{n}

%%
%
% 
% N.B. This format is cribbed from one obtained from the University
% of Karlsruhe, so some macro names and parameters are in German
% Here is a short glossary:
% Breite: width
% Hoehe: height
% Spalte: column
% Kasten: box
%
% All style files necessary are part of standard TeTeX distribution
%
% The recommended procedure is to first generate a ``Special Format" size poster
% file, which is relatively easy to manipulate and view.  It can be
% resized later to A0 (900 x 1100 mm) full poster size, or A4 or Letter size
% as desired (see web site).  Note the large format printers currently
% in use at UF's OIR have max width of about 90cm or 3 ft., but the paper
% comes in rolls so the length is variable.  See below the specifications
% for width and height of various formats.  Default in the template is
% ``Special Format",  with 4 columns.
%%
%% 
%% Choose your poster size:
%% For printing you will later RESIZE your poster by a factor
%%        2*sqrt(2) = 2.828    (for A0)
%%        2         = 2.00     (for A1) 
%%  
%% 

 \def\breite{20in}
 \def\hoehe{15in}
 \def\anzspalten{4}

%%
%% Procedure:
%%   Generate poster.dvi with latex
%%   Check with Ghostview
%%   Make a .ps-file with ``dvips -o poster.ps poster''
%%   Scale it with poster_resize poster.ps S
%%   where S is scale factor
%%     for Special Format->A0 S= 2.828 (= 2^(3/2)))
%%     for Special Format->A1 S= 2 (= 2^(2/2)))
%% 
%% Sizes (European:)
%%   A3: 29.73 X 42.04 cm
%%   A1: 59.5 X 84.1 cm
%%   A0: 84.1 X 118.9 cm
%%   N.B. The recommended procedure is ``Special Format x 2.82"
%%   which gives 90cm x 110cm (not quite A0 dimensions).
%%
%% --------------------------------------------------------------------------
%%
%% Load the necessary packages
%% 
\usepackage{palatino}
\usepackage[latin1]{inputenc}
\usepackage{epsf}
\usepackage{graphicx,psfrag,color,pst-grad}
\usepackage{amsmath,amssymb}
\usepackage{latexsym}
\usepackage{calc}
\usepackage{multicol}
\usepackage{amsthm}

\newtheorem*{define}{Definition}

%%
%% Define the required numbers, lengths and boxes 
%%
\newsavebox{\dummybox}
\newsavebox{\spalten}

\newlength{\bgwidth}\newlength{\bgheight}
\setlength\bgheight{\hoehe} \addtolength\bgheight{-1mm}
\setlength\bgwidth{\breite} \addtolength\bgwidth{-1mm}

\newlength{\kastenwidth}

%% Set paper format
\setlength\paperheight{\hoehe}                                             
\setlength\paperwidth{\breite}
\special{papersize=\breite,\hoehe}

\topmargin -1in
\marginparsep0mm
\marginparwidth0mm
\headheight0mm
\headsep0mm

%% Minimal Margins to Make Correct Bounding Box
\setlength{\oddsidemargin}{-2.47cm}
\addtolength{\topmargin}{-4mm}
\textwidth\paperwidth
\textheight\paperheight

\parindent0cm
\parskip1.5ex plus0.5ex minus 0.5ex
\pagestyle{empty}


\definecolor{MyBlue}{rgb}{0,0.08,0.45}
\definecolor{MyGreen}{rgb}{0,0.5,0.25}
\definecolor{TitleGreen}{rgb}{0,0.5,0.35}
\definecolor{recoilcolor}{rgb}{1,0,0}
\definecolor{occolor}{rgb}{0,1,0}
\definecolor{pink}{rgb}{0,1,1}
\definecolor{mybrown}{rgb}{0.6, 0.3, 0}

\def\UberStil{\normalfont\sffamily\bfseries\large}
\def\UnterStil{\normalfont\sffamily\small}
\def\LabelStil{\normalfont\sffamily\tiny}
\def\LegStil{\normalfont\sffamily\tiny}

%%
%% Define some commands
%%
\definecolor{JG}{rgb}{0.1,0.9,0.3}

\newenvironment{kasten}{%
  \begin{lrbox}{\dummybox}%
    \begin{minipage}{0.96\linewidth}}%
    {\end{minipage}%
  \end{lrbox}%
  \raisebox{-\depth}{\psshadowbox[framesep=1em]{\usebox{\dummybox}}}\\[0.5em]}
\newenvironment{spalte}{%
  \setlength\kastenwidth{1.2\textwidth}
  \divide\kastenwidth by \anzspalten
  \begin{minipage}[t]{\kastenwidth}}{\end{minipage}\hfill}

\renewcommand{\emph}[1]{{\color{red}\textbf{#1}}}
\renewcommand{\refname}{{\color{red}\underline{References}} } 

\begin{document}
%%%%%%%%%%%%%%%%%%%%%%%%%%%%%%%%%%%%%%%%%%%%%%%%%%%%
%%%               Background                     %%%             
%%%%%%%%%%%%%%%%%%%%%%%%%%%%%%%%%%%%%%%%%%%%%%%%%%%%
{\newrgbcolor{gradbegin}{0.4 0.4 1}%
  \newrgbcolor{gradend}{0.85 0.4 1}%{1 1 0.5}%
  \psframe[fillstyle=gradient,gradend=gradend,%
  gradbegin=gradbegin,gradmidpoint=0.1](\bgwidth,-\bgheight)}
%%%%%%%%%%%%%%%%%%%%%%%%%%%%%%%%%%%%%%%%%%%%%%%%%%%%
%%%                     Header                   %%%
%%%%%%%%%%%%%%%%%%%%%%%%%%%%%%%%%%%%%%%%%%%%%%%%%%%%
\begin{center}
\hspace{-0.1in} \psshadowbox{\makebox[0.46\textwidth]{
\parbox[c]{\linewidth}{
    \parbox[c]{1.0\linewidth}{
      \begin{center}
        \textbf{\Huge \color{TitleGreen}
{\  Perfect Powers in Lucas Sequences  \\  via Galois Representations}}\\
\vspace{.1in}
        \textsc{\large \color{blue} Jesse Silliman (Chicago) and Isabel Vogt (Harvard)}\\
\vspace{.1in}
        {\color{mybrown} Joint Mathematical Meetings,  \ Baltimore, Maryland}
      \end{center}}
}}}
\end{center}

\begin{lrbox}{\spalten}
  \parbox[t][\textheight]{1.3\textwidth}{%
    \hfill
%%%%%%%%%%%%%%%%%%%%%%%%%%%%%%%%%%%%%%%%%%%%%%%%%%%%
%%%               first column                   %%%             
%%%%%%%%%%%%%%%%%%%%%%%%%%%%%%%%%%%%%%%%%%%%%%%%%%%%
    \begin{spalte}
\vspace{-3.4in}
      \begin{kasten}
\section*{1 \hspace{0.1cm} {\color{red} \underline{Introduction}}}
\end{kasten}

\begin{kasten}
\subsection*{\color{blue} Abstract}
{
Let $u_n$ be a nondegenerate Lucas sequence.  We generalize the results of Bugeaud, Mignotte, and Siksek \cite{siksek06} to give a systematic approach towards the problem of determining all perfect powers in any particular Lucas sequence.  We then prove a general bound on admissible prime powers in a Lucas sequence assuming the Frey-Mazur conjecture on isomorphic mod $p$ Galois representations of elliptic curves.  Finally we develop a computationally efficient elementary sieve to conditionally determine all powers in several more example sequences.
} 
\end{kasten}

\begin{kasten}

\subsection*{\color{blue} Motivation/Previous Work}
In 2006, Bugeaud, Mignotte, and Siksek proved that
$F_0=0^p$, $F_1=1^p$, $F_6=2^3$, and $F_{12} = 12^2$ are the only perfect powers in the Fibonacci sequence.  Their proof relied upon the theory of elliptic curves and their associated Galois representations, especially the modularity theorem for elliptic curves over $\Q$.

\end{kasten}

\begin{kasten}
%\subsection*{\color{blue} Elliptic Curves}

%An elliptic curve is a genus 1 abelian variety with 1 marked point.  In characteristic $\neq 2,3$ every elliptic curve may be written as a plane cubic in Weierstrass form $E \colon y^2 = x^3 +ax +b$.

%\subsection*{\color{blue} Modular Forms}
%A modular form of weight $2$ on the group $\Gamma_0(N) \subset \SL_2(\Z)$ is a holomorphic function $f$ on the upper half plane $\H$ such that $f \big( \frac{az+b}{cz+d} \big) = (cz+d)^2 f(z) \ \text{for all} \ \begin{pmatrix} a & b \\ c & d \end{pmatrix} \in \Gamma_0(N)$, and is bounded at the cusps. 

\subsection*{\color{blue} Lucas Sequences}
A Lucas sequence $u_n$ is a nondegenerate integral linear binary recurrence relation defined by
%\vspace{-.1in}
\[u_{n+2} = bu_{n+1} + cu_n\]
%\vspace{-.1in}
with $u_0=0$ and $u_1 = 1$.  
The companion sequence $v_n$ is defined by $v_{n+2} = bv_{n+1} + cv_n$ with $v_0=2$ and $v_1 = b$.
A Lucas sequence $(b,c)$ has characteristic polynomial and roots
\[ g(z) = z^2 - bz - c, \qquad \qquad \alpha, \beta = \frac{b \pm \sqrt{b^2+4c}}{2},\]
and we may write
$u_n = \frac{\alpha^n - \beta^n}{\alpha - \beta}$ and $v_n = \alpha^n +\beta^n$, which gives the relations
\[(b^2+4c)u_n^2 = v_n^2 - 4(-c)^n\]

\end{kasten}

\begin{kasten}

\subsection*{ \color{blue} Frey Elliptic Curves}

Some explanation \\

Consider a hypothetical solution $u_n = y^p$ to the Lucas sequence $(3,-2)$, then $13y^{2p} + 2^{n+2} = v_n^2$.
Let $E$ be the Frey elliptic curve
\[ E: Y^2 + XY = X^3 + \left(\frac{w_n-1}{4} \right)X^2 + 2^{n-4}X \]
where $w_n = \pm v_n$ so that $w_n \equiv 1 \pmod{4}$, and $n \geq 5$.  Then we can compute the discriminant and conductor
\[ \Delta_E = 2^{2n-8}y^{2p} \qquad \qquad N_E = 2 \prod_{\ell | y} \ell. \]

\end{kasten}


\end{spalte}
%%%%%%%%%%%%%%%%%%%%%%%%%%%%%%%%%%%%%%%%%%%%%%%%%%%%
%%%               second column                   %%%             
%%%%%%%%%%%%%%%%%%%%%%%%%%%%%%%%%%%%%%%%%%%%%%%%%%%%
    \begin{spalte}
\begin{kasten}
\section*{3 \hspace{0.1cm} {\color{red} 
\underline{The Modular Method}}}
\end{kasten}

\begin{kasten}

\subsection*{ \color{blue} Galois Representations}

Stuff

\end{kasten}

\begin{kasten}

\subsection*{ \color{blue} Modularity of Elliptic Curves}

Stuff

\end{kasten}

\begin{kasten}

\subsection*{ \color{blue} Level Lowering}

Stuff

\end{kasten}

\begin{kasten}

\subsection*{ \color{blue} Example}

Prove that now pth powers for $p>7$ for previous example.

\end{kasten}

\end{spalte}
%%%%%%%%%%%%%%%%%%%%%%%%%%%%%%%%%%%%%%%%%%%%%%%%%%%%
%%%               third column                  %%%             
%%%%%%%%%%%%%%%%%%%%%%%%%%%%%%%%%%%%%%%%%%%%%%%%%%%%
    \begin{spalte}
\begin{kasten}
\section*{3 \hspace{0.1cm} {\color{red} 
\underline{Results}}}
\end{kasten}

\begin{kasten}

Theorem 1 and Theorem 2.

\end{kasten}



\begin{kasten}

\subsection*{\color{blue} Sketch of Proof of Theorem 1}

\end{kasten}

\begin{kasten}

\subsection*{\color{blue} Sketch of Proof of Theorem 2}

\end{kasten}

\end{spalte}
%%%%%%%%%%%%%%%%%%%%%%%%%%%%%%%%%%%%%%%%%%%%%%%%%%%%
%%%               fourth column                  %%%             
%%%%%%%%%%%%%%%%%%%%%%%%%%%%%%%%%%%%%%%%%%%%%%%%%%%%
    \begin{spalte}
\vspace{-3.4in}
\begin{kasten}
 \section*{5 \hspace{0.1cm} {\color{red} \underline{Conclusion
}}}
\end{kasten}



\begin{kasten}
  \subsection*{{\color{blue} \large Examples}}

\begin{citemize}
\item Some stuff.
\end{citemize}

\end{kasten}

\begin{kasten}
\subsection*{{\color{blue} \large Future Work}}
Text here.
\end{kasten}

\begin{kasten}
\subsection*{{\color{blue} Acknowledgements}}
I would like to thank \ldots
\begin{citemize}
\item Some people.
\end{citemize}
\end{kasten}

\begin{kasten}
         {\small
\begin{thebibliography}{99}
\setlength{\itemsep}{-2mm}

\bibitem{something} Works cited.

\end{thebibliography}
} % end \small
\end{kasten}

\end{spalte}
    }
    \end{lrbox}
\resizebox*{0.98\textwidth}{!}{%
  \usebox{\spalten}}\hfill\mbox{}\vfill
\end{document}
