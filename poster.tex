\documentclass[12pt]{scrartcl}
\usepackage{url}

\newenvironment{citemize}{
\begin{list}{$\bullet$}{\setlength{\itemsep}{0pt} \setlength{\rightmargin}{0pt} \setlength{\leftmargin}{0.5\labelwidth} \setlength{\topsep}{0pt}}
}{\end{list}}

%%
%
% 
% N.B. This format is cribbed from one obtained from the University
% of Karlsruhe, so some macro names and parameters are in German
% Here is a short glossary:
% Breite: width
% Hoehe: height
% Spalte: column
% Kasten: box
%
% All style files necessary are part of standard TeTeX distribution
%
% The recommended procedure is to first generate a ``Special Format" size poster
% file, which is relatively easy to manipulate and view.  It can be
% resized later to A0 (900 x 1100 mm) full poster size, or A4 or Letter size
% as desired (see web site).  Note the large format printers currently
% in use at UF's OIR have max width of about 90cm or 3 ft., but the paper
% comes in rolls so the length is variable.  See below the specifications
% for width and height of various formats.  Default in the template is
% ``Special Format",  with 4 columns.
%%
%% 
%% Choose your poster size:
%% For printing you will later RESIZE your poster by a factor
%%        2*sqrt(2) = 2.828    (for A0)
%%        2         = 2.00     (for A1) 
%%  
%% 

 \def\breite{355.6mm}
 \def\hoehe{321.733mm}
 \def\anzspalten{4}

%%
%% Procedure:
%%   Generate poster.dvi with latex
%%   Check with Ghostview
%%   Make a .ps-file with ``dvips -o poster.ps poster''
%%   Scale it with poster_resize poster.ps S
%%   where S is scale factor
%%     for Special Format->A0 S= 2.828 (= 2^(3/2)))
%%     for Special Format->A1 S= 2 (= 2^(2/2)))
%% 
%% Sizes (European:)
%%   A3: 29.73 X 42.04 cm
%%   A1: 59.5 X 84.1 cm
%%   A0: 84.1 X 118.9 cm
%%   N.B. The recommended procedure is ``Special Format x 2.82"
%%   which gives 90cm x 110cm (not quite A0 dimensions).
%%
%% --------------------------------------------------------------------------
%%
%% Load the necessary packages
%% 
\usepackage{palatino}
\usepackage[latin1]{inputenc}
\usepackage{epsf}
\usepackage{graphicx,psfrag,color,pst-grad}
\usepackage{amsmath,amssymb}
\usepackage{latexsym}
\usepackage{calc}
\usepackage{multicol}
\usepackage{amsthm}

\newtheorem*{define}{Definition}

%%
%% Define the required numbers, lengths and boxes 
%%
\newsavebox{\dummybox}
\newsavebox{\spalten}

\newlength{\bgwidth}\newlength{\bgheight}
\setlength\bgheight{\hoehe} \addtolength\bgheight{-1mm}
\setlength\bgwidth{\breite} \addtolength\bgwidth{-1mm}

\newlength{\kastenwidth}

%% Set paper format
\setlength\paperheight{\hoehe}                                             
\setlength\paperwidth{\breite}
\special{papersize=\breite,\hoehe}

\topmargin -1in
\marginparsep0mm
\marginparwidth0mm
\headheight0mm
\headsep0mm

%% Minimal Margins to Make Correct Bounding Box
\setlength{\oddsidemargin}{-2.47cm}
\addtolength{\topmargin}{-4mm}
\textwidth\paperwidth
\textheight\paperheight

\parindent0cm
\parskip1.5ex plus0.5ex minus 0.5ex
\pagestyle{empty}


\definecolor{MyBlue}{rgb}{0,0.08,0.45}
\definecolor{MyGreen}{rgb}{0,0.5,0.25}
\definecolor{TitleGreen}{rgb}{0,0.5,0.35}
\definecolor{recoilcolor}{rgb}{1,0,0}
\definecolor{occolor}{rgb}{0,1,0}
\definecolor{pink}{rgb}{0,1,1}
\definecolor{mybrown}{rgb}{0.6, 0.3, 0}

\def\UberStil{\normalfont\sffamily\bfseries\large}
\def\UnterStil{\normalfont\sffamily\small}
\def\LabelStil{\normalfont\sffamily\tiny}
\def\LegStil{\normalfont\sffamily\tiny}

%%
%% Define some commands
%%
\definecolor{JG}{rgb}{0.1,0.9,0.3}

\newenvironment{kasten}{%
  \begin{lrbox}{\dummybox}%
    \begin{minipage}{0.96\linewidth}}%
    {\end{minipage}%
  \end{lrbox}%
  \raisebox{-\depth}{\psshadowbox[framesep=1em]{\usebox{\dummybox}}}\\[0.5em]}
\newenvironment{spalte}{%
  \setlength\kastenwidth{1.2\textwidth}
  \divide\kastenwidth by \anzspalten
  \begin{minipage}[t]{\kastenwidth}}{\end{minipage}\hfill}

\renewcommand{\emph}[1]{{\color{red}\textbf{#1}}}
\renewcommand{\refname}{{\color{red}\underline{References}} } 

\begin{document}
%%%%%%%%%%%%%%%%%%%%%%%%%%%%%%%%%%%%%%%%%%%%%%%%%%%%
%%%               Background                     %%%             
%%%%%%%%%%%%%%%%%%%%%%%%%%%%%%%%%%%%%%%%%%%%%%%%%%%%
{\newrgbcolor{gradbegin}{0.4 0.4 1}%
  \newrgbcolor{gradend}{0.85 0.4 1}%{1 1 0.5}%
  \psframe[fillstyle=gradient,gradend=gradend,%
  gradbegin=gradbegin,gradmidpoint=0.1](\bgwidth,-\bgheight)}
%%%%%%%%%%%%%%%%%%%%%%%%%%%%%%%%%%%%%%%%%%%%%%%%%%%%
%%%                     Header                   %%%
%%%%%%%%%%%%%%%%%%%%%%%%%%%%%%%%%%%%%%%%%%%%%%%%%%%%
\begin{center}
\hspace{-0.1in} \psshadowbox{\makebox[0.46\textwidth]{
\parbox[c]{\linewidth}{
    \parbox[c]{1.0\linewidth}{
      \begin{center}
        \textbf{\Huge \color{TitleGreen}
{\  Perfect Powers in Lucas Sequences  \\  via Galois Representations}}\\
\vspace{.1in}
        \textsc{\large \color{blue} Jesse Silliman (Chicago) and Isabel Vogt (Harvard)}\\
\vspace{.1in}
        {\color{mybrown} Joint Mathematical Meetings,  \ Baltimore, Maryland}
      \end{center}}
}}}
\end{center}

\begin{lrbox}{\spalten}
  \parbox[t][\textheight]{1.3\textwidth}{%
    \hfill
%%%%%%%%%%%%%%%%%%%%%%%%%%%%%%%%%%%%%%%%%%%%%%%%%%%%
%%%               first column                   %%%             
%%%%%%%%%%%%%%%%%%%%%%%%%%%%%%%%%%%%%%%%%%%%%%%%%%%%
    \begin{spalte}
\vspace{-2.8in}
      \begin{kasten}
\section*{1 \hspace{0.1cm} {\color{red} \underline{Introduction}}}
\end{kasten}

\begin{kasten}
\subsection*{\color{blue} What am I Doing?}
{\color{blue}
Text here.
} % end of blue color.
\end{kasten}

\begin{kasten}
\subsection*{\color{blue} An Understandable Example}
Text here.
\end{kasten}

\end{spalte}
%%%%%%%%%%%%%%%%%%%%%%%%%%%%%%%%%%%%%%%%%%%%%%%%%%%%
%%%               second column                   %%%             
%%%%%%%%%%%%%%%%%%%%%%%%%%%%%%%%%%%%%%%%%%%%%%%%%%%%
    \begin{spalte}
\begin{kasten}
\section*{3 \hspace{0.1cm} {\color{red} 
\underline{Second Column}}}
\end{kasten}

\end{spalte}
%%%%%%%%%%%%%%%%%%%%%%%%%%%%%%%%%%%%%%%%%%%%%%%%%%%%
%%%               third column                  %%%             
%%%%%%%%%%%%%%%%%%%%%%%%%%%%%%%%%%%%%%%%%%%%%%%%%%%%
    \begin{spalte}
\begin{kasten}
\section*{3 \hspace{0.1cm} {\color{red} 
\underline{Third Column}}}
\end{kasten}

\end{spalte}
%%%%%%%%%%%%%%%%%%%%%%%%%%%%%%%%%%%%%%%%%%%%%%%%%%%%
%%%               fourth column                  %%%             
%%%%%%%%%%%%%%%%%%%%%%%%%%%%%%%%%%%%%%%%%%%%%%%%%%%%
    \begin{spalte}
\vspace{-2.8in}
\begin{kasten}
 \section*{5 \hspace{0.1cm} {\color{red} \underline{Conclusion
}}}
\end{kasten}

\begin{kasten}
\subsection*{{\color{blue} \large Sketch of Proof}}
Text here.
\end{kasten}

\begin{kasten}
  \subsection*{{\color{blue} \large Significance and Applications}}

\begin{citemize}
\item Some stuff.
\end{citemize}

\end{kasten}


\begin{kasten}
\subsection*{{\color{blue} Acknowledgements}}
I would like to thank \ldots
\begin{citemize}
\item Some people.
\end{citemize}
\end{kasten}

\begin{kasten}
         {\small
\begin{thebibliography}{99}
\setlength{\itemsep}{-2mm}

\bibitem{something} Works cited.

\end{thebibliography}
} % end \small
\end{kasten}

\end{spalte}
    }
    \end{lrbox}
\resizebox*{0.98\textwidth}{!}{%
  \usebox{\spalten}}\hfill\mbox{}\vfill
\end{document}
