\documentclass[12pt]{scrartcl}
\usepackage{url}

\newenvironment{citemize}{
\begin{list}{$\bullet$}{\setlength{\itemsep}{0pt} \setlength{\rightmargin}{0pt} \setlength{\leftmargin}{0.5\labelwidth} \setlength{\topsep}{0pt}}
}{\end{list}}

\def\B{{\mathbb B}}
\def\C{{\mathbb C}}
\def\D{{\mathbb D}}
\def\Fp{{\mathbb F}_p}
\def\Fell{{\mathbb F}_{\ell}}
\def\F{{\mathbb F}}
\def\H{{\mathbb H}}
\def\M{{\mathbb M}}
\def\N{{\mathcal N}}
\def\O{{\mathcal O}}
\def\0{{\mathbb 0}}
\def\P{{{\mathbb P}}}
\def\Q{{\mathbb Q}}
\def\R{{\mathbb R}}
\def\T{{\mathbb T}}
\def\Z{{\mathbb Z}}

\newcommand{\sol}{_{a^p,b^p,c^p}}
\newcommand{\bound}{\partial}
\newcommand{\la}[1]{\mathfrak{#1}}
\newcommand{\im}{\operatorname{Im}}
\newcommand{\ann}{\text{Ann} \hspace{0.1em} }
\newcommand{\rank}{\text{rank} \hspace{0.1em} }
\newcommand{\coker}[1]{\text{coker}\hspace{0.1em}{#1}}
\newcommand{\sgn}{\text{sgn}}
\newcommand{\lcm}{\text{lcm}}
\newcommand{\re}{\text{Re}  \hspace{0.1em} }
\newcommand{\ext}[1]{\text{Ext}(#1)}
\newcommand{\Hom}[1]{\text{Hom}(#1)}
\newcommand{\End}[1]{\text{End(#1)}}
\newcommand{\bs}{\setminus}
\newcommand{\rpp}[1]{\mathbb{R}\text{P}^{#1}}
\newcommand{\cpp}[1]{\mathbb{C}\text{P}^{#1}}
\newcommand{\tr}{\text{tr}\hspace{0.1em} }
\newcommand{\inner}[1]{\langle {#1}\rangle}
\newcommand{\tensor}{\otimes}
\newcommand{\Cl}{\text{Cl}}
\renewcommand{\sp}[1]{\text{Sp}_{#1}}
\newcommand{\GL}{\text{GL}}
\newcommand{\PGL}{\text{PGL}}
\newcommand{\SL}{\text{SL}}
\newcommand{\so}[1]{\text{SO}_{#1}}
\newcommand{\SO}{\text{SO}}
\newcommand{\pso}[1]{\text{PSO}_{#1}}
\renewcommand{\o}[1]{\text{O}_{#1}}
\renewcommand{\sp}[1]{\text{Sp}_{#1}}
\newcommand{\psp}[1]{\text{PSp}_{#1}}
\newcommand{\Span}{\rm Span}
\newcommand{\Frob}{\rm Frob}
\newcommand{\tor}{\rm tor}
\newcommand{\rad}{{\rm{rad}}}
\newcommand{\denom}{\rm denom}
\renewcommand{\bar}{\overline}
\newcommand{\notdiv}{\nmid}
\newcommand{\pfrac}[2]{\left( \frac{#1}{#2} \right)}
\newcommand{\bfrac}[2]{\left| \frac{#1}{#2} \right|}
\newcommand{\Ell}{\rm{Ell}}
\newcommand{\AV}{\rm{AV}}
\newcommand{\Gal}{\text{Gal}}
\newcommand{\ord}{{\rm{ord}}}

\newcommand{\p}{\mathfrak{p}}
\newcommand{\Nm}{\operatorname{Nm}}

\newcommand{\kron}[2]{\bigl(\frac{#1}{#2}\bigr)}
\newcommand{\leg}[2]{\Biggl(\frac{#1}{#2}\Biggr)}

\DeclareSymbolFont{bbold}{U}{bbold}{m}{n}

%%
%
% 
% N.B. This format is cribbed from one obtained from the University
% of Karlsruhe, so some macro names and parameters are in German
% Here is a short glossary:
% Breite: width
% Hoehe: height
% Spalte: column
% Kasten: box
%
% All style files necessary are part of standard TeTeX distribution
%
% The recommended procedure is to first generate a ``Special Format" size poster
% file, which is relatively easy to manipulate and view.  It can be
% resized later to A0 (900 x 1100 mm) full poster size, or A4 or Letter size
% as desired (see web site).  Note the large format printers currently
% in use at UF's OIR have max width of about 90cm or 3 ft., but the paper
% comes in rolls so the length is variable.  See below the specifications
% for width and height of various formats.  Default in the template is
% ``Special Format",  with 4 columns.
%%
%% 
%% Choose your poster size:
%% For printing you will later RESIZE your poster by a factor
%%        2*sqrt(2) = 2.828    (for A0)
%%        2         = 2.00     (for A1) 
%%  
%% 

 \def\breite{20in}
 \def\hoehe{15in}
 \def\anzspalten{4}

%%
%% Procedure:
%%   Generate poster.dvi with latex
%%   Check with Ghostview
%%   Make a .ps-file with ``dvips -o poster.ps poster''
%%   Scale it with poster_resize poster.ps S
%%   where S is scale factor
%%     for Special Format->A0 S= 2.828 (= 2^(3/2)))
%%     for Special Format->A1 S= 2 (= 2^(2/2)))
%% 
%% Sizes (European:)
%%   A3: 29.73 X 42.04 cm
%%   A1: 59.5 X 84.1 cm
%%   A0: 84.1 X 118.9 cm
%%   N.B. The recommended procedure is ``Special Format x 2.82"
%%   which gives 90cm x 110cm (not quite A0 dimensions).
%%
%% --------------------------------------------------------------------------
%%
%% Load the necessary packages
%% 
\usepackage{palatino}
\usepackage[latin1]{inputenc}
\usepackage{epsf}
\usepackage{graphicx,psfrag,color,pst-grad}
\usepackage{amsmath,amssymb}
\usepackage{latexsym}
\usepackage{calc}
\usepackage{multicol}
\usepackage{amsthm}

\newtheorem*{define}{Definition}

%%
%% Define the required numbers, lengths and boxes 
%%
\newsavebox{\dummybox}
\newsavebox{\spalten}

\newlength{\bgwidth}\newlength{\bgheight}
\setlength\bgheight{\hoehe} \addtolength\bgheight{-1mm}
\setlength\bgwidth{\breite} \addtolength\bgwidth{-1mm}

\newlength{\kastenwidth}

%% Set paper format
\setlength\paperheight{\hoehe}                                             
\setlength\paperwidth{\breite}
\special{papersize=\breite,\hoehe}

\topmargin -1in
\marginparsep0mm
\marginparwidth0mm
\headheight0mm
\headsep0mm

%% Minimal Margins to Make Correct Bounding Box
\setlength{\oddsidemargin}{-2.47cm}
\addtolength{\topmargin}{-4mm}
\textwidth\paperwidth
\textheight\paperheight

\parindent0cm
\parskip1.5ex plus0.5ex minus 0.5ex
\pagestyle{empty}


\definecolor{MyBlue}{rgb}{0,0.08,0.45}
\definecolor{MyGreen}{rgb}{0,0.5,0.25}
\definecolor{TitleGreen}{rgb}{0,0.5,0.35}
\definecolor{recoilcolor}{rgb}{1,0,0}
\definecolor{occolor}{rgb}{0,1,0}
\definecolor{pink}{rgb}{0,1,1}
\definecolor{mybrown}{rgb}{0.6, 0.3, 0}
\definecolor{mypurple}{rgb}{.3, 0, 1}

\def\UberStil{\normalfont\sffamily\bfseries\large}
\def\UnterStil{\normalfont\sffamily\small}
\def\LabelStil{\normalfont\sffamily\tiny}
\def\LegStil{\normalfont\sffamily\tiny}

%%
%% Define some commands
%%
\definecolor{JG}{rgb}{0.1,0.9,0.3}

\newenvironment{kasten}{%
  \begin{lrbox}{\dummybox}%
    \begin{minipage}{0.96\linewidth}}%
    {\end{minipage}%
  \end{lrbox}%
  \raisebox{-\depth}{\psshadowbox[framesep=1em]{\usebox{\dummybox}}}\\[0.5em]}
\newenvironment{spalte}{%
  \setlength\kastenwidth{1.2\textwidth}
  \divide\kastenwidth by \anzspalten
  \begin{minipage}[t]{\kastenwidth}}{\end{minipage}\hfill}

\renewcommand{\emph}[1]{{\color{red}\textbf{#1}}}
\renewcommand{\refname}{{\color{red}\underline{References}} } 

\begin{document}
%%%%%%%%%%%%%%%%%%%%%%%%%%%%%%%%%%%%%%%%%%%%%%%%%%%%
%%%               Background                     %%%             
%%%%%%%%%%%%%%%%%%%%%%%%%%%%%%%%%%%%%%%%%%%%%%%%%%%%
{\newrgbcolor{gradbegin}{0.4 0.4 1}%
  \newrgbcolor{gradend}{0.75 0.4 1}%{1 1 0.5}%
  \psframe[fillstyle=gradient,gradend=gradend,%
  gradbegin=gradbegin,gradmidpoint=0.1](\bgwidth,-\bgheight)}
%%%%%%%%%%%%%%%%%%%%%%%%%%%%%%%%%%%%%%%%%%%%%%%%%%%%
%%%                     Header                   %%%
%%%%%%%%%%%%%%%%%%%%%%%%%%%%%%%%%%%%%%%%%%%%%%%%%%%%
\begin{center}
\hspace{-0.1in} \psshadowbox{\makebox[0.46\textwidth]{
\parbox[c]{\linewidth}{
    \parbox[c]{1.0\linewidth}{
      \begin{center}
        \textbf{\Huge \color{mypurple}
{\  Perfect Powers in Lucas Sequences  \\  via Galois Representations}}\\
\vspace{.1in}
        \textsc{\large \color{blue} Jesse Silliman (Chicago) and Isabel Vogt (Harvard)}\\
\vspace{.1in}
        { Joint Mathematical Meetings,  \ Baltimore, Maryland}
      \end{center}}
}}}
\end{center}

\begin{lrbox}{\spalten}
  \parbox[t][\textheight]{1.3\textwidth}{%
    \hfill
%%%%%%%%%%%%%%%%%%%%%%%%%%%%%%%%%%%%%%%%%%%%%%%%%%%%
%%%               first column                   %%%             
%%%%%%%%%%%%%%%%%%%%%%%%%%%%%%%%%%%%%%%%%%%%%%%%%%%%
    \begin{spalte}
\vspace{-2.81in}
      \begin{kasten}
\section*{1 \hspace{0.1cm} {\color{red} \underline{Introduction}}}
\end{kasten}

\begin{kasten}
\subsection*{\color{blue} Abstract}
{
%Let $u_n$ be a nondegenerate Lucas sequence.  
We generalize the results of Bugeaud, Mignotte, and Siksek \cite{siksek06} to give a systematic approach towards the problem of determining all perfect powers in any particular Lucas sequence.  We then prove a general bound on admissible prime powers in a Lucas sequence assuming the Frey-Mazur conjecture on isomorphic mod $p$ Galois representations of elliptic curves.  Finally we develop a computationally efficient elementary sieve to conditionally determine all powers in several more example sequences.
} 
\end{kasten}

\begin{kasten}

\subsection*{\color{blue} Motivation/Previous Work}
In 2006, Bugeaud, Mignotte, and Siksek proved that
$F_0=0^p$, $F_1=1^p$, $F_6=2^3$, and $F_{12} = 12^2$ are the only perfect powers in the Fibonacci sequence.  Their proof relied upon the theory of elliptic curves and their associated Galois representations, especially the modularity theorem for elliptic curves over $\Q$.

\end{kasten}

\begin{kasten}
%\subsection*{\color{blue} Elliptic Curves}

%An elliptic curve is a genus 1 abelian variety with 1 marked point.  In characteristic $\neq 2,3$ every elliptic curve may be written as a plane cubic in Weierstrass form $E \colon y^2 = x^3 +ax +b$.

%\subsection*{\color{blue} Modular Forms}
%A modular form of weight $2$ on the group $\Gamma_0(N) \subset \SL_2(\Z)$ is a holomorphic function $f$ on the upper half plane $\H$ such that $f \big( \frac{az+b}{cz+d} \big) = (cz+d)^2 f(z) \ \text{for all} \ \begin{pmatrix} a & b \\ c & d \end{pmatrix} \in \Gamma_0(N)$, and is bounded at the cusps. 

\subsection*{\color{blue} Lucas Sequences}
A Lucas sequence $u_n$ is a nondegenerate integral linear binary recurrence relation defined by
%\vspace{-.1in}
\[u_{n+2} = bu_{n+1} + cu_n\]
%\vspace{-.1in}
with $u_0=0$ and $u_1 = 1$.  
The companion sequence $v_n$ is defined by $v_{n+2} = bv_{n+1} + cv_n$ with $v_0=2$ and $v_1 = b$.
A Lucas sequence $(b,c)$ has characteristic polynomial and roots
\[ g(z) = z^2 - bz - c, \qquad \qquad \alpha, \beta = \frac{b \pm \sqrt{b^2+4c}}{2},\]
and we may write
$u_n = \frac{\alpha^n - \beta^n}{\alpha - \beta}$ and $v_n = \alpha^n +\beta^n$, which gives the relations
\[(b^2+4c)u_n^2 = v_n^2 - 4(-c)^n\]

\end{kasten}

\begin{kasten}

\subsection*{\color{blue} Finding Perfect Powers}

Fix a Lucas sequence $u_n$.  A perfect power is a solution to the Diophantine equation $u_n = y^p$ for some integers $y>1$ and $p$.  By factoring $p$, it suffices to consider prime exponents.  This naturally breaks into 3 steps:

\begin{enumerate}

\item Bound the exponent $p$ that can appear

\item Bound the index $n \leq B(p)$ in terms of $p$

\item Check all terms of index less than $B(p)$ and determine if they are $p$th powers

\end{enumerate}
The last step, though nontrivial, is possible via the following elementary algorithm:
\subsubsection*{\color{red} Sieve for $p$th Powers}

\begin{citemize}

\item Look at $u_n$ mod $q$ for $q \equiv 1 \pmod{p}$, i.e. $p \mid \#\F_q^\times$

\item There should be $1/p$ $p$th powers mod $q$



\item This gives nontrivial congruence data on $n$ mod the period of $u_n$



\item Do this for many such primes and combine via the CRT


\item The modulus $M$ grows, as do some congruences $\bar{a}$ in the list
\[ \bar{n} \in \{0,1,...,\bar{a}_1,...,\bar{a}_m\} \subseteq \Z/M\Z \]


\item If $\bar{a}_1 > B(p) \Rightarrow $ the fixed congruences are a complete list

\end{citemize}

%The bounds in question in the second step are effective known due to \cite{petho82} and \cite{shorey83}, and can sometimes be computed explicitly using the theory of Thue equations.  For that reason, 

The bound $B(p)$ can come from the theory of Thue equations.  Here we focus on answering the first question by demonstrating an explicit bound on $p$ depending on $b$ and $c$.

\end{kasten}

%\begin{kasten}

%\subsection*{\color{blue} Elliptic Curves}
%***I AM NOT SURE WE NEED TO DEFINE THIS***
%An elliptic curve is a genus 1 smooth projective curve with 1 marked point.  Every elliptic curve may be written as a plane cubic in Weierstrass form $E \colon y^2 + a_1xy +a_3y = x^3 +a_2x^2 + a_4x +a_6$.  \\

%***DEFINE REDUCTION PERHAPS?***
%The minimal discriminant and conductor of an elliptic curve measure the bad reduction at primes and can be computed using Tate's algorithm \cite{tate}.

%***WE COULD MENTION TATE CURVES***
%***WE COULD MENTION MODULAR CURVES***
%\subsection*{\color{blue} Modular Forms}
%***DO WE ACTUALLY NEED MODULAR FORMS?***
%A modular form of weight $2$ on the group $\Gamma_0(N) \subset \SL_2(\Z)$ is a holomorphic function $f$ on the upper half plane $\H$ such that $f \big( \frac{az+b}{cz+d} \big) = (cz+d)^2 f(z) \ \text{for all} \ \begin{pmatrix} a & b \\ c & d \end{pmatrix} \in \Gamma_0(N)$, and is bounded at the cusps for the action of $\Gamma_0(N)$ on $\H$. 



\begin{kasten}
\section*{2 \hspace{0.1cm} {\color{red} 
\underline{The Modular Method}}}
\end{kasten}

\begin{kasten}

\subsection*{\color{blue} Overview}

\begin{citemize}

\item Given an alleged perfect power $u_n = y^p$, we first associate an elliptic curve whose coefficients depend in a crucial way upon $y$, $n$, and $p$.  Such an elliptic curve is called a Frey curve.  It has conductor $N$, which depends upon the input $y$.

\item To this elliptic curve is associated a mod $p$ Galois representation coming from the Galois action on the $p$-torsion points.

\item By the Modularity Theorem, this mod $p$ Galois representation is isomorphic to one coming from a weight $2$ newform of level $N$.

\item This representation level-lowers to an isomorphic representation of a newform of level $N'$, where now $N'$ \emph{does not} depend upon $y$.

\item Then use arithmetic information about newforms at level $N'$ to derive a contradiction or congruence conditions.

\end{citemize}

\end{kasten}



\end{spalte}
%%%%%%%%%%%%%%%%%%%%%%%%%%%%%%%%%%%%%%%%%%%%%%%%%%%%
%%%               second column                   %%%             
%%%%%%%%%%%%%%%%%%%%%%%%%%%%%%%%%%%%%%%%%%%%%%%%%%%%
    \begin{spalte}


\begin{kasten}

\subsection*{\color{blue} Frey Elliptic Curves}

A Frey curve is an elliptic curve built out of a hypothetical solution to a Diophantine problem, whose minimal discriminant will be of the form $C\cdot D^p$ where $C$ does not depend upon the hypothetical solution. The shape of the discriminant is crucial: it allows us to produce hypothetical modular forms whose level is independent of the hypothetical solution.\\

For example, consider a hypothetical solution $u_n = y^p$ to the Lucas sequence $(3,-2)$, then $13y^{2p} + 2^{n+2} = v_n^2$.
Let $E$ be the Frey elliptic curve
\[ E: Y^2 + XY = X^3 + \left(\frac{w_n-1}{4} \right)X^2 + 2^{n-4}X \]
where $w_n = \pm v_n$ so that $w_n \equiv 1 \pmod{4}$, and $n \geq 5$.  Then we can compute the discriminant and conductor
\[ \Delta_E = 2^{2n-8}y^{2p} \qquad \qquad N_E = 2 \prod_{\ell | y} \ell. \]

\end{kasten}

\begin{kasten}

\subsection*{\color{blue} Galois Representations}
%We can obtain Galois representations from both elliptic curves and modular forms.

For $E$ an elliptic curve and $p$ a prime, the absolute $p$-torsion points $E[p] \simeq \F_p \times \F_p$.
The Galois action of $G_\Q = \Gal(\bar{\Q}/\Q)$ on $E[p]$ gives a linear representation of the Galois group 
\[ \rho_{E,p} \colon G_\Q \rightarrow \GL_2(\F_p). \]
It is ramified at a prime $\ell$ iff $\ell$ ramifies in $\Q(E[p])$.  

Conversely, given a weight $2$ newform $f$, Eichler-Shimura theory associates an abelian variety $A_f$, from which we obtain a Galois representation $\rho_{f,p}$.

\end{kasten}

\begin{kasten}

\subsection*{\color{blue} Tate Curves}
We can understand the ramification of $\rho_{E,p}$ in terms of the discriminant $\Delta_E$. When $E$ is semistable at $\ell$, there is a Galois isomorphism (over an, at most quadratic, unramified extension $K$ of $\Q_{\ell}$) $E(\bar{\Q_{\ell}}) \cong \bar{\Q_{\ell}}^*/q^\Z$, for some $q \in K$. This is an exponentiated analogy to the isomorphism $E(\C) \cong \C/\Lambda$. 

From this, we can compute $E(\bar{\Q_{\ell}})[p] \cong \zeta_p \oplus q^{1/p}$. For $\ell \neq p$, this implies that $\rho_{E,p}$ ramifies at $\ell$ iff the extension $\Q_{\ell}(q^{1/p})/\Q_{\ell}$ is ramified. This relates ramification to the discriminant via the formula $\Delta_E = q\prod_{n \geq 1} (1-q^n)^{24}$.
\end{kasten}




\begin{kasten}

\subsection*{\color{blue} Modularity of Elliptic Curves}

Let $E$ be an elliptic curve over $\Q$ with conductor $N$.  For any prime $p$, there exists a weight 2 newform $f \in S_2(\Gamma_0(N))$ such that
%\[ \rho_{E,p} \simeq \rho_{f,p}. \]
\[ \rho_{E,p^{\infty}} \simeq \rho_{f,p^{\infty}},\]
for $\rho_{E,p^\infty}$ and $\rho_{f,p^\infty}$ the $p$-adic representations. 

\subsection*{\color{blue} Level Lowering}

Let $f$ be a newform of level $\ell N$ with absolutely irreducible 2-dimensional mod $p$ Galois representation $\rho_{f,p}$ unramified at $\ell$ if $\ell \neq p$ and finite flat if $\ell = p$. 

 Then there exists a weight 2 newform $g$ of level $N$ such that
 \vspace{-10pt}
\[ \rho_{f,p} \simeq \rho_{g,p}. \]

\end{kasten}


%\begin{kasten}

%\subsection*{ \color{blue} Example}

%We can use the modular method to show that there are no $p$th powers in the Lucas sequence $(3,-2)$ for $p\geq 5$.  Let $E$ be the Frey elliptic curve in the above example with $\Delta_E = 2^{2n-8}y^{2p}$ and $N_E = 2 \prod_{\ell | y} \ell$.

%One can show that $\rho_{E,p}$ is absolutely irreducible for $p \geq 5$ by studying the modular curves $X_0(p)$ and $X_0(2p)$.  Further, as all primes $\ell$ dividing $y$ appear to an $\ell$th power in the discriminant, a Tate curve argument shows that $\rho_{E,p}$ is unramified at $\ell\neq p$ and finite flate at $\ell = p$.

%Therefore the Modularity Theorem and Level Lowering guarantee that if this hypothetical Frey curve exists, then there also exists a newform $g$ of level $2$ such that $\rho_{E,p} \simeq \rho_{f, p}$.  But the are no newforms of level $2$, which is a contradiction.

%\end{kasten}


%\begin{kasten}
%\section*{3 \hspace{0.1cm} {\color{red} 
%\underline{Results}}}
%\end{kasten}


\begin{kasten}
 \section*{5 \hspace{0.1cm} {\color{red} \underline{Results
}}}
\end{kasten}

\begin{kasten}

We find all perfect powers in several example sequences, and prove a conditional bound on the admissible prime exponents. \\

\textbf{\large \color{blue} Theorem 1.} \textit{For the following values of $b$ and $c$:
\[ (b,c) = (3,-2), (5,-6), (7,-12), (17,-72), (9,-20) \]
$u_n$ has no nontrivial $p$th powers, except $u_2 = 3^2$ in $(9,-20)$.} \\

\textbf{\large\color{blue} Theorem 2.} \textit{Consider a solution $u_n = y^p$, with $b^2+4c >0$. Let \[ N = 2^8  {\rad}'(c)\rad'(b^2+4c),\]
where $\rad'$ denotes the product over odd prime divisors.  Then, assuming the Frey-Mazur Conjecture. 
\[ p \leq \max\left\{17,   \psi(N)^{(\psi(N)/12+1)}, 4\log{|\alpha|} \cdot  \max\{30,( N+1)\}  \right\}. \]}

\end{kasten}




\end{spalte}
%%%%%%%%%%%%%%%%%%%%%%%%%%%%%%%%%%%%%%%%%%%%%%%%%%%%
%%%               third column                  %%%             
%%%%%%%%%%%%%%%%%%%%%%%%%%%%%%%%%%%%%%%%%%%%%%%%%%%%
    \begin{spalte}


\begin{kasten}
\section*{4 \hspace{0.1cm} {\color{red} 
\underline{Techniques}}}
\end{kasten}

\begin{kasten}

\subsection*{\color{blue} Overview}

\begin{citemize}
\item When the modular method does not work as smoothly as in the examples in Theorem 1, we break our problem into two cases
\begin{enumerate}

\item $E$ level lowers to an elliptic curve

\item $E$ level lowers to a higher dimensional abelian variety

\end{enumerate}

\item In the second case, the irrationality of the associated newform places nontrivial congruence conditions on its coefficients mod $p$ which can be used to bound $p$.

\item In the first case, we assume the Frey-Mazur conjecture, which gives that the level-lowering was actually trivial and forces $y$ to be smooth by a bounded set of primes.  We then use a theorem of Gyory on Lucas sequences representing smooth numbers ot bound $n$ and hence $p$.

\end{citemize}

\end{kasten}

\begin{kasten}

\subsection*{ \color{blue} Congruence Conditions for Isomorphism}
%The condition that two mod $p$ Galois representations be isomorphic imposes congruence constraints on the Fourier coefficients of the newforms.

If $g$ is a rational newform (coresponding to an elliptic curve) and coefficients $a_\ell$ and conductor $N_g$, and $f$ is the level-lowered newform with coefficients $c_\ell$ and conductor $N_f$, then
\begin{align*}
c_\ell &\equiv a_\ell \pmod{\p} & \text{ if $\ell \not\mid pN_fN_g$} \\
c_\ell &\equiv \pm(\ell+1) \pmod{\p} & \text{ if $\ell \| N_g, \ell \not\mid pN_f$},
\end{align*}
where $\p$ is a prime of the field $K$ generated by the $c_\ell$ above rational $p$. \\

If $f$ is an irrational newform, then being integral modulo some prime is a nontrivial condition.  Further the $a_\ell$ are restricted to be in the Hasse interval $[-2\sqrt{\ell}, 2\sqrt{\ell}]$.  So for $c_\ell \not\in \Z$, we have
\[ p\, \Big| \Nm_{K/\Q}\left(c_\ell^2 - (\ell+1)^2\right) \cdot \prod_{r \text{ in Hasse} \atop \text{interval}} \Nm_{K/\Q} \left( c_\ell - r \right). \]


\end{kasten}

\begin{kasten}

\subsection*{\color{blue} Bounds on the Frey Curve Conductor}

First we compute all possible Frey elliptic curves for this Diophantine problem and find that in all cases,
\[N_E \leq 2^8 \cdot \rad'(c) \cdot \rad'((b^2+4c)\cdot y).\]


\subsection*{\color{blue} Sturm's Bound}

If two weight 2 modular forms $f$ and $g$ on $\Gamma_0(N)$ are not equal, then Sturm's bound gives that they must differ in a Fourier coefficient whose index $n$ is bounded by 
\[n \leq \frac{\psi(N)}{6}.\] 

This essentially follows from a computing of $\dim S_2(\Gamma_0(N)) = \genus(X_0(N))$.
\end{kasten}

\begin{kasten}

\subsection*{ \color{blue} The Frey-Mazur Conjecture}

The Frey-Mazur Conjecture states that for $p$ large enough, two elliptic curves with isomorphic mod $p$ Galois representations must be isogenous. In particular if $E$ and $E'$ are nonisogenous elliptic curves over $\Q$, and $\rho_{E,p} \cong \rho_{E',p}$, then $p\leq 17$.

This conjecture can be motivated geometrically. It can be shown that there exists a surface $Z(p)$ whose rational points correspond to pairs of elliptic curves $(E,E')$ over $\Q$ such that $\rho_{E,p} \cong \rho_{E',p}$. Outside of lines and elliptic curves along which $E$ is isogenous to $E'$, it parametrizes counterexamples to Frey-Mazur. The Bombieri-Lang conjecture predicts that almost all points on varieties of general type (which $Z(p)$ is, for $p >> 0$) lie on lines and elliptic curves in $Z(p)$.

An analog of this conjecture for function fields was recently proven by Bakker and Tsimerman \cite{bakker01}.
%***HOW DO WE STATE IT IN TERMS OF MODULAR CURVES?***<-- is this necessary?
% Not necessary, but it helps to motivate why the conjecture is expected to be true.

\end{kasten}

\begin{kasten}

\subsection*{ \color{blue} Lucas Sequences Representing Smooth Numbers}

Let $S$ be the set of all integers whose prime factors lie in some finite set $\{p_1,p_2,...,p_m\}$ with $p_m \geq p_i$ for all $i$.  Let $u_n$ be a Lucas sequence; if the $n$th term $u_n \in S$ then
\[ n \leq \max\{30, p_m +1 \}. \]

\end{kasten}




\end{spalte}
%%%%%%%%%%%%%%%%%%%%%%%%%%%%%%%%%%%%%%%%%%%%%%%%%%%%
%%%               fourth column                  %%%             
%%%%%%%%%%%%%%%%%%%%%%%%%%%%%%%%%%%%%%%%%%%%%%%%%%%%
    \begin{spalte}
\vspace{-2.81in}

\begin{kasten}
\section*{5 \hspace{0.1cm} {\color{red} 
\underline{Conclusion}}}
\end{kasten}

\begin{kasten}

\subsection*{\color{blue} Sketch of Proof of Theorem 1}

\begin{citemize}

\item Using the fact that $u_{2k} = u_kv_k$, one may reduce to the case of odd index.  Then using the solution to the $(k,k,2)$ and $(k,k, 3)$ equations, finding all of the squares and cubes is a finite computation.

\item All conductors of the associated Frey curves are of the form $N = \rad(c \cdot y)$.

\item Using a Tate curve argument based upon the fact that all primes dividing $y$ appear to $p$th power in the discriminant, it follows that $\rho_{E,p}$ is unramified for $\ell \notdiv c$, and finite flat at $p$.

\item A modular curve argument implies that $\rho_{E,p}$ is absolutely irreducible for $p \geq 5$.

\item Combining these, level lowering shows that $\rho_{E,p} \simeq \rho_{f,p}$ for $f$ of level $N = \rad(c) = 2,6,10$

\item There are no newforms at those levels \qed

\end{citemize}

\end{kasten}

\begin{kasten}

\subsection*{\color{blue} Sketch of Proof of Theorem 2}

\begin{citemize}

\item Let $p$ be a prime such that $E$ level lowers mod $p$ to an elliptic curve $F$.  Then either $p \leq 17$, or
\[ N_E = 2^\gamma \cdot \rad'(c \cdot (b^2+4c)\cdot y) = 2^\gamma \cdot \rad'(c \cdot (b^2+4c)) = N_F, \]
as the elliptic curves are isogenous.

\item In this case $\rad(y) \mid \rad(2c \cdot (b^2+4c))$, so the term $u_n = y^p$ is smooth by the primes dividing $\rad{N} = 2\rad{c\cdot (b^2+4c)}$, and hence $n < \max\{30, N+1\}$.  And $\log y^p \approx \log(n \alpha^{n-1})$ so
\[p \leq 4 \log |\alpha| \cdot \max\{30, N+1\}. \]

\item Otherwise it level-lowers to a higher dimensional abelian variety, and the congruence conditions give
\[ p \leq \Nm_{K/\Q}(2\sqrt{\ell} +\ell+1) \leq (2\sqrt{\ell} +\ell+1)^{\operatorname{deg}{K}}.\]

\item Further, Sturm's Bound gives that $\ell \leq \psi(N)/6$, as $f$ is not equal to its Galois conjugates.  Further the degree of $K$ is bounded by $1+\psi(N)/12$.  So
\[p \leq \psi(N)^{(\psi(N)/12+1)}.\]
\qed
\end{citemize}

\end{kasten}



%\begin{kasten}
 % \subsection*{{\color{blue} \large Examples}}

%\begin{citemize}
%\item Some stuff.
%\end{citemize}

%We also give an algorithm for finding all $p$th powers up to a bound on the index $B(p)$.  This can come from the theory of Thue equations.



%\subsubsection*{\color{red} Conditional Examples}

%Assuming the Frey-Mazur Conjecture, there are no nontrivial $p$th powers in the Lucas sequences $(b,c) = (3,1)$, $(5,1)$, and $(7,1)$.

%\end{kasten}

%\begin{kasten}
%\subsection*{{\color{blue} \large Future Work}}
%Text here.
%\end{kasten}

\begin{kasten}
\subsection*{{\color{blue} Acknowledgements}}
We would like to thank \ldots
\begin{citemize}
\item Professors Ken Ono and David Zureick-Brown suggesting the topic and advising us throughout the project

\item Eric Larson and other members of the 2013 Emory REU for useful discussions

\item Professor Barry Mazur for reading a draft of the paper and Professor Samir Siksek for answering questions

\item The NSF for their support
\end{citemize}
\end{kasten}

\begin{kasten}
         {\small
\bibliographystyle{plain}
\begin{thebibliography}{10}

\bibitem{bakker01}
Benjamin Bakker and Jacob Tsimerman.
\newblock On the Frey-Mazur conjecture over low genus curves
\newblock {\em ArXiv e-prints}

%\bibitem{bennett05}
%Michael~A. Bennett.
%\newblock Powers in recurrence sequences: {P}ell equations.
%\newblock {\em Trans. Amer. Math. Soc.}, 357(4):1675--1691 (electronic), 2005.

\bibitem{bennett04}
Michael~A. Bennett and Chris~M. Skinner.
\newblock Ternary {D}iophantine equations via {G}alois representations and
  modular forms.
\newblock {\em Canad. J. Math.}, 56(1):23--54, 2004.

\bibitem{conrad01}
Christophe Breuil, Brian Conrad, Fred Diamond, and Richard Taylor.
\newblock On the modularity of elliptic curves over {$\bold Q$}: wild 3-adic
  exercises.
\newblock {\em J. Amer. Math. Soc.}, 14(4):843--939 (electronic), 2001.

\bibitem{brown12}
David Brown.
\newblock Primitive integral solutions to {$x^2+y^3=z^{10}$}.
\newblock {\em Int. Math. Res. Not. IMRN}, (2):423--436, 2012.

%\bibitem{bugeaud96}
%Yann Bugeaud and K{\'a}lm{\'a}n Gyory.
%\newblock Bounds for the solutions of {T}hue-{M}ahler equations and norm form
%  equations.
%\newblock {\em Acta Arith.}, 74(3):273--292, 1996.

\bibitem{siksek06}
Yann Bugeaud, Maurice Mignotte, and Samir Siksek.
\newblock Classical and modular approaches to exponential {D}iophantine
  equations. {I}. {F}ibonacci and {L}ucas perfect powers.
\newblock {\em Ann. of Math. (2)}, 163(3):969--1018, 2006.

%\bibitem{cohen07}
%Henri Cohen.
%\newblock {\em Number theory. {V}ol. {II}. {A}nalytic and modern tools}, volume
%  240 of {\em Graduate Texts in Mathematics}.
%\newblock Springer, New York, 2007.

%\bibitem{darmon97}
%Henri Darmon and Lo{\"{\i}}c Merel.
%\newblock Winding quotients and some variants of {F}ermat's last theorem.
%\newblock {\em J. Reine Angew. Math.}, 490:81--100, 1997.

%\bibitem{gyory82}
%K.~Gyory.
%\newblock On some arithmetical properties of {L}ucas and {L}ehmer numbers.
%\newblock {\em Acta Arith.}, 40(4):369--373, 1981/82.

%\bibitem{gyory81}
%K.~Gyory, P.~Kiss, and A.~Schinzel.
%\newblock On {L}ucas and {L}ehmer sequences and their applications to
%  {D}iophantine equations.
%\newblock {\em Colloq. Math.}, 45(1):75--80 (1982), 1981.

\bibitem{gyory03}
K{\'a}lm{\'a}n Gyory.
\newblock On some arithmetical properties of {L}ucas and {L}ehmer numbers.
  {II}.
\newblock {\em Acta Acad. Paedagog. Agriensis Sect. Mat. (N.S.)}, 30:67--73,
  2003.
\newblock Dedicated to the memory of Professor Dr. P{\'e}ter Kiss.

\bibitem{mazur78}
B.~Mazur.
\newblock Rational isogenies of prime degree (with an appendix by {D}.
  {G}oldfeld).
\newblock {\em Invent. Math.}, 44(2):129--162, 1978.

%\bibitem{mih04}
%Preda Mih{\u{a}}ilescu.
%\newblock Primary cyclotomic units and a proof of {C}atalan's conjecture.
%\newblock {\em J. Reine Angew. Math.}, 572:167--195, 2004.

%\bibitem{petho92}
%A.~Petho.
%\newblock The {P}ell sequence contains only trivial perfect powers.
%\newblock In {\em Sets, graphs and numbers ({B}udapest, 1991)}, volume~60 of
%  {\em Colloq. Math. Soc. J\'anos Bolyai}, pages 561--568. North-Holland,
%  Amsterdam, 1992.

\bibitem{petho82}
Attila Petho.
\newblock Perfect powers in second order linear recurrences.
\newblock {\em J. Number Theory}, 15(1):5--13, 1982.

\bibitem{ribet91}
Kenneth~A. Ribet.
\newblock Lowering the levels of modular representations without multiplicity
  one.
\newblock {\em Internat. Math. Res. Notices}, (2):15--19, 1991.

%\bibitem{robbins83}
%Neville Robbins.
%\newblock On {F}ibonacci numbers which are powers. {II}.
%\newblock {\em Fibonacci Quart.}, 21(3):215--218, 1983.

\bibitem{shorey83}
T.~N. Shorey and C.~L. Stewart.
\newblock On the {D}iophantine equation {$ax^{2t}+bx^{t}y+cy^{2}=d$} and pure
  powers in recurrence sequences.
\newblock {\em Math. Scand.}, 52(1):24--36, 1983.

%\bibitem{code}
%Jesse Silliman and Isabel Vogt.
%\newblock Sage code for finding perfect powers in recurrence sequences.
%\newblock Available online at
%  \url{http://people.fas.harvard.edu/~ivogt/perfect_powers_code.sage}.

%\bibitem{stein07}
%William Stein.
%\newblock {\em Modular forms, a computational approach}, volume~79 of {\em
%  Graduate Studies in Mathematics}.
%\newblock American Mathematical Society, Providence, RI, 2007.
%\newblock With an appendix by Paul E. Gunnells.

\bibitem{taylorwiles95}
Richard Taylor and Andrew Wiles.
\newblock Ring-theoretic properties of certain {H}ecke algebras.
\newblock {\em Ann. of Math. (2)}, 141(3):553--572, 1995.

\bibitem{wiles95}
Andrew Wiles.
\newblock Modular elliptic curves and {F}ermat's last theorem.
\newblock {\em Ann. of Math. (2)}, 141(3):443--551, 1995.


\end{thebibliography}
} % end \small
\end{kasten}

\end{spalte}
    }
    \end{lrbox}
\resizebox*{0.98\textwidth}{!}{%
  \usebox{\spalten}}\hfill\mbox{}\vfill
\end{document}
