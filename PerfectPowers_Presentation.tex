\documentclass[pdf]{beamer}
\mode<presentation>{\usetheme{Warsaw}}
\usecolortheme[RGB={100,90,150}]{structure}

\renewcommand{\emph}[1]{{\color{red}\textbf{#1}}}
\newcommand{\tw}{\textwidth}
\usefonttheme{serif}
\usepackage{tikz}
%\usepackage{movie15}
%\usepackage{multimedia}
\usetikzlibrary{decorations.pathreplacing}
\usetikzlibrary{arrows,chains,matrix,positioning,scopes}

\makeatletter
\tikzset{join/.code=\tikzset{after node path={%
\ifx\tikzchainprevious\pgfutil@empty\else(\tikzchainprevious)%
edge[every join]#1(\tikzchaincurrent)\fi}}}
\makeatother
%
\tikzset{>=stealth',every on chain/.append style={join},
         every join/.style={->}}
\tikzstyle{labeled}=[execute at begin node=$\scriptstyle,
   execute at end node=$]
\usetikzlibrary{patterns}

\def\A{{\mathcal A}}
\def\B{{\mathbb B}}
\def\C{{\mathbb C}}
\def\D{{\mathbb D}}
\def\F{{\mathbb F}}
\def\H{{\mathbb H}}
\def\M{{\mathbb M}}
\def\N{{\mathbb N}}
\def\0{{\mathbb 0}}
\def\P{{{\mathbb P}}}
\def\Q{{\mathbb Q}}
\def\R{{\mathbb R}}
\def\T{{\mathbb T}}
\def\Z{{\mathbb Z}}

\newcommand{\bound}{\partial}
\newcommand{\la}[1]{\mathfrak{#1}}
\newcommand{\im}{\text{Im} \hspace{0.1em} }
\newcommand{\ann}{\text{Ann} \hspace{0.1em} }
\newcommand{\rank}{\text{rank} \hspace{0.1em} }
\newcommand{\coker}[1]{\text{coker}\hspace{0.1em}{#1}}
\newcommand{\sgn}{\text{sgn}}
\newcommand{\lcm}{\text{lcm}}
\newcommand{\re}{\text{Re}  \hspace{0.1em} }
\newcommand{\ext}[1]{\text{Ext}(#1)}
\newcommand{\Hom}[1]{\text{Hom}(#1)}
\newcommand{\End}[1]{\text{End(#1)}}
\newcommand{\bs}{\setminus}
\newcommand{\rpp}[1]{\mathbb{R}\text{P}^{#1}}
\newcommand{\cpp}[1]{\mathbb{C}\text{P}^{#1}}
\newcommand{\tr}{\text{tr}\hspace{0.1em} }
\newcommand{\inner}[1]{\langle {#1}\rangle}
\newcommand{\tensor}{\otimes}
\newcommand{\sq}{\text{Sq}}
\renewcommand{\sp}[1]{\text{Sp}_{#1}}
\newcommand{\gl}[1]{\text{GL}_{#1}}
\newcommand{\pgl}[1]{\text{PGL}_{#1}}
\renewcommand{\sl}[1]{\text{SL}_{#1}}
\newcommand{\so}[1]{\text{SO}_{#1}}
\newcommand{\SO}{\text{SO}}
\newcommand{\pso}[1]{\text{PSO}_{#1}}
\renewcommand{\o}[1]{\text{O}_{#1}}
\renewcommand{\sp}[1]{\text{Sp}_{#1}}
\newcommand{\psp}[1]{\text{PSp}_{#1}}

\DeclareSymbolFont{bbold}{U}{bbold}{m}{n}
\DeclareSymbolFontAlphabet{\mathbbold}{bbold}



\title{Perfect Powers in Lucas Sequences}
\author{Jesse Silliman and Isabel Vogt}


\date{\today}

\begin{document}

\begin{frame}
\thispagestyle{empty}         
\titlepage



\end{frame}
\addtocounter{framenumber}{-1} 

\begin{frame}{The Fibonacci Sequence}

\[0,1,1,2,3,5,8,13,21,34,55,89, 144,233, 377 \cdots \]


\end{frame}

\begin{frame}{The Fibonacci Sequence}

\[\emph{0},\emph{1},\emph{1},2,3,5,\emph{8},13,21,34,55,89,\emph{144},233, 377 \cdots \]

\pause

Folklore conjecture: these are the \emph{only} perfect powers!


\end{frame}





\begin{frame}{Perfect Powers in the Fibonacci Sequence}

\begin{theorem}[Siksek, Bugeaud, Mignotte, 2006]
0,1,8,144 are the only pp
\end{theorem}

\pause

\begin{proof}
Modularity of elliptic curves :-)
\end{proof}


\end{frame}

\begin{frame}{Our Aims}

Generalize the result of Siksek 2006 to other binary recurrence sequences.

\begin{enumerate}[1.]

\item Find more examples

\item Prove a general result for all Lucas sequences

\end{enumerate}

\end{frame}

\begin{frame}{Lucas Sequences}

Definition.

\end{frame}

\begin{frame}{Theorem 1}

\begin{theorem}
Unconditional specific sequences

\end{theorem}


\end{frame}

\begin{frame}{Theorem 2}

\begin{theorem}
General Theorem

\end{theorem}


\end{frame}

\begin{frame}{Theorem 3}

\begin{theorem}
Conditional specific sequences

\end{theorem}


\end{frame}

\begin{frame}{The modular method}

Frey curves \\

Galois Representations \\

Modularity of Elliptic curves / Modular forms \\

Level-lowering

\end{frame}


\begin{frame}{Proof of Theorem 1}


\end{frame}

\begin{frame}{Why it's hard}


\end{frame}


\begin{frame}{Sketch of Theorem 2}

\end{frame}


\begin{frame}{Frey-Mazur Conjecture}

State the conjecture.


\end{frame}


\begin{frame}{The Elliptic Curve Case}


\end{frame}


\begin{frame}{The Higher Dimensional Abelian Variety Case}


\end{frame}


\begin{frame}{Conditional Examples}

statement of thm


\end{frame}


\begin{frame}{Bounding $p$}


\end{frame}


\begin{frame}{Bounding $n$}


\end{frame}


\begin{frame}{The Sieve}


\end{frame}

\begin{frame}{Conclusion}

\end{frame}











\end{document}