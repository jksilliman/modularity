\documentclass[12pt]{article}
\usepackage{latexsym}
\usepackage{amssymb,amsmath}
\usepackage[pdftex]{graphicx}
\usepackage{listings}
\usepackage{courier}
\usepackage{color}
\usepackage[usenames,dvipsnames]{xcolor}
\usepackage{enumerate}
\usepackage{endnotes}
%\usepackage{extpfeil}
\usepackage{stackrel}
\usepackage{bbm}
\usepackage{tikz}
\usepackage[margin=2cm]{geometry}
\usepackage{hyperref}

\lstset{
	basicstyle=\small\ttfamily,
	keywordstyle=\color{blue},
	language=python,
	xleftmargin=16pt,
}

\newtheorem{thm}{Theorem}[section]
\newtheorem{ithm}{Theorem}
\newtheorem{lem}[thm]{Lemma}
\newtheorem{prop}[thm]{Proposition}
\newtheorem{cor}[thm]{Corollary}
\newenvironment{proof}[1][Proof.]{\begin{trivlist}
\item[\hskip \labelsep {\bfseries #1}]}{\end{trivlist}}

\newtheorem{defi}[thm]{Definition}
\newtheorem{example}[thm]{Example}
\newtheorem{exercise}[thm]{Exercise}
\newtheorem{rem}[thm]{Remark}

   
\def\B{{\mathbb B}}
\def\C{{\mathbb C}}
\def\D{{\mathbb D}}
\def\Fp{{\mathbb F}_p}
\def\Fell{{\mathbb F}_{\ell}}
\def\F{{\mathbb F}}
\def\H{{\mathbb H}}
\def\M{{\mathbb M}}
\def\N{{\mathbb N}}
\def\O{{\mathcal O}}
\def\0{{\mathbb 0}}
\def\P{{{\mathbb P}}}
\def\Q{{\mathbb Q}}
\def\R{{\mathbb R}}
\def\T{{\mathbb T}}
\def\Z{{\mathbb Z}}

\newcommand{\sol}{_{a^p,b^p,c^p}}
\newcommand{\bound}{\partial}
\newcommand{\la}[1]{\mathfrak{#1}}
\newcommand{\im}{\text{Im} \hspace{0.1em} }
\newcommand{\ann}{\text{Ann} \hspace{0.1em} }
\newcommand{\rank}{\text{rank} \hspace{0.1em} }
\newcommand{\coker}[1]{\text{coker}\hspace{0.1em}{#1}}
\newcommand{\sgn}{\text{sgn}}
\newcommand{\lcm}{\text{lcm}}
\newcommand{\re}{\text{Re}  \hspace{0.1em} }
\newcommand{\ext}[1]{\text{Ext}(#1)}
\newcommand{\Hom}[1]{\text{Hom}(#1)}
\newcommand{\End}[1]{\text{End(#1)}}
\newcommand{\bs}{\setminus}
\newcommand{\rpp}[1]{\mathbb{R}\text{P}^{#1}}
\newcommand{\cpp}[1]{\mathbb{C}\text{P}^{#1}}
\newcommand{\tr}{\text{tr}\hspace{0.1em} }
\newcommand{\inner}[1]{\langle {#1}\rangle}
\newcommand{\tensor}{\otimes}
\newcommand{\Cl}{\text{Cl}}
\renewcommand{\sp}[1]{\text{Sp}_{#1}}
\newcommand{\GL}{\text{GL}}
\newcommand{\pgl}[1]{\text{PGL}_{#1}}
\renewcommand{\sl}[1]{\text{SL}_{#1}}
\newcommand{\so}[1]{\text{SO}_{#1}}
\newcommand{\SO}{\text{SO}}
\newcommand{\pso}[1]{\text{PSO}_{#1}}
\renewcommand{\o}[1]{\text{O}_{#1}}
\renewcommand{\sp}[1]{\text{Sp}_{#1}}
\newcommand{\psp}[1]{\text{PSp}_{#1}}
\newcommand{\Span}{\rm Span}
\newcommand{\Frob}{\rm Frob}
\newcommand{\tor}{\rm tor}
\newcommand{\rad}{\rm rad}
\newcommand{\denom}{\rm denom}
\renewcommand{\bar}{\overline}
\newcommand{\notdiv}{\nshortmid}

\newcommand{\kron}[2]{\bigl(\frac{#1}{#2}\bigr)}
\newcommand{\leg}[2]{\Biggl(\frac{#1}{#2}\Biggr)}

\DeclareSymbolFont{bbold}{U}{bbold}{m}{n}
\DeclareSymbolFontAlphabet{\mathbbold}{bbold}


\begin{document}


\begin{center}
{\bf \large{Progress in the direction of perfect powers in binary recursion sequences}} \\
\smallskip
Isabel Vogt\\
Last Edited: \today \\
\end{center}

This document contains information on perfect powers in binary recursion sequences.

\section{Previous Results}

From Siksek 2006:

\begin{thm}\label{fibluc}
The only perfect powers in the Fibonacci sequence are $F_0 = 0, F_1 = 1, F_2 = 1, F_6 = 8, F_{12} = 144$.  The only perfect powers in the Lucas sequence are $F_1 = 1$ and $F_3 = 4$. 
\end{thm}

\section{Preliminaries}

We define a binary recursion relation
\[ u_{n+2} = b\cdot u_{n+1}+ c\cdot u_n. \]

We can derive a Binet-like formula as follows, the characteristic polynomial is
\[ g(z) = z^2 - bz - c\]
with roots
\[ \alpha, \beta = \frac{b \pm \sqrt{b^2+4c}}{2} \]

Then let the ``Fibonacci" sequence for this relation be define by the initial conditions
\begin{align*}
F_0 &= 0 \\
F_1 & = 1 
\end{align*}
and the ``Lucas" sequence define by the initial conditions
\begin{align*}
L_0 &= 2 \\
L_1 & = b. 
\end{align*}

Then we have the formulas:
\[F_n = \frac{\alpha^n - \beta^n}{\alpha - \beta} \qquad \qquad L_n = \alpha^n +\beta^n \]

The following general facts hold:

\begin{enumerate}

\item $F_{2k} = F_kL_k$

\item $(\alpha - \beta)^2F_n^2 = L_n^2 - 4(\alpha\beta)^n$

\item $(\alpha - \beta)^2 = b^2+4c \equiv 0,1 \pmod{4}$

\end{enumerate}

The second fact in the list allows us to generate Frey curves.  In particular, say that we posit a solution $F_n = y^p$, that is
\[ (b^2+4c)y^{2p}+4(-c)^n = L_n^2 .\]
This is a solution $(u,v,w)$ to the diophantine equation
\[ (b^2+4c)X^p +4(-c)^nY^p = Z^2 \]
with 
\[ u = y^2 \qquad v = 1 \qquad w = L_n .\]
Note that this is of the general form $Ax^p+By^p = Cz^2$.  Alternatively with a solution $L_n = y^p$ we get a solution to the diophantine equation
\[ X^p-4(-c)^nY^p = (b^2+4c)Z^2 \]
with
\[ u = y^2 \qquad v = 1 \qquad w = F_n .\]

To associate Frey curves, using the theory of Bennett and Skinner.  First, assume that $F_n$ and $L_n$ are odd.  For the Fibonacci case we have the Diophantine equation:
\[(b^2+4c)y^{2p} +2^2(-c)^n = L_n^2 \]
Now we have two possibilities:

\begin{itemize}

\item if $2 |c$ then for $n \geq 5$ we can associate the Frey curve
\[ E: Y^2 +XY = X^3 +\frac{H_n -1}{4} X^2 +2^{-4}(-c)^n \]
for $H_n = \pm L_n$ such that $H_n \equiv 1 \pmod{4}$.  This gives
\[ N_E = \prod_{\ell | b^2+4c} \ell \cdot \prod_{\ell | c} \ell \cdot \prod_{\ell | y} \ell. \]
The mod $p$ Galois representation is unramified at all primes dividing $y$ by a Tate curve argument, and irreducible for $p \geq 7$ by a theorem of Mazur.  So this descends to level 
\[N = \prod_{\ell | b^2+4c} \ell \cdot \prod_{\ell | c} \ell . \]
note in particular that this is squarefree (semistable).

\item If $2 \not | c$ then we can associate the Frey curve
\[ E: Y^2 = X^3 + H_n X^2 +(-c)^n X\]
with $H_n = \pm L_n$ so that $H_n \equiv -(-c)^n \pmod{4}$.  This has conductor
\[N_E = 2^{\alpha} \cdot  \prod_{\ell | b^2+4c} \ell \cdot \prod_{\ell | c} \ell \cdot \prod_{\ell | y} \ell\] 
with
\[ \alpha = \begin{cases} 2 & : (-c)^n \equiv -1 \pmod{4} \\ 3 & : (-c)^n \equiv 1 \pmod{4} \end{cases}. \]

Again the mod $p$ Galois representation is unramified at primes dividing $y$, so this descends to level
\[ N = 2^\alpha \cdot \prod_{\ell | b^2+4c} \ell \cdot \prod_{\ell | c} \ell . \]

\end{itemize}




\section{Some Results}

Here are some (mostly trivial) results so far:

\begin{prop}
The sequence define by
\[ u_{n+2} = 3u_{n+1}-u_n \]
has no perfect powers besides $0,1,8,144$.
\end{prop}

\begin{proof}
The characteristic equation for this recursion sequence is
\[g(z) = z^2-3z+1\]
with roots
\[ \alpha, \beta = \frac{3 \pm \sqrt{5}}{2} = \omega+1, \tau+1 \]
for $\omega$ and $\tau$ the roots of $z^2-z-1$.  Note that $\omega^2 = \omega+1$, $\tau^2 = \tau + 1$.  So the terms of the sequence beginning with $u_0 = 0$ and $u_1 = 1$ are
\[ \tilde{F}_n = \frac{(\omega+1)^n - (\tau+1)^n}{\omega-\tau} = \frac{\omega^{2n} - \tau^{2n}}{\sqrt{5}} = F_{2n} \]
So the terms in this sequence are every other Fibonacci term, so the perfect powers are the same.
\end{proof}

(Note to self: there might be some hope of generalizing this: we can easily see that
\[ \omega^n = F_n\omega +F_{n-1} \]
so taking the sequence $\tilde{F}_k$ with Fibonacci starting conditions and roots of the characteristic polynomial $\omega^n$ and $\bar{\omega^n}$ we get 
\[ \tilde{F}_k = \frac{\omega^{kn} - \bar{\omega^{kn}}}{\omega^n-\bar{\omega^n}} = \frac{F_{nk}}{F_n} \]
which might show some promise... but more to think about later).


Another fruitful place to look is generalized sequences where $b^2+4c = +1$ (this is possible infinitely often as there are infinitely many odd squares).

Some examples are:

\begin{center}
\begin{tabular}{ c | c}
b & c \\ \hline \hline
3 & -2 \\
5 & -6 \\
7 & -12 \\
9 & -20 \\
11 & -30 \\
13 & -42 \\
15 & -56 \\
17 & -72 \\
19 & -90 
\end{tabular}
\end{center}

We would like to be able to do away with all of these.  We can begin with the first few.

\begin{prop}
There are no nontrivial perfect powers for in the sequence $F_n$ defined by $F_0=0$, $F_1=1$ and 
\[ F_{n+2} = 3F_{n+1}-2F_n \]
\end{prop}
\begin{proof}
Clearly, we define a nontrivial perfect power not to equal $0,1$.  Using the fact that the characteristic polynomial for this recurrence is
\[ g(z) = z^2-3z+2 = (z-2)(z-1) \]
we have the relation
\[ F_n^2 = L_n^2 - 4 \cdot 2^n \]
where $L_n$ is the generalized Lucas sequence define by the same recurrence relation but starting conditions $L_0 = 2$, $L_1 = b$.  $F_n, L_n, 2$ are all relatively prime for $n \geq 2$ as $F_n$ and $L_n$ are odd, and the above relation rules out the possibility that they are divisible by any other common primes that do not divide $2$.  Thus this is a primitive solution that would	 give rise to a Frey curve:
\[ E: Y^2 + XY = X^3 + \left(\frac{H_n-1}{4} \right)X^2 + 2^{n-4}X \]
where $H_n = \pm L_n$ so that $H_n \equiv 1 \pmod{4}$, and $n \geq 5$.  Then
\[ \Delta_E = 2^{2n-8}y^{2p} \qquad \qquad N_E = 2 \prod_{\ell | y} \ell \]
and further $\rho_{E,p}$ is irreducible and unramified outside $2p$ (weakly at $p$) for $p \geq 7$ by a Tate curve argument as well as theorems of Mazur on rational torsion.  So this descends to a representation $\rho_f \simeq \rho_{E,p}$ of some $f$ a weight 2 newform of level $2$.  As this cannot exist, there are no perfect $p$th powers for $p \geq 7$.  To check the others is a another possible computation.
\end{proof}

\begin{prop}
There are no nontrivial perfect powers in the sequence $F_n$ define by $F_0 = 0$, $F_1 = 1$ and the recursion relation
\[ F_{n+2} = 5F_{n+1}-6F_n. \]
\end{prop}

\begin{proof}
Many of the techniques of the previous problem are explicitly applicable here.  First we derive the equation
\[ F_n^2 = L_n^2 - 2^{n+2}3^n \qquad \Rightarrow \qquad y^{2p} + 2^{n+2}3^n = L_n\]
for a posited solution $F_n = y^p$.  We then check that $F_n$ and $L_n$ are not divisible by $2$ and $3$.  We associate this to the Frey curve
\[ E: Y^2 +XY = X^3 + \left(\frac{H_n-1}{4} \right)X^2 + 2^{n-4}3^nX \]
We can also calculate
\[ \Delta_E = 2^{2n-8}3^ny^{2p} \qquad \qquad N_E = 2 \cdot 3 \cdot \prod_{\ell | y} \ell .\]
By the same arguments as above, for $n \geq 5$ and $p \geq 7$, there are no nontrivial solutions as the space of newforms of level $6$ has dimension $0$.  To confirm that there are no nontrivial powers of smaller $p$ is a possible computation.
\end{proof}

This technique works for the other examples as well.  Note that we can prove in general that no terms of the generalized Fibonacci or Lucas sequences for our restricted set of $b^2+4c=1$ sequences will be divisible by the factors of $c$.  By the above equation it is clear that $b$ and $c$ are relatively prime.  Out general recurrence relation is
\[ u_{n+2} = bu_{n+1} +cu_n \]
This clearly hold for $u_2$ as $u_1=1$ so $bu_{1}$ is not divisible by any primes dividing $c$, so neither is $u_2$.  Then clearly the full claim holds by induction.  So our general solution
\[ F_n^2 = L_n^2 - 4c^n\]
does in general represent a primitive solution.  The other arguments hold as well for $n \geq 5$ and $p \geq 7$.  So we in general can descend to a level
\[ N = \rad(c) \]
which may, or may not, be populated by newforms.  For the examples above this simplifies to the following information:

\begin{center}
\begin{tabular}{ c c | c c}
$b$ & $c$ & $N$ & space of newforms  \\ \hline \hline
3 & -2 & 2 & no newforms \\
5 & -6 & 6 & no newforms\\
7 & -12 & 6 & no newforms\\
9 & -20 & 10 & no newforms\\
11 & -30 & 30 & 1D, $E = Y^2 +XY + Y = X^3+X+2$,  **check** \\
13 & -42 & 42 & 1D, $E = Y^2 +XY +Y = X^3 + X^2 +-4X+5$,   **check** \\
15 & -56 & 14& 1D, $E = Y^2 +XY+Y = X^3 +4X-6$,   **check**  \\
17 & -72 & 6 & no newforms \\
19 & -90 & 30 &  **check**
\end{tabular}
\end{center}


The fact that we can rule out $2 \cdot 3$ and $2 \cdot 5$ (and for that matter $2 \cdot 11$) easily means that if 
\[ \rad(c) = 2 \cdot 3 \text{ or } 2 \cdot 5 \text{ or }2 \cdot 11 \]
then we have a contradiction.  There are only finitely many solutions to
\[ b^2 +4c = 1 : \rad(c) = 2 \cdot 3 \text{ or } 2 \cdot 5 \text{ or }2 \cdot 11 \]
namely:
\begin{center}
\begin{tabular}{c c c}
b & c & \rad(c) \\ \hline \hline
3 & -2 & 2  \\
5 & -6 & $2\cdot$3  \\
7 & -12 & $2\cdot$3  \\
17 & -72 & $2\cdot$3 \\
9 & -20 & $2\cdot$5
\end{tabular}
\end{center}
so the above is a complete list of easy cases.  This can be summarized in the following theorem:

\begin{ithm}[No Newforms]
Let $F_n$ be the binary recursion sequence defined by 
\[ F_{n+2} = bF_{n+1} +cF_n \]
with $b^2+4c = 1$ and initial conditions $F_0 = 0$ and $F_1=1$.  Then for the following values of $b$ and $c$:
\begin{center}
\begin{tabular}{c | c }
b & c \\  \hline \hline
3 & -2 \\
5 & -6  \\
7 & -12 \\
17 & -72  \\
9 & -20 \\ \hline \hline
\end{tabular}
\end{center}
$F_n$ has no nontrivial $p$th powers for $p \geq 7$ and $n \geq 5$.
\end{ithm}



\section{All recursion relations with $b,c \leq 5$ and relatively prime}

In this section we treat all binary recurrence relations
\[ u_{n+2} = b u_{n+1} + c u_n \]
with $u_0 = 0$ and $u_1=1$ for $b,c \leq 5$ and $\gcd(b,c)=1$.
These are

\begin{center}
\begin{tabular}{c c | c c}
$|b|$ & $c$ & no perfect powers for $20 \leq n < 10000$ & perfect power for $n=$ \\ \hline \hline
1 & -5 & True & 2,7 \\
1 & -4 & True & 2 \\
1 & -3 & True & 2,5,6 \\
1 & -2 & True & 2,3,5,13 \\
1 & -1 & False & everything \\
1 & 1 & True & 2,6,12 \\
1 & 2 & True & 2 \\
1 & 3 & True & 2,3 \\
1 & 4 & True & 2,4,8 \\
1 & 5 & True & 2 \\
2 & -5 & True & 3 \\
2 & -3 & True & 3 \\
2 & -1 & False & many \\
2 & 1 & True & 7 \\
2 & 3 & True & 3 \\
2 & 5 & True & 3 \\ 
3 & -5 & True & 3 \\
3 & -4 & True & none \\
3 & -2 & True & none \\
3 & -1 & True & 3,6 \\
3 & 1 & True & none \\
3 & 2 & True & none \\
3 & 4 & True & none \\
3 & 5 & True & none \\ \hline \hline
\end{tabular}

\begin{tabular}{c c | c c}
$|b|$ & $c$ & no perfect powers for $20 \leq n < 10000$ & perfect power for $n=$ \\ \hline \hline
4 & -5 & True & 2 \\
4 & -3 & True & 2,5 \\
4 & -1 & True & 2 \\
4 & 1 & True & 2 \\
4 & 3 & True & 2 \\
4 & 5 & True & 2 \\
5 & -4 & True & none\\
5 & -3 & True & none\\
5 & -2 & True & none\\
5 & -1 & True & none\\
5 & 1 & True & none\\
5 & 2 & True & 3\\
5 & 3 & True & none\\
5 & 4 & True & none \\ \hline \hline
\end{tabular}
\end{center}

\subsection{The cases we can deal with thus far}

\begin{enumerate}

\item[\textbf{1.}] $b = 3$, $c = -2$

We have already proven that there are no prime powers for $p \geq 7$ by a Frey curve argument.  So we need only show that there are no solutions for $p = 2,3,5$.

For $p=2$ we would have a solution to
\[ x^2-y^{4} = (x+y^2)(x-y^2)= 2^{n+2} \]
where we know that $x$ and $y$ are odd for $x,y >1$.  That is
\begin{align*}
x+y^2 & = 2^{n_1} \\
x-y^2 & = 2^{n_2} \\ 
\end{align*}
where our above requirements on $x,y$ means $n_1 > n_2  \geq 1$.  So solving:
\[ x = 2^{n_1-1} +2^{n_2-1} \]
this will be even for $n_2 >1$, giving a contradiction.  So assume that $n_2 = 1$.  Then $x = 1+2^{n_1-1}$ and so 
\[ y^2 = 2^{n_1-1} - 1 \]
but as $n_1+n_2 = n >3$, $n_1 > 2$ so $4|2^{n_1-1}$ implying that $-1$ is a square mod $4$, which is a contradiction.   This argument only works for even $p$, ie $p = 2$.  \textbf{Need to do $p=3,5$}.  This follows from Catalan's conjecture.  But clearly as above we get this corollary:

\begin{cor}
No Mersenne number $2^n-1$ is a perfect power for $n>1$.
\end{cor}

which is obviously already known by Catalan's theorem.
 
\item[\textbf{2.}] $b = 3$, $c = 1$
In this case we have the Diophantine equation
\[13y^{2p}  \pm 4 = L_n^2\]
where it is $+4$ if $n$ is even and $-4$ if $n$ is odd.  
In addition we have the congruences
\begin{center}
\begin{tabular}{c| c c}
$n \pmod{6}$ & $F_n \pmod{4}$ & $L_n \pmod{4}$ \\ \hline \hline
0 & 0 & 2 \\
1 & 1 &  3 \\
2 & 3 & 3 \\
3 & 2 & 0 \\
4 & 1 & 3 \\
5 & 1 & 1  \\ \hline \hline
\end{tabular}
\end{center}

First we deal with the small exponents $p = 2,3,5$.  That is first we want to rule out solutions to
\[ 13y^{4} -x^{2} = \pm 4 \]

**think about maybe factoring this over $\Q(\sqrt{13})$ which has class number 1.  Maybe something along the line of 
\[ (x + \sqrt{13} y^2)(x - \sqrt{13}y^2) = (2)^2 \]
This means that $(x+\sqrt{13}y^2)$ and $(x-\sqrt{13}y^2)$ both need to be $(2 \cdot u^r)$ for $u$ the fundamental unit 
\[ u = \frac{3 + \sqrt{13}}{2} \]
**

We can prove that there are no nontrivial squares as follows:

The Diophantine equation
\[ Ln^2 = 13y^4 \pm 4\]
gives an integral point $(13y^2, 13yL_n)$ on the elliptic curve
\[ Y^2 = X^3 \pm 4 \cdot 13 X .\]
With sage we can check that for $+52$ the only integral point is $(0:0:1)$ and with $-52$, the only integral points are $(-4:12:1), (0:0:1), (13:39:0)$.  This means that the only solution is the trivial for $y = 1, L_n = 3$.   So we conclude that there are no nontrivial squares.



-----


We want to rule out seperately the cases of $F_n$ and $L_n$ both even (ie $n \equiv 0,3 \pmod{4}$) and both odd.

First treating the even case, we can immediately rule out the case of $n \equiv 3 \pmod{4}$ as no perfect power is $2 \pmod{4}$  This implies that $n$ is even.  Next, we define the variables
\[ z = y/2 \qquad \qquad \tilde{L} = \pm L_n/2 \text{ such that } \tilde{L} \equiv 1 \pmod{4}. \]
Then we have the relation
\[ 13 \cdot 2^{2p-2} \cdot z^{2p}+1 = \tilde{L}^2 \] 
which we associate to a Frey curve
\[E: Y^2 +XY = X^3 +\frac{\tilde{L}-1}{4} X^2 +13 \cdot 2^{2p-8} \cdot z^{2p} X .\]
This has conductor
\[ N_E = 2 \cdot 13 \cdot \prod_{\ell | z} \ell \]
and the mod $p$ representation for $p \geq 7$ descends to level $26$ as it is unramified at primes dividing $z$.  Unfortunately, $S_2(\Gamma_0(26))$ is dimension $2$, with 2 rational newforms, corresponding to the isogeny classes of elliptic curves $E^{26a}$ and $E^{26b}$ in the Cremona database.  These have representatives
\begin{align*}
E^{26a} & : y^2 + xy + y = x^3 - 5x - 8 \\
E^{26b} & : y^2 + xy + y = x^3 - x^2 - 3x + 3 \\
\end{align*}
and we can not immediately rule these out. **Probably rule one of these out using congruences a la Siksek page 986**


Now assume both $L_n$ and $y$ are odd and $n$ is odd.  We associate the Frey curve
\[ E: Y^2 = X^3 + H_nX^2 - X \]
where $H_n = \pm L_n$ such that $H_n \equiv 1 \pmod{4}$.  
This has conductor
\[ N_E = 2^2 \cdot 13 \cdot \prod_{\ell | y} \ell. \]
The mod $p$ representation $\rho_{E,p}$ is unramified at $\ell | y$ and so we conclude that for $p \geq 7$ this is the shadow of a representation $\rho_f$ for $f$ a newform of level 52.  This space is $1$-dimensional, with a single rational newform corresponding to the isogeny class of the elliptic curve
\[ E^{52a}: y^2 = x^3 + x - 10\]
in the Cremona database.

If $n$ were even then we would need
\[ E': Y^2 = X^3 + H_n'X^2 + X \]
where $H_n' = \pm L_n$ such that $H_n' \equiv -1 \pmod{4}$.  This descends to a level $N = 2^3 \cdot 13$.  The space of cusp forms of level $104$ is $2$-dimensional, consisting of one rational and one irrational newform.  These respectively correspond to an elliptic curve ($E^{104}: y^2 = x^3 + x^2 - 16*x - 32$) and abelian variety respectively.



\item[\textbf{3.}] $b = 2$, $c = 1$

In this case we are working with the Diophantine equation
\[ 8y^{2p} \pm 4 = L_n^2 \]
where the sequence $L_n$ has initial conditions $L_0 = 2$ and $L_2 = 2$.  The congruences look like
\begin{center}
\begin{tabular}{c| c c}
$n \pmod{4}$ & $F_n \pmod{4}$ & $L_n \pmod{4}$ \\ \hline \hline
0 & 0 & 2 \\
1 & 1 &  2 \\
2 & 2 & 2\\
3 & 1 & 2 \\ \hline \hline
\end{tabular}
\end{center}

No perfect power is $2 \pmod{4}$, so $L_n$ is never a perfect power.  In addition $F_n$ is not a perfect power for $n \equiv 2 \pmod{4}$.  Assume that $n > 1$.

Assume that $n \equiv 0 \pmod{4}$.  Then let $z = y/2$ and $\tilde{L} = \pm L_n/2$ so that $\tilde{L} \equiv 1 \pmod{4}$; we get the Diophantine equation
\[ 2^{2p+1}z^{2p} + 1 = \tilde{L}^2 \]
which we associate to the elliptic curve
\[ E: Y^2 +XY = X^3 + \frac{\tilde{L}-1}{4} X^2 + 2^{2p-5}z^{2p}X \]
which has conductor 
\[N_E = 2 \cdot \prod_{\ell | z} \ell \]
unramified at all primes dividing $z$, meaning we arrive at a contradiction.  

Similarly for $ n \equiv \pm 1 \pmod{4}$, we again use $\tilde{L} = L_n/2$.  We have the Diophantine equation
\[ 2y^{2p} -1 = \tilde{L}^2 \] 
which we associate to the Frey curve
\[ E: Y^2 = X^3 + 2\tilde{L}X^2+2y^{2p}X \]
which has conductor
\[N_E = 2^7 \cdot \prod_{\ell |y} \ell. \]
It descends to level $2^7 = 128$.  The space of newforms of level $128$ is dimension $4$, spanned by $4$ rational newforms, all of which have CM.  **think about ruling out CM elliptic curves**


\item[\textbf{4.}] $b = 4$, $c = 1$

The Diophantine equation here is 
\[ 20y^{2p} \pm4 = L_n^2\]
with congruences
\begin{center}
\begin{tabular}{c| c c}
$n \pmod{2}$ & $F_n \pmod{4}$ & $L_n \pmod{4}$ \\ \hline \hline
0 & 0 & 2 \\
1 & 1 &  0 \\ \hline \hline
\end{tabular}
\end{center}

We we can divide our attention into the cases of even and odd $n$.  In the even $n$ case first, note that $L_n$ is not a perfect power.  In this case, both $y$ and $L_n$ are even so we employ the same trick as above and change variables to $z = y/2$ and $\tilde{L} =  L_n/2$:
\[ 2^{2p} \cdot 5 z^{2p} + 1 = \tilde{L}^2. \]
We associate this to the elliptic curve 
\[ E: Y^2 + XY = X^3 + \frac{\tilde{L} -1}{4} X^2 + 2^{2p-6} \cdot 5 z^{2p} X \]
with conductor
\[N_E = 2 \cdot 5 \cdot \prod_{\ell | z} \ell \]
the mod $p$ Galois representation of which descends to level $10$, where there are no newforms, so we reach a contradiction.

Now for the case that $n$ is odd, we get the Diophantine equation
\[ 5 \cdot y^{2p} -1 = \tilde{L}^2 \]
which we associate to the Frey curve
\[ E: Y^2 = X^3 + 2\tilde{L} X^2 - X \]
with conductor
\[N_E = 2^5\cdot 5 \prod_{\ell | y} \ell. \]
As above the Galois representation descends to level $2^5 \cdot 5 = 160$, which is 4 dimensional, spanned by $2$ rational newforms and one irrational. **work on ruling out the two elliptic curves and the abelian variety**


\item[\textbf{4.}] $b = 5$, $c = 1$

In this case we have the relation 
\[ 29y^{2p} \pm 4 = L_n^2 \]
with the congruence conditions:
\begin{center}
\begin{tabular}{c| c c}
$n \pmod{6}$ & $F_n \pmod{4}$ & $L_n \pmod{4}$ \\ \hline \hline
0 & 0 & 2 \\
1 & 1 &  1 \\
2 & 1 & 3 \\
3 & 2 & 0 \\
4 & 3 & 3 \\
5 & 1 & 3  \\ \hline \hline
\end{tabular}
\end{center}

We must again separate by what residue classes of $n$ we want to consider.  






















\end{enumerate}


\section{In general for $c = 1$}

We have several possibilities based upon congruences:

\begin{enumerate}


\item We can immediately rule out the possibility of $F_n \equiv 2 \pmod{4}$ as being a perfect power

\item It is not possible that both $F_n$ and $L_n$ are $F_n, L_n \equiv 0 \pmod{4}$ (as can be seen by reducing mod $16$ on both sides of the standard Diophantine equation)

\item $F_n \equiv 0,2 \pmod{4}$ and $L_n \equiv 2 \pmod{4}$

We simplify our Diophantine equation using $z = y/2$ and $\tilde{L} = L/2$:
\[ (b^2+4c) 2^{2p-2} z^{2p} \pm 1 = \tilde{L}^2 \]
and then associate it to the Frey curve
\[E: Y^2 +XY = X^3 + \frac{\tilde{L} -1}{4} X^2  + (b^2+4c) 2^{2p-8} z^{2p}X \]
with conductor
\[N_E = 2 \cdot \prod_{\substack{\ell | b^2+4c \\ \ell \notdiv 2}} \ell \]
ie it is square free.

\item $F_n$ and $L_n$ are odd and $b^2+4c$ is odd

Then we have the usual Diophantine equation 
\[ (b^2+4c)y^{2p} \pm 4 = H_n^2 \]
where it is $+4$ if $n$ is even and $-4$ if $n$ is odd and $H_n = \pm L_n$ such that $H_n \equiv -1\pmod{4}$ if $n$ is even and $-1 \pmod{4}$ if $n$ is odd.
We associate this to the elliptic curve
\[ E: Y^3 = X^3 + H_n X^2 \pm X \]
where the $\pm$ corresponds to that in the original Diophantine equation.  This has conductor
\[N_E = 2^{\alpha} \cdot  \prod_{\substack{\ell | b^2+4c \\ \ell \notdiv 2}} \ell \cdot \prod_{\ell | y} \ell \]
with 
\[ \alpha = \begin{cases} 2 &: n \text{ odd} \\ 3 &:  n \text{ even} \end{cases} \]

\item It is not possible for $F_n$ and $L_n$ to be odd and $b^2+4c$ to be even (reduce the Diophantine equation mod $2$)

\item $F_n$ is odd and $L_n$ and $b^2+4c$ are even

Let $k$ be the order of $2$ dividing $b^2+4c$ such that $b^2 + 4c = 2^k \cdot A$.  Note that $k \geq 2$ as $b^2 + 4c \equiv 0,1 \pmod{4}$.  Then we again split into several cases:

\begin{enumerate}

\item $k = 2$:  We write our Diophantine equation as
\[ A y^{2p} \pm 1 = \tilde{L}^2 \]
where as above $\tilde{L} = \pm L_n/2$.  We can also say that $A \equiv \mp 1 \pmod{4}$ and $y^{2p} \equiv 1 \pmod{4}$ by congruences reasons.  We then associate the following Frey curve

\[ E : \begin{cases} Y^2 = X^3 + 2 \tilde{L}X^2 + A X & : A \equiv -1 \pmod{4} \qquad (n \text{ even}) \\  Y^2 = X^3 + 2 \tilde{L}X^2 - X & : A \equiv 1 \pmod{4} \qquad (n \text{ odd}) \end{cases} \]

These have conductor

\[ N_E =  2^5 \cdot \prod_{\ell |  A} \ell \cdot \prod_{\ell | y} \ell \]


\item $k =3$: we write our Diophantine equation as
\[ 2A y^{2p} \pm 1 = \tilde{L}^2 \]
which we associate the the Frey curve
\[ E : Y^2 = X^3 + 2\tilde{L}X^2 + 2Ay^{2p} X \]
which has conductor
\[ N_E = 2^7 \cdot \prod_{\ell | A} \ell \cdot \prod_{\ell | y} \ell. \]

\item $k = 4$: we write our Diophantine equation as
\[ 2^2A y^{2p} + 1  = \tilde{L}^2. \]
Note that in this case it is clear that $n$ is even as $\tilde{L}^2 \equiv 1 \pmod{4}$.  We choose $\tilde{L}$ such that it is congruent to $-A \pmod{4}$.  We can then associate this to the Frey curve given by
\[ E: Y^2 = X^3 + \tilde{L}X^2 + Ay^{2p}X \]
which has conductor
\[N_E = 2^\alpha \cdot \prod_{\ell | A} \ell \cdot \prod_{\ell | y} \ell \]
with 
\[ \alpha = \begin{cases} 2 & : F_n \equiv -A \pmod{4} \\ 3 & : F_n \equiv  A \pmod{4} \end{cases} .\]


\item $k=5, 6, 7$: we write our Diophantine equation as
\[ 2^{k-2}A y^{2p} \pm 1 = \tilde{L}^2 \]
note that in this case $\tilde{L} \equiv 1 \pmod{4}$.  We associate this to the Elliptic curve
\[ E: Y^2 = X^3 + \tilde{L}X^2 + 2^{k-4}A y^{2p} X \]
with conductor
\[ N_E = 2^\alpha \cdot \prod_{\ell | A} \ell \cdot \prod_{\ell | y} \ell \]
where
\[ \alpha = \begin{cases} 5 & : k = 5 \\ 3 & : k = 6,7 \end{cases} .\]

\item $k \geq 8$ : we write
\[ 2^{k-2} A y^{2p} \pm 1  = \tilde{L}^2 \]
where we take $\tilde{L} \equiv 1 \pmod{4}$.  We associate to this the elliptic curve
\[ E: Y^2 + XY = X^3 + \frac{\tilde{L} -1}{4}X^2 + 2^{k-8}Ay^{2p}X \]
with conductor
\[ N_E = 2^\alpha \cdot \prod_{\ell | A} \ell \cdot \prod_{\ell | y} \ell \]
where 
\[ \alpha = \begin{cases} 0 & : k = 8 \\ 1 & : k > 8 \end{cases} .\]



\end{enumerate}

All of this can be easily made into a sage algorithm see appendix \ref{findlevels}.



\end{enumerate}



Isabel is typing here!







\appendix


\section{Find Levels}\label{findlevels}

\begin{lstlisting}

def find_period_fib(b,c):
    R = Integers(4)
    f0 = 0; f1 = 1
    A1 = 0; A2 = 1
    for i in xrange(20):
        x = b*A2 + c*A1
        A1 = A2
        A2 = x
        if (i>= 1) and R(A1) == R(f0) and R(A2) == R(f1):
            return i+1
            
def find_period_luc(b,c):
    R = Integers(4)
    l0 = 2; l1 = b
    A1 = 2; A2 = b
    for i in xrange(20):
        x = b*A2 + c*A1
        A1 = A2
        A2 = x
        if (i>= 0) and R(A1) == R(l0) and R(A2) == R(l1):
            return i+1
            
def  get_congruences(b,c):
    R = Integers(4)
    f0 = 0; f1 = 1
    l0 = 2; l1 = b
    fib_per = find_period_fib(b,c)
    luc_per = find_period_luc(b,c)
    m = lcm(fib_per, luc_per)
    F = [R(0), R(1)]
    L = [R(2), R(b)]
    for i in xrange(m-2):
        x = b*f1 + c*f0
        f0 = f1
        f1 = x
        y = b*l1 + c*l0
        l0 = l1
        l1 = y
        F.append(R(x))
        L.append(R(y))
    return F,L
    
def find_levels(b,c):
    R = Integers(4)
    F,L = get_congruences(b,c)
    m = len(F)
    D = b^2+4*c
    for n in xrange(m):
        f = F[n]
        l = L[n]
        if R(f) == R(2):
            print "For term", n, "mod", m, "F is",f, \
            "mod 4 and is not a perfect power"
            continue
        elif GF(2)(f) == GF(2)(0) and R(l) == R(2):
            j = (radical(D)).valuation(2)
            radD = radical(D)/2^j
            N = 2*radD
        elif GF(2)(f) == GF(2)(1) and GF(2)(l) == GF(2)(1) \ 
        and GF(2)(D) == GF(2)(1):
            if GF(2)(n) == GF(2)(1):
                N = 2^2*radical(D)
            if GF(2)(n) == GF(2)(0):
                N = 2^3*radical(D)
        elif GF(2)(f) == GF(2)(1) and GF(2)(l) == GF(2)(0) \ 
        and GF(2)(D) == GF(2)(0):
            k = D.valuation(2)
            A = D/2^k
            if k == 2:
                N = 2^5*radical(A)
            if k == 3:
                N = 2^7*radical(A)
            if k == 4:
                if R(-A) == R(f):
                    N = 2^2*radical(A)
                if R(A) == R(f):
                    N = 2^3*radical(A)
            if k == 5:
                N = 2^5*radical(A)
            if k == 6 or k == 7:
                N = 2^3*radical(A)
            if k == 8:
                N = radical(A)
            if k >8:
                N = 2*radical(A)
        print "For term", n, "mod", m, "F is", f, "mod 4 and L is", \
        l, "mod 4 and it descends to level", N


\end{lstlisting}

\section{Analyze Levels}

\end{document}