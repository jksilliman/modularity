\documentclass[12pt]{article}
\usepackage{latexsym}
\usepackage{amssymb,amsmath}
\usepackage[pdftex]{graphicx}
\usepackage{listings}
\usepackage{courier}
\usepackage{color}
\usepackage[usenames,dvipsnames]{xcolor}
\usepackage{enumerate}
\usepackage{endnotes}
%\usepackage{extpfeil}
\usepackage{stackrel}
\usepackage{bbm}
\usepackage{tikz}
\usepackage[margin=2cm]{geometry}
\usepackage{hyperref}

\lstset{
	basicstyle=\small\ttfamily,
	keywordstyle=\color{blue},
	language=python,
	xleftmargin=16pt,
}

\newtheorem{thm}{Theorem}[section]
\newtheorem{ithm}{Theorem}
\newtheorem{lem}[thm]{Lemma}
\newtheorem{prop}[thm]{Proposition}
\newtheorem{cor}[thm]{Corollary}
\newenvironment{proof}[1][Proof.]{\begin{trivlist}
\item[\hskip \labelsep {\bfseries #1}]}{\end{trivlist}}

\newtheorem{defi}[thm]{Definition}
\newtheorem{example}[thm]{Example}
\newtheorem{exercise}[thm]{Exercise}
\newtheorem{rem}[thm]{Remark}

   
\def\B{{\mathbb B}}
\def\C{{\mathbb C}}
\def\D{{\mathbb D}}
\def\Fp{{\mathbb F}_p}
\def\Fell{{\mathbb F}_{\ell}}
\def\F{{\mathbb F}}
\def\H{{\mathbb H}}
\def\M{{\mathbb M}}
\def\N{{\mathbb N}}
\def\O{{\mathcal O}}
\def\0{{\mathbb 0}}
\def\P{{{\mathbb P}}}
\def\Q{{\mathbb Q}}
\def\R{{\mathbb R}}
\def\T{{\mathbb T}}
\def\Z{{\mathbb Z}}

\newcommand{\sol}{_{a^p,b^p,c^p}}
\newcommand{\bound}{\partial}
\newcommand{\la}[1]{\mathfrak{#1}}
\newcommand{\im}{\text{Im} \hspace{0.1em} }
\newcommand{\ann}{\text{Ann} \hspace{0.1em} }
\newcommand{\rank}{\text{rank} \hspace{0.1em} }
\newcommand{\coker}[1]{\text{coker}\hspace{0.1em}{#1}}
\newcommand{\sgn}{\text{sgn}}
\newcommand{\lcm}{\text{lcm}}
\newcommand{\re}{\text{Re}  \hspace{0.1em} }
\newcommand{\ext}[1]{\text{Ext}(#1)}
\newcommand{\Hom}[1]{\text{Hom}(#1)}
\newcommand{\End}[1]{\text{End(#1)}}
\newcommand{\bs}{\setminus}
\newcommand{\rpp}[1]{\mathbb{R}\text{P}^{#1}}
\newcommand{\cpp}[1]{\mathbb{C}\text{P}^{#1}}
\newcommand{\tr}{\text{tr}\hspace{0.1em} }
\newcommand{\inner}[1]{\langle {#1}\rangle}
\newcommand{\tensor}{\otimes}
\newcommand{\Cl}{\text{Cl}}
\renewcommand{\sp}[1]{\text{Sp}_{#1}}
\newcommand{\GL}{\text{GL}}
\newcommand{\pgl}[1]{\text{PGL}_{#1}}
\renewcommand{\sl}[1]{\text{SL}_{#1}}
\newcommand{\so}[1]{\text{SO}_{#1}}
\newcommand{\SO}{\text{SO}}
\newcommand{\pso}[1]{\text{PSO}_{#1}}
\renewcommand{\o}[1]{\text{O}_{#1}}
\renewcommand{\sp}[1]{\text{Sp}_{#1}}
\newcommand{\psp}[1]{\text{PSp}_{#1}}
\newcommand{\Span}{\rm Span}
\newcommand{\Frob}{\rm Frob}
\newcommand{\tor}{\rm tor}
\newcommand{\rad}{\rm rad}
\newcommand{\denom}{\rm denom}
\renewcommand{\bar}{\overline}
\newcommand{\notdiv}{\nshortmid}
\newcommand{\pfrac}[2]{\left( \frac{#1}{#2} \right)}

\newcommand{\kron}[2]{\bigl(\frac{#1}{#2}\bigr)}
\newcommand{\leg}[2]{\Biggl(\frac{#1}{#2}\Biggr)}

\DeclareSymbolFont{bbold}{U}{bbold}{m}{n}
\DeclareSymbolFontAlphabet{\mathbbold}{bbold}

\begin{document}


\begin{center}
{\bf \large{Progress in the direction of perfect powers in binary recursion sequences}} \\
\smallskip
Isabel Vogt, Jesse Silliman\\
Last Edited: \today \\
\end{center}

This document contains information on perfect powers in binary recursion sequences.

\section{Previous Results}


\begin{thm}[Siksek 2006]\label{fibluc}
The only perfect powers in the Fibonacci sequence are $F_0 = 0, F_1 = 1, F_2 = 1, F_6 = 8, F_{12} = 144$.  The only perfect powers in the Lucas sequence are $F_1 = 1$ and $F_3 = 4$. 
\end{thm}


\begin{thm}[Petho 1992]\label{pellseq}
There are no nontrivial perfect powers in the Pell sequence defined by $P_0 = 0$, $P_1 = 1$ and $P_{n+2} = 2P_{n+1}+P_n $.
\end{thm}


\section{Plan of Attack/Jesse's Musings}
We seek to enumerate the perfect powers in binary linear recurrence relations. To begin with, we look at Lucas sequences, which are of the form $F_m = b F_{m-1} + c F_{m-2}, F_0 = 0, F_1 = 1$. This is a generalization of the Fibonacci sequence. Given a solution $F_n=y^p$:
\begin{enumerate}
\item
  Consider the Diophantine equation $Dy^{2p} = L_n^2 - 4(-c)^n$.
\item
  Rule out p=2,3,5, using Thue equations, by solving for integral points on elliptic/higher genus curves.
\item
  Construct the Frey curve(s). There may be many cases which need to be dealt with, if $DF_n, L_n, 2c$ are not pairwise coprime.
\item Deal with the Frey curve, using modular methods. That is, we note that our Frey curve arises mod p from a modular form in a particular small level. First, the trivial cases:
    For $\ell$ a prime where $\rho_p$ is unramified, p must divide 
        \[ N_{K/Q}((l+1)^2 - c_l^2)\prod_{-2\sqrt{l} \leq 2r \leq 2\sqrt{l}} N_{K/Q}(2r - c_l), \]
        where $c_l$ are the coefficients of the associated modular form.
    If nonzero, this congruence condition bounds $p$. In particular, if $p \leq 5$, we are done. Note that this always succeeds for abelian varieties, since $c_l \notin \Z$ infinitely often. Unfortunately, it can fail to give information on elliptic curves.
    \begin{rem}
     If $p \mid a_{\ell}(E)$ for almost all $\ell$, then by Cebotarev Density (and some linear algebra, found in Serre's Abelian l-adic Repr.), we would have a p-torsion point on our Frey curve, contradicting Mazur's torsion theorem if $p \geq 11$. If our Frey curves had full 2-torsion, this would be more useful, for we might obtain 7-torsion on the curve below, which then would "lift" to our curve above, by the irreducibility of the representation (see Serre). Unfortunately, $p$ may divide the product of norms above for all such $\ell$ without this occuring. For an example, see the "Multi-Frey" paper by Siksek. 
     **QUESTION What "conditions" give rise to Frey curves with full two torsion? Why is it natural to consider curves with only some two torsion here, but all the families in "Multi-Frey" have full two-torsion?**
	\end{rem}    

	\begin{rem}
	If we descend to a curve (or abelian variety) with complex multiplication, we can likely find an immediate contradiction. However, this is of little use to us, as we are often in square-free levels, i.e. semistable elliptic curves, which cannot be CM, since CM implies potentially good. To tell if your abelian varieties have complex multiplication, the only method we know of is to construct enough distinct Hecke characters of the specified conductor. This appears to work only if all abelian varieties at the level had complex multiplication. We only need to study Galois conjugacy classes of abelian varieties/modular forms? **QUESTION Can we detect CM abelian varieties?**
	\end{rem}

\item If such tricks do not work, we imitate the paper of Siksek. 
**TODO: Verify the applications of linear forms in two/three logs**
\begin{itemize}
	\item A sieve ought to, in all cases, prove $\log y$ astronomical, $p > 1000$.
	\item Bound p above by about $10^8$, using the theory of linear forms in three logarithms, applied to 
	\[\Lambda = n \log \alpha - \log \sqrt{D} - p \log{y}. \]
	[BMS] did this by first applying Matveev's bound, in order to satisfy the quite complicated conditions of their improvement.
	 **CHECK this may require $n, p$ prime, y large**
	\item For p less than this bound, use the method of Kraus. For fixed p, this gives congruence conditions on $a_l$ for "good" $\ell$, and we do a computer search for such $l$, hoping to find that the conditions are not satisfied by any curves in our level. We will not be so lucky: there may be a Frey reduction which is equal to a curve in our level.
	\begin{rem}	
	 In order to define "good", we need to find a condition on $\ell$ which implies that $\ell \nmid N(E)$ which only depends on $p$, not (n, y). This is easier said than done: [BMS] required $n$ prime, and used the condition $1 \le F_p \le 2^p$, which are respectively we don't know how to achieve, and only work for the Fibonacci sequence. **TODO: Prove y large, in order to make the latter work.**
\end{rem}
\begin{rem}
	We believe that when [BMS] says that the trivial solution complicates their solution to the Fibonacci sequence, it is here where it occurs: for in their work, there exists a curve in their level which always agrees with a possible mod l reduction of the Frey curve. 
**TODO confirm that this corresponds to the trivial solution. Understand why this does not seem to complicate every Lucas sequence we consider, even though they all have same trivial solution: see the section on Kraus' method for egs**
\end{rem}

\item If Kraus' method does not rule out descent to a specific elliptic curve entirely, for $p < 10^8$, it might give us "local information" about the solution. In all of the examples we have seen, this information is $n = \pm 1 \mod p$, ruling out all but the equiv. classes of the two trivial solutions.
\item At this point, the condition $n = \pm 1 \mod p$ lets us rewrite $\Lambda$ as a linear form in two logs 
\[ \Lambda = p \log(\frac{\alpha^s}{y}) -  \log(\sqrt{D}\alpha^{\pm 1}). \]
Applying the powerful results of [LMN], we should get $p < 1000$, contradicting our sieve.
\end{itemize}
\end{enumerate}

\section{Status report of various families}
\subsection{$c=1$}

If $b^2+4$ is a prime less than $100000$, say, we can use an elementary argument of Robbins to reduce to prime index. It seems difficult to prove this in general with the same methods, for we can see no way of solving
\[\frac{F(d^2)}{d^2} = y^p, \]
or even simpler equations such as
\[\frac{F(9)}{F(3)} = y^p,\]
when the best tools we have seen can only potentially solve such an equation for \textbf{fixed} $p$ using Chabauty's method. We might be able to solve the latter for an infinite family if we could find lower bounds of the divisors of class numbers of $\mathbb{Q}(-d),\mathbb{Q}(-2d)$.

Then, for $F_n$ odd, Kraus' method then gives, for $p$ as large as we can compute, either that there are no solutions, or that $n \equiv \pm 1 \mod p$. If the level has \textbf{no abelian varieties}, then at this point, we \textbf{expect} to be able to use linear forms in logs and sieves as in [BMS] to complete any specific recurrence relation in about $150*N$ hours (N=\# of remaining elliptic curves) , but this has not been verified.

For $F_n$ even, we would have to use a different Frey curve, which descends to level $2\cdot d$ as well.

\textbf{Questions:}
\begin{enumerate}
\item Can we detect CM abelian varieties?
\item What can be done for non-CM abelian varieties, besides the most naive congruences?
\item Does Kraus work for $F_n$ even?
\item Verify linear forms in logs, etc, for at least 1 sequence.
\end{enumerate}


\subsection{$c=-1$}

The sequence $b=2$ is degenerate.
For $b=4,6,8$, we find that either the level is small, or that we can rewrite our Diophantine equation as $x^2 - C y^2p = 1$, which has no solutions which agree with ours, by Bennett (2004).

\textbf{Questions:}
\begin{enumerate}
\item Can we do a reduction to prime index, Kraus, etc?
\end{enumerate}


\subsection{$b^2 + 4c = 1$}
\begin{center}
\begin{tabular}{c c c}
b & c & \rad(c) \\ \hline \hline
3 & -2 & 2  \\
5 & -6 & $2\cdot$3  \\
7 & -12 & $2\cdot$3  \\
17 & -72 & $2\cdot$3 \\
9 & -20 & $2\cdot$5
\end{tabular}
\end{center}
have no newforms in their levels, hence $F_n$ has no nontrivial pth powers for $p \geq 7, n \geq 5$.

$p=2:$ No solutions for $(3,-2)$ by the parametrization of pythagorean triples.

$p=3:$ No solutions for any $(b,c)$, using a reduction to odd index, since $x^r+y^r=z^3$ has no solutions.

$p=5:$ The Fermat equation $(r,r,5)$ is not yet understood, unfortunately...

\textbf{Associated Diophantine Equations}:

$x^n + (x-1)^n = y^p$, for all n odd, p prime. Although this is useful in low levels, (n,n,p) has not been generally solved for $p \geq 5$. There are some methods of Darmon, Frietas, using Q-curves, Hilbert modforms, multifrey, which conjecturally could give more information.

$y^{2p} + 4(-c)^n = x^2$, which is specialization of Fermat $(p, (4(-c)^n, p), 2)$, which has Frey curves from [BS].

\textbf{Questions:}
\begin{enumerate}
\item  Can we deal with $p=5$ by methods of [Bro]? Our discriminants here are always squares, which at first glance makes the use of his curve $X_{\Delta,E}$ too easy.
\item Can we utilize two Frey curves, since we have two Diophantine equations? If we assume the conjectures in Frietas, can we attack large levels with multi-Frey techniques?
\end{enumerate}


\section{Preliminaries}

We define a binary recursion relation
\[ u_{n+2} = b\cdot u_{n+1}+ c\cdot u_n. \]

We can derive a Binet-like formula as follows, the characteristic polynomial is
\[ g(z) = z^2 - bz - c\]
with roots
\[ \alpha, \beta = \frac{b \pm \sqrt{b^2+4c}}{2} \]

Then let the ``Fibonacci" sequence for this relation be define by the initial conditions
\begin{align*}
F_0 &= 0 \\
F_1 & = 1 
\end{align*}
and the ``Lucas" sequence define by the initial conditions
\begin{align*}
L_0 &= 2 \\
L_1 & = b. 
\end{align*}

Then we have the formulas:
\[F_n = \frac{\alpha^n - \beta^n}{\alpha - \beta} \qquad \qquad L_n = \alpha^n +\beta^n \]

The following general facts hold:

\begin{enumerate}

\item $F_{2k} = F_kL_k$

\item $(\alpha - \beta)^2F_n^2 = L_n^2 - 4(\alpha\beta)^n$

\item $(\alpha - \beta)^2 = b^2+4c \equiv 0,1 \pmod{4}$

\end{enumerate}

The second fact in the list allows us to generate Frey curves.  In particular, say that we posit a solution $F_n = y^p$, that is
\[ (b^2+4c)y^{2p}+4(-c)^n = L_n^2 .\]
This is a solution $(u,v,w)$ to the diophantine equation
\[ (b^2+4c)X^p +4(-c)^nY^p = Z^2 \]
with 
\[ u = y^2 \qquad v = 1 \qquad w = L_n .\]
Note that this is of the general form $Ax^p+By^p = Cz^2$.  Alternatively with a solution $L_n = y^p$ we get a solution to the diophantine equation
\[ X^p-4(-c)^nY^p = (b^2+4c)Z^2 \]
with
\[ u = y^2 \qquad v = 1 \qquad w = F_n .\]

To associate Frey curves, using the theory of Bennett and Skinner.  First, assume that $F_n$, $L_n$, and $D = b^2+4c$ are odd.  For the Fibonacci case we have the Diophantine equation:
\[(b^2+4c)y^{2p} +2^2(-c)^n = L_n^2 \]
Now we have two possibilities:

\begin{itemize}

\item if $2 |c$ then for $n \geq 5$ we can associate the Frey curve
\[ E: Y^2 +XY = X^3 +\frac{H_n -1}{4} X^2 +2^{-4}(-c)^n \]
for $H_n = \pm L_n$ such that $H_n \equiv 1 \pmod{4}$.  This gives
\[ N_E = \prod_{\ell | b^2+4c} \ell \cdot \prod_{\ell | c} \ell \cdot \prod_{\ell | y} \ell. \]
The mod $p$ Galois representation is unramified at all primes dividing $y$ by a Tate curve argument, and irreducible for $p \geq 7$ by a theorem of Mazur.  So this descends to level 
\[N = \prod_{\ell | b^2+4c} \ell \cdot \prod_{\ell | c} \ell . \]
note in particular that this is squarefree (semistable).

\item If $2 \not | c$ then we can associate the Frey curve
\[ E: Y^2 = X^3 + H_n X^2 +(-c)^n X\]
with $H_n = \pm L_n$ so that $H_n \equiv -(-c)^n \pmod{4}$.  This has conductor
\[N_E = 2^{\alpha} \cdot  \prod_{\ell | b^2+4c} \ell \cdot \prod_{\ell | c} \ell \cdot \prod_{\ell | y} \ell\] 
with
\[ \alpha = \begin{cases} 2 & : (-c)^n \equiv -1 \pmod{4} \\ 3 & : (-c)^n \equiv 1 \pmod{4} \end{cases}. \]

Again the mod $p$ Galois representation is unramified at primes dividing $y$, so this descends to level
\[ N = 2^\alpha \cdot \prod_{\ell | b^2+4c} \ell \cdot \prod_{\ell | c} \ell . \]

\end{itemize}




\section{Some Results}

Here are some (mostly trivial) results so far:

\begin{prop}
The sequence define by
\[ u_{n+2} = 3u_{n+1}-u_n \]
has no perfect powers besides $0,1,8,144$.
\end{prop}

\begin{proof}
The characteristic equation for this recursion sequence is
\[g(z) = z^2-3z+1\]
with roots
\[ \alpha, \beta = \frac{3 \pm \sqrt{5}}{2} = \omega+1, \tau+1 \]
for $\omega$ and $\tau$ the roots of $z^2-z-1$.  Note that $\omega^2 = \omega+1$, $\tau^2 = \tau + 1$.  So the terms of the sequence beginning with $u_0 = 0$ and $u_1 = 1$ are
\[ \tilde{F}_n = \frac{(\omega+1)^n - (\tau+1)^n}{\omega-\tau} = \frac{\omega^{2n} - \tau^{2n}}{\sqrt{5}} = F_{2n} \]
So the terms in this sequence are every other Fibonacci term, so the perfect powers are the same.
\end{proof}

(Note to self: there might be some hope of generalizing this: we can easily see that
\[ \omega^n = F_n\omega +F_{n-1} \]
so taking the sequence $\tilde{F}_k$ with Fibonacci starting conditions and roots of the characteristic polynomial $\omega^n$ and $\bar{\omega^n}$ we get 
\[ \tilde{F}_k = \frac{\omega^{kn} - \bar{\omega^{kn}}}{\omega^n-\bar{\omega^n}} = \frac{F_{nk}}{F_n} \]
which might show some promise... but more to think about later).


Another fruitful place to look is generalized sequences where $b^2+4c = +1$ (this is possible infinitely often as there are infinitely many odd squares).

Some examples are:

\begin{center}
\begin{tabular}{ c | c}
b & c \\ \hline \hline
3 & -2 \\
5 & -6 \\
7 & -12 \\
9 & -20 \\
11 & -30 \\
13 & -42 \\
15 & -56 \\
17 & -72 \\
19 & -90 
\end{tabular}
\end{center}

We would like to be able to do away with all of these.  We can begin with the first few.

\begin{prop}
There are no nontrivial perfect powers for in the sequence $F_n$ defined by $F_0=0$, $F_1=1$ and 
\[ F_{n+2} = 3F_{n+1}-2F_n \]
\end{prop}
\begin{proof}
Clearly, we define a nontrivial perfect power not to equal $0,1$.  Using the fact that the characteristic polynomial for this recurrence is
\[ g(z) = z^2-3z+2 = (z-2)(z-1) \]
we have the relation
\[ F_n^2 = L_n^2 - 4 \cdot 2^n \]
where $L_n$ is the generalized Lucas sequence define by the same recurrence relation but starting conditions $L_0 = 2$, $L_1 = b$.  $F_n, L_n, 2$ are all relatively prime for $n \geq 2$ as $F_n$ and $L_n$ are odd, and the above relation rules out the possibility that they are divisible by any other common primes that do not divide $2$.  Thus this is a primitive solution that would	 give rise to a Frey curve:
\[ E: Y^2 + XY = X^3 + \left(\frac{H_n-1}{4} \right)X^2 + 2^{n-4}X \]
where $H_n = \pm L_n$ so that $H_n \equiv 1 \pmod{4}$, and $n \geq 5$.  Then
\[ \Delta_E = 2^{2n-8}y^{2p} \qquad \qquad N_E = 2 \prod_{\ell | y} \ell \]
and further $\rho_{E,p}$ is irreducible and unramified outside $2p$ (weakly at $p$) for $p \geq 7$ by a Tate curve argument as well as theorems of Mazur on rational torsion.  So this descends to a representation $\rho_f \simeq \rho_{E,p}$ of some $f$ a weight 2 newform of level $2$.  As this cannot exist, there are no perfect $p$th powers for $p \geq 7$.  To check the others is a another possible computation.
\end{proof}

\begin{prop}
There are no nontrivial perfect powers in the sequence $F_n$ define by $F_0 = 0$, $F_1 = 1$ and the recursion relation
\[ F_{n+2} = 5F_{n+1}-6F_n. \]
\end{prop}

\begin{proof}
Many of the techniques of the previous problem are explicitly applicable here.  First we derive the equation
\[ F_n^2 = L_n^2 - 2^{n+2}3^n \qquad \Rightarrow \qquad y^{2p} + 2^{n+2}3^n = L_n\]
for a posited solution $F_n = y^p$.  We then check that $F_n$ and $L_n$ are not divisible by $2$ and $3$.  We associate this to the Frey curve
\[ E: Y^2 +XY = X^3 + \left(\frac{H_n-1}{4} \right)X^2 + 2^{n-4}3^nX \]
We can also calculate
\[ \Delta_E = 2^{2n-8}3^ny^{2p} \qquad \qquad N_E = 2 \cdot 3 \cdot \prod_{\ell | y} \ell .\]
By the same arguments as above, for $n \geq 5$ and $p \geq 7$, there are no nontrivial solutions as the space of newforms of level $6$ has dimension $0$.  To confirm that there are no nontrivial powers of smaller $p$ is a possible computation.
\end{proof}

This technique works for the other examples as well.  Note that we can prove in general that no terms of the generalized Fibonacci or Lucas sequences for our restricted set of $b^2+4c=1$ sequences will be divisible by the factors of $c$.  By the above equation it is clear that $b$ and $c$ are relatively prime.  Out general recurrence relation is
\[ u_{n+2} = bu_{n+1} +cu_n \]
This clearly hold for $u_2$ as $u_1=1$ so $bu_{1}$ is not divisible by any primes dividing $c$, so neither is $u_2$.  Then clearly the full claim holds by induction.  So our general solution
\[ F_n^2 = L_n^2 - 4c^n\]
does in general represent a primitive solution.  The other arguments hold as well for $n \geq 5$ and $p \geq 7$.  So we in general can descend to a level
\[ N = \rad(c) \]
which may, or may not, be populated by newforms.  For the examples above this simplifies to the following information:

\begin{center}
\begin{tabular}{ c c | c c}
$b$ & $c$ & $N$ & space of newforms  \\ \hline \hline
3 & -2 & 2 & no newforms \\
5 & -6 & 6 & no newforms\\
7 & -12 & 6 & no newforms\\
9 & -20 & 10 & no newforms\\
11 & -30 & 30 & 1D, $E = Y^2 +XY + Y = X^3+X+2$,  **check** \\
13 & -42 & 42 & 1D, $E = Y^2 +XY +Y = X^3 + X^2 +-4X+5$,   **check** \\
15 & -56 & 14& 1D, $E = Y^2 +XY+Y = X^3 +4X-6$,   **check**  \\
17 & -72 & 6 & no newforms \\
19 & -90 & 30 &  **check**
\end{tabular}
\end{center}


The fact that we can rule out $2 \cdot 3$ and $2 \cdot 5$ (and for that matter $2 \cdot 11$) easily means that if 
\[ \rad(c) = 2 \cdot 3 \text{ or } 2 \cdot 5 \text{ or }2 \cdot 11 \]
then we have a contradiction.  There are only finitely many solutions to
\[ b^2 +4c = 1 : \rad(c) = 2 \cdot 3 \text{ or } 2 \cdot 5 \text{ or }2 \cdot 11 \]
namely:
\begin{center}
\begin{tabular}{c c c}
b & c & \rad(c) \\ \hline \hline
3 & -2 & 2  \\
5 & -6 & $2\cdot$3  \\
7 & -12 & $2\cdot$3  \\
17 & -72 & $2\cdot$3 \\
9 & -20 & $2\cdot$5
\end{tabular}
\end{center}
so the above is a complete list of easy cases.  This can be summarized in the following theorem:

\begin{ithm}[No Newforms]
Let $F_n$ be the binary recursion sequence defined by 
\[ F_{n+2} = bF_{n+1} +cF_n \]
with $b^2+4c = 1$ and initial conditions $F_0 = 0$ and $F_1=1$.  Then for the following values of $b$ and $c$:
\begin{center}
\begin{tabular}{c | c }
b & c \\  \hline \hline
3 & -2 \\
5 & -6  \\
7 & -12 \\
17 & -72  \\
9 & -20 \\ \hline \hline
\end{tabular}
\end{center}
$F_n$ has no nontrivial $p$th powers for $p \geq 7$ and $n \geq 5$.
\end{ithm}



\section{All recursion relations with $b,c \leq 5$ and relatively prime}

In this section we treat all binary recurrence relations
\[ u_{n+2} = b u_{n+1} + c u_n \]
with $u_0 = 0$ and $u_1=1$ for $b,c \leq 5$ and $\gcd(b,c)=1$.
These are

\begin{center}
\begin{tabular}{c c | c c}
$|b|$ & $c$ & no perfect powers for $20 \leq n < 10000$ & perfect power for $n=$ \\ \hline \hline
1 & -5 & True & 2,7 \\
1 & -4 & True & 2 \\
1 & -3 & True & 2,5,6 \\
1 & -2 & True & 2,3,5,13 \\
1 & -1 & False & everything \\
1 & 1 & True & 2,6,12 \\
1 & 2 & True & 2 \\
1 & 3 & True & 2,3 \\
1 & 4 & True & 2,4,8 \\
1 & 5 & True & 2 \\
2 & -5 & True & 3 \\
2 & -3 & True & 3 \\
2 & -1 & False & many \\
2 & 1 & True & 7 \\
2 & 3 & True & 3 \\
2 & 5 & True & 3 \\ 
3 & -5 & True & 3 \\
3 & -4 & True & none \\
3 & -2 & True & none \\
3 & -1 & True & 3,6 \\
3 & 1 & True & none \\
3 & 2 & True & none \\
3 & 4 & True & none \\
3 & 5 & True & none \\ \hline \hline
\end{tabular}

\begin{tabular}{c c | c c}
$|b|$ & $c$ & no perfect powers for $20 \leq n < 10000$ & perfect power for $n=$ \\ \hline \hline
4 & -5 & True & 2 \\
4 & -3 & True & 2,5 \\
4 & -1 & True & 2 \\
4 & 1 & True & 2 \\
4 & 3 & True & 2 \\
4 & 5 & True & 2 \\
5 & -4 & True & none\\
5 & -3 & True & none\\
5 & -2 & True & none\\
5 & -1 & True & none\\
5 & 1 & True & none\\
5 & 2 & True & 3\\
5 & 3 & True & none\\
5 & 4 & True & none \\ \hline \hline
\end{tabular}
\end{center}

\subsection{Examples with $b^2+4c = 1$}

\begin{enumerate}

\item[\textbf{1.}] $b = 3$, $c = -2$

We have already proven that there are no prime powers for $p \geq 7$ by a Frey curve argument.  So we need only show that there are no solutions for $p = 2,3,5$.

For $p=2$ we would have a solution to
\[ x^2-y^{4} = (x+y^2)(x-y^2)= 2^{n+2} \]
where we know that $x$ and $y$ are odd for $x,y >1$.  That is
\begin{align*}
x+y^2 & = 2^{n_1} \\
x-y^2 & = 2^{n_2} \\ 
\end{align*}
where our above requirements on $x,y$ means $n_1 > n_2  \geq 1$.  So solving:
\[ x = 2^{n_1-1} +2^{n_2-1} \]
this will be even for $n_2 >1$, giving a contradiction.  So assume that $n_2 = 1$.  Then $x = 1+2^{n_1-1}$ and so 
\[ y^2 = 2^{n_1-1} - 1 \]
but as $n_1+n_2 = n >3$, $n_1 > 2$ so $4|2^{n_1-1}$ implying that $-1$ is a square mod $4$, which is a contradiction.   This argument only works for even $p$, ie $p = 2$.  \textbf{Need to do $p=3,5$}.  This follows from Catalan's conjecture.  But clearly as above we get this corollary:

\begin{cor}
No Mersenne number $2^n-1$ is a perfect power for $n>1$.
\end{cor}

which is obviously already known by Catalan's theorem.
 
 \end{enumerate}
 
 \subsection{Examples with $c=  +1$}
 
 \begin{enumerate}
 
\item[\textbf{2.}] $b = 3$, $c = 1$
In this case we have the Diophantine equation
\[13y^{2p}  \pm 4 = L_n^2\]
where it is $+4$ if $n$ is even and $-4$ if $n$ is odd.  
In addition we have the congruences
\begin{center}
\begin{tabular}{c| c c}
$n \pmod{6}$ & $F_n \pmod{4}$ & $L_n \pmod{4}$ \\ \hline \hline
0 & 0 & 2 \\
1 & 1 &  3 \\
2 & 3 & 3 \\
3 & 2 & 0 \\
4 & 1 & 3 \\
5 & 1 & 1  \\ \hline \hline
\end{tabular}
\end{center}

First we deal with the small exponents $p = 2,3,5$.  That is first we want to rule out solutions to
\[ 13y^{4} -x^{2} = \pm 4 \]


We can prove that there are no nontrivial squares $p=2$ as follows:

The Diophantine equation
\[ Ln^2 = 13y^4 \pm 4\]
gives an integral point $(13y^2, 13yL_n)$ on the elliptic curve
\[ Y^2 = X^3 \pm 4 \cdot 13 X .\]
With sage we can check that for $+52$ the only integral point is $(0:0:1)$ and with $-52$, the only integral points are $(-4:12:1), (0:0:1), (13:39:0)$.  This means that the only solution is the trivial for $y = 1, L_n = 3$.   So we conclude that there are no nontrivial squares.


**This does not give anything** In general, we may be able to do this by factoring this over $\Q(\sqrt{13})$ which has class number 1:
\[ (x + \sqrt{13} y^p)(x - \sqrt{13}y^p) = (2)^2 \]
This means that $(x+\sqrt{13}y^p)$ and $(x-\sqrt{13}y^p)$ need to be $(2 \cdot u^r)$ and $(-2 \cdot u^{-r})$ for $u$ the fundamental unit 
\[ u = \frac{3 + \sqrt{13}}{2} \]
as they are Galois conjugate.  So we add these two to get the relation
\[2x = 2(u^r-u^{-r}) \Rightarrow x = u^r-u^{-r} \]
Plugging this into our old equation we get
\[u^r - u^{-r} + \sqrt{13}y^p = 2u^r  \Rightarrow \sqrt{13}y^p = u^r + u^{-r}\]
which is the condition that
\[ y^p = \frac{u^r + u^{-r}}{\sqrt{13}} \]
which is exactly equivalent to the original formula for $F_n$.

-----


We want to rule out separately the cases of $F_n$ and $L_n$ both even (ie $n \equiv 0,3 \pmod{4}$) and both odd.

First treating the even case, we can immediately rule out the case of $n \equiv 3 \pmod{4}$ as no perfect power is $2 \pmod{4}$  This implies that $n$ is even.  Next, we define the variables
\[ z = y/2 \qquad \qquad \tilde{L} = \pm L_n/2 \text{ such that } \tilde{L} \equiv 1 \pmod{4}. \]
Then we have the relation
\[ 13 \cdot 2^{2p-2} \cdot z^{2p}+1 = \tilde{L}^2 \] 
which we associate to a Frey curve
\[E: Y^2 +XY = X^3 +\frac{\tilde{L}-1}{4} X^2 +13 \cdot 2^{2p-8} \cdot z^{2p} X .\]
This has conductor
\[ N_E = 2 \cdot 13 \cdot \prod_{\ell | z} \ell \]
and the mod $p$ representation for $p \geq 7$ descends to level $26$ as it is unramified at primes dividing $z$.  Unfortunately, $S_2(\Gamma_0(26))$ is dimension $2$, with 2 rational newforms, corresponding to the isogeny classes of elliptic curves $E^{26a}$ and $E^{26b}$ in the Cremona database.  These have representatives
\begin{align*}
E^{26a} & : y^2 + xy + y = x^3 - 5x - 8 \\
E^{26b} & : y^2 + xy + y = x^3 - x^2 - 3x + 3 \\
\end{align*}
and we can not immediately rule these out.  By the congruence conditions of \texttt{analyze\_level()} we can rule out the first of these.  So it must be isomorphic to the mod $p$ Galois representation arising from $E^{26b}$.

Now assume both $L_n$ and $y$ are odd and $n$ is odd.  We associate the Frey curve
\[ E: Y^2 = X^3 + H_nX^2 - X \]
where $H_n = \pm L_n$ such that $H_n \equiv 1 \pmod{4}$.  
This has conductor
\[ N_E = 2^2 \cdot 13 \cdot \prod_{\ell | y} \ell. \]
The mod $p$ representation $\rho_{E,p}$ is unramified at $\ell | y$ and so we conclude that for $p \geq 7$ this is the shadow of a representation $\rho_f$ for $f$ a newform of level 52.  This space is $1$-dimensional, with a single rational newform corresponding to the isogeny class of the elliptic curve
\[ E^{52a}: y^2 = x^3 + x - 10\]
in the Cremona database.

If $n$ were even then we would need
\[ E': Y^2 = X^3 + H_n'X^2 + X \]
where $H_n' = \pm L_n$ such that $H_n' \equiv -1 \pmod{4}$.  This descends to a level $N = 2^3 \cdot 13$.  The space of cusp forms of level $104$ is $2$-dimensional, consisting of one rational and one irrational newform.  These respectively correspond to an elliptic curve ($E^{104}: y^2 = x^3 + x^2 - 16*x - 32$) and abelian variety respectively.  They can both be ruled out using the method of congruences (Bennett and Skinner proposition 4.2) implemented in the sage code \texttt{analyze\_level()}.



\item[\textbf{3.}] $b = 2$, $c = 1$

In this case we are working with the Diophantine equation
\[ 8y^{2p} \pm 4 = L_n^2 \]
where the sequence $L_n$ has initial conditions $L_0 = 2$ and $L_2 = 2$.  The congruences look like
\begin{center}
\begin{tabular}{c| c c}
$n \pmod{4}$ & $F_n \pmod{4}$ & $L_n \pmod{4}$ \\ \hline \hline
0 & 0 & 2 \\
1 & 1 &  2 \\
2 & 2 & 2\\
3 & 1 & 2 \\ \hline \hline
\end{tabular}
\end{center}

No perfect power is $2 \pmod{4}$, so $L_n$ is never a perfect power.  In addition $F_n$ is not a perfect power for $n \equiv 2 \pmod{4}$.  Assume that $n > 1$.

Assume that $n \equiv 0 \pmod{4}$.  Then let $z = y/2$ and $\tilde{L} = \pm L_n/2$ so that $\tilde{L} \equiv 1 \pmod{4}$; we get the Diophantine equation
\[ 2^{2p+1}z^{2p} + 1 = \tilde{L}^2 \]
which we associate to the elliptic curve
\[ E: Y^2 +XY = X^3 + \frac{\tilde{L}-1}{4} X^2 + 2^{2p-5}z^{2p}X \]
which has conductor 
\[N_E = 2 \cdot \prod_{\ell | z} \ell \]
unramified at all primes dividing $z$, meaning we arrive at a contradiction.  

Similarly for $ n \equiv \pm 1 \pmod{4}$, we again use $\tilde{L} = L_n/2$.  We have the Diophantine equation
\[ 2y^{2p} -1 = \tilde{L}^2 \] 
which we associate to the Frey curve
\[ E: Y^2 = X^3 + 2\tilde{L}X^2+2y^{2p}X \]
which has conductor
\[N_E = 2^7 \cdot \prod_{\ell |y} \ell. \]
It descends to level $2^7 = 128$.  The space of newforms of level $128$ is dimension $4$, spanned by $4$ rational newforms.   

We can actually rule this out using the theorem that there are only two solutions to
\[ x^2 - 2y^{2p} = \pm 1\]
namely $(1,1,p)$ and $(234, 13, 2)$ (cited in Bennett, but due originally to Petho and Cohn).

This is the Diophantine equation one arrives at by dividing the original Diophantine equation by $4$.


\item[\textbf{4.}] $b = 4$, $c = 1$

The Diophantine equation here is 
\[ 20y^{2p} \pm4 = L_n^2\]
with congruences
\begin{center}
\begin{tabular}{c| c c}
$n \pmod{2}$ & $F_n \pmod{4}$ & $L_n \pmod{4}$ \\ \hline \hline
0 & 0 & 2 \\
1 & 1 &  0 \\ \hline \hline
\end{tabular}
\end{center}

We we can divide our attention into the cases of even and odd $n$.  In the even $n$ case first, note that $L_n$ is not a perfect power.  In this case, both $y$ and $L_n$ are even so we employ the same trick as above and change variables to $z = y/2$ and $\tilde{L} =  L_n/2$:
\[ 2^{2p} \cdot 5 z^{2p} + 1 = \tilde{L}^2. \]
We associate this to the elliptic curve 
\[ E: Y^2 + XY = X^3 + \frac{\tilde{L} -1}{4} X^2 + 2^{2p-6} \cdot 5 z^{2p} X \]
with conductor
\[N_E = 2 \cdot 5 \cdot \prod_{\ell | z} \ell \]
the mod $p$ Galois representation of which descends to level $10$, where there are no newforms, so we reach a contradiction.

Now for the case that $n$ is odd, we get the Diophantine equation
\[ 5 \cdot y^{2p} -1 = \tilde{L}^2 \]
which we associate to the Frey curve
\[ E: Y^2 = X^3 + 2\tilde{L} X^2 - X \]
with conductor
\[N_E = 2^5\cdot 5 \prod_{\ell | y} \ell. \]
As above the Galois representation descends to level $2^5 \cdot 5 = 160$, which is 4 dimensional, spanned by $2$ rational newforms and one irrational.   The abelian variety can be ruled out easily using \texttt{analyze\_level()}.

\item[\textbf{4.}] $b = 5$, $c = 1$

In this case we have the relation 
\[ 29y^{2p} \pm 4 = L_n^2 \]
with the congruence conditions:
\begin{center}
\begin{tabular}{c| c c}
$n \pmod{6}$ & $F_n \pmod{4}$ & $L_n \pmod{4}$ \\ \hline \hline
0 & 0 & 2 \\
1 & 1 &  1 \\
2 & 1 & 3 \\
3 & 2 & 0 \\
4 & 3 & 3 \\
5 & 1 & 3  \\ \hline \hline
\end{tabular}
\end{center}

We must again separate by what residue classes of $n$ we want to consider.  

\end{enumerate}


\subsection{Examples with $c = -1$}

\begin{enumerate}

\item[\textbf{5.}] $b = 4, c = -1$: This is in some sense the simplest case for $b$ even as the case $2, -1$ has many nontrivial perfect powers as $b^2+4c = 0$.

We have the Diophantine equation
\[ 12y^{2p} - 4 = L_n^2 \]
with the congruence conditions (and levels descended to) given by the Sage script \texttt{find\_levels()}.

\begin{lstlisting}

D is 12
For term 0 mod 4 F is 0 mod 4 and L is 2 mod 4; it descends to level 6
For term 1 mod 4 F is 1 mod 4 and L is 0 mod 4; it descends to level 96
For term 2 mod 4 F is 0 mod 4 and L is 2 mod 4; it descends to level 6
For term 3 mod 4 F is 3 mod 4 and L is 0 mod 4; it descends to level 96


\end{lstlisting}

So for $n \equiv 0, 2 \pmod{4}$, $F_n$ has no nontrivial perfect powers for $p \geq 7$.  To rule out the other cases, we consider the fact that $D$ and $L_n$ are even.  So letting $x = L_n/2$ we get the Diophantine equation:
\[ x^2 - 3y^{2p} = 1. \]
By Thm 7.1 in Bennett (2004), the only nontrivial solution is 
\[ 7^2 -3 \cdot 2^4 = 1\]
and as $16$ is not a term in the Fibonacci sequence this is not possible.  **need to make this slightly less sketchy**


\item[\textbf{6.}] $b = 6, c = -1$: A similar case holds here.  We get the congruences by \texttt{find\_levels()}

\begin{lstlisting}

D is 32
For term 0 mod 4 F is 0 mod 4 and L is 2 mod 4; it descends to level 2
For term 1 mod 4 F is 1 mod 4 and L is 2 mod 4; it descends to level 32
For term 2 mod 4 F is 2 mod 4 and is not a perfect power
For term 3 mod 4 F is 3 mod 4 and L is 2 mod 4; it descends to level 32

\end{lstlisting}

If $F_n \equiv 2 \pmod{4}$, it is clearly not a perfect power.  If $F_n \equiv 0 \pmod{4}$ then for $p \geq 7$ we can rule this out by the modular method.

For $F_n \equiv 1, 3 \pmod{4}$ we can transform this as above
\[ x^2 - 8y^{2p} = 1 \]
which by Bennett (2004) Thm 7.1 has no solutions.


\end{enumerate}

\section{Reduction to prime index when $c = 1, d = b^2+4$ prime}

\begin{lem}
$(F(n), \frac{F(kn)}{F(n)}) | k$
\end{lem}
\begin{proof}
We have verified that Halton's General Fibonacci Identity works for general Lucas sequences. This is surely well known, but we have not seen it in the literature. Then identity (34) in that paper proves Lemma 16 in his Divisibility Properties paper, which is exactly this lemma.
\end{proof}

\begin{lem}
Let $K = \mathbb{Q}(\sqrt{-D}), L = \mathbb{Q}(\sqrt{-2D})$. If $p \nmid h_Kh_L$, then $L(2m) = y^p + 1$ has no solutions for $m > 0$.
\end{lem}
\begin{proof}
We have verified that the proof in [Steiner] for $b=1, c=1$ works for general Lucas sequences, given these hypotheses.
\end{proof}
We only want to use this for $m=3$, although it is not strong enough, due to the hypotheses.


We verify that the argument of Robbins for the Fibonacci sequence applies more generally.

\begin{prop}
Let $p \geq 5$ a prime. Suppose $b^2 + 4 = d < 100000$, d prime. If $b \neq 1$, $b not a perfect power$, then there are no solutions $F(2n) = c^p$, $c > 1$, such that $F(m) \neq z^p$ for $m < 2n$. (ie a minimal index solution has odd index)
\end{prop}
\begin{proof}
We consider $F(n)$ mod 4, hence n mod 6.
\begin{itemize}
\item{$n \not\equiv 0 \mod 3$}

$c^p = F(2n) = F(n)L(n)$, and $(F(n), L(n)) = 1$, so $F(n) = x^p$, contradicting minimality unless $x=1$, in which case $n=1,2$ for $b=1$, $n=1$ for $b>1$. Since $b$ is not a perfect power by assumption, $n \neq 1$. Thus $c^p = F(4) = L(2)F(2) = b(b^2 + 2)$. But $L(2), F(2)$ coprime for $b \neq 2$, which implies $b, b^2+2$ are perfect powers, hence $b=1$ by minimality. But $b^2+2=3$ is then not a perfect power.
 
\item{$n \equiv 3 \mod 6$}

The period mod 4 tells us that $F(n) = 2r, L(n) = 4s, (r,s)=1, rs odd$. Thus $c^p = F(2n) = 2^3rs$, so $p|3$, hence $p=3$. 

\item{$n \equiv 0 \mod 6$}

Let $n = n_0 = 2^j3^kt$ for $j,k \geq 1, (6, t) = 1$. Since $b$ odd, the mod 8 period tells us that $F(n_0) = 2^{2+j}r_0^p, L(n_0)=2s_0^p,$ $r_0s_0$ odd, $(r_0,s_0)=1$. Let $n_i = 2^{-i}n$. Inducting, we see $F(n_j) = 2r_j^p$ (as $3|n_i$ implies $2|F(n_i)$, and $(F(k), L(k)) | 2$ ), and so $L(n_j) = 4s_j^p$, with $r_is_i$ odd, $(r_i, s_i) = 1$.

Let $h_0 = n_j, h_i = 2^{-i}h_0$, so $F(h_0) = 2u_0^p$. Let $g = (F(h_i), \frac{F(h_{i-1})}{F(h_i)})$. By lemma above, $g|3$. If $g = 3$, then $9 | F(3h_i)$. We checked the mod 9 periods for $b=1,\cdots,17$ odd, and thus for all odd $b$, and see that $9 \nmid F(3k)$ for k odd. Thus $g=1$. Hence we induct, and get $F(h_i) = 2u_i^p, \frac{F(h_{i-1})}{F(h_i)} = v_i^p$ for $i=0,\cdots,k-1$, $F(t) = F(h_k) = u_k^p, \frac{F(h_{i-1})}{F(h_i)} = 2v_k^p$ for $(u_i, v_i) = 1, u_i v_i$ odd.

Since $u_k^p = F(t) < F(6t) \leq F(n) < F(2n) = c^p$, the minimality of the index implies $u_k = 1, t = 1$. If $k \geq 2$, then $\frac{F(h_{k-2})}{F(h_{k-1})} = \frac{F(9)}{F(3)} = v_{k-1}^p$. If $k = 1, j \geq 2$, then $L(n_{j-2}) = L(12) = 2 s_{j-1}^p$. If $k = j = 1$, then $n = 6$, $F(12) = c^p$. 

Now, $F(12) = F(6)L(6)$, and as before, the mod 8 period says $8 || F(6), 2 || L(6)$, so $c^p = 2^4rs$ with $rs$ odd, so $p|4$, $p=2$.

At this point, we run a computer search. For $1 \leq d \le 10^5$, we find that none of $2^{-1}L(12), \frac{F(9)}{F(3)}$ are perfect powers, with the exception of the Fibonacci sequence itself, where only $F(12)$ is a perfect power.
\end{itemize}
\end{proof}

\begin{rem}
Note that $\frac{F(9)}{F(3)} = L(6) - 1$, so by above lemma, $\frac{F(9)}{F(3)} = c^p$ has no solutions for $p$ odd, not dividing the class numbers of either $\mathbb{Q}(\sqrt{-D}), \mathbb{Q}(\sqrt{-2D})$. This is not helpful yet, but perhaps if we could show that these had only arbitrary large primes relative to $D$. In particular, I could get a contradiction if I could bound the prime divisors of those class numbers below by \[\log(3)\log\left(\left(\frac{1+\sqrt{D}}{2}\right)^6 + \frac{1}{2}\right),\]
i.e. approximately
\[ C\log(D)\]
Perhaps looks at:

Kohnen-Ono - Indivisibility of Class Numbers

Cohen-Lenstra Heuristics 

Goldfeld-Gross-Zagier - Gauss' Class Number Problem
\end{rem}

\begin{prop}With the same conditions on $b$, if $F(m) = y^p$ has a solution with $m$ odd, $p \geq 5$, then there exists a solution $F(n) = z^p$ with n prime.
\end{prop}
\begin{proof}
Robbins proof for Fibonacci works for our sequences as well, since d is prime, and we check that $\frac{F(d^2)}{d^2}$ is not a perfect power for $d < 100000$.
**VERIFY If $p < q$ primes, $p \mid F(q^k)$, then p = 2, q = 3.**
\end{proof}
\begin{rem}
How on earth could you go about proving that $\frac{F(d^2)}{d^2}$ is not a perfect power for all d? It is a degree $2^{d^2-2}$ polynomial in $b$!
\end{rem}

\section{Kraus' Method applied to b odd, c=1, $F_n$ odd}
We consider $b$ odd.
We deal with odd solutions to $F_n = y^p$, $n$ an odd prime, $p \geq 7$. We record the parity of the n, as we have to apply Kraus' method separately in each case.

The restriction to $F_n$ odd may not be too problematic. Isabel believes that a paper of Bennett rules out solutions to $X^2-DY^{2p}=1$, which applies here to rule out the case $F_n$ even, n even. Further, Ribenboim tells us that:
\begin{enumerate}
\item If b odd, then $F_n$ even implies $3 | n$
\item If b even, then Fn is never even.
\end{enumerate}
Note that this implies that the period mod 4 is 6 when b is odd, is 2 or 4 when b is even. Thus in our situation, an even solution would require $3|n$. We may be able to rule such indices out using Robbins' method when 3 does not divide $D$.

\begin{enumerate}
\item{b=3, c=1}
  Empty levels. Thus no solutions $y >= 2, p >= 7$
    
\item{b=5, c=1} 
\begin{itemize} 
  \item{n=1}
    Conditional success. If:
      $Fn = y^p$ has a solutions for $y \geq 6$ n = 1 mod 2 p in [7, 1000]
    then:
      $n = \pm 1$ mod p
  \item{n=0}
    An abelian variety found with bad prime p=7.
    No elliptic curves found.
\end{itemize}  
\item{b=7, c=1}
\begin{itemize}
  \item{n=0}
    An abelian variety found with bad prime p=11.
    No elliptic curves found.
  \item{n=1}
    An abelian variety found with bad primes $3^3 \cdot 7^2 \cdot 11$
    We cannot descend to an elliptic curve,
      for $y \geq 8, 7 \leq p \leq 1000$.
\end{itemize}
\item{b=9, c=1}
\begin{itemize}
  
  \item{n=1}
    An abelian variety found with bad primes $2^{10} 3^4 7$
    Conditional success. If:
      $Fn = y^p$ has no solutions for $y \geq 10$, n = 1 mod 2 p in [7, 1000]
      and
      we descend to an elliptic curve
    then:
      $n = \pm 1$ mod p
  \item{n=0}
    Three abelian varieties found, with bad primes:	
      $2^5 \cdot 3 \cdot 11, 2^5 \cdot 7^2, 2^{10} \cdot 7$
    We cannot descend to an elliptic curve,
      for $y \geq 10, 7 \leq p \leq 1000$.

\end{itemize}
\end{enumerate}

\section{In general for $c = \pm 1$}

We have several possibilities based upon congruences:

\begin{enumerate}


\item We can immediately rule out the possibility of $F_n \equiv 2 \pmod{4}$ as being a perfect power

\item It is not possible that both $F_n$ and $L_n$ are $F_n, L_n \equiv 0 \pmod{4}$ (as can be seen by reducing mod $16$ on both sides of the standard Diophantine equation)

\item $F_n \equiv 0,2 \pmod{4}$ and $L_n \equiv 2 \pmod{4}$

We simplify our Diophantine equation using $z = y/2$ and $\tilde{L} = L/2$:
\[ (b^2+4c) 2^{2p-2} z^{2p} \pm 1 = \tilde{L}^2 \]
where it is always $+1$ if $c = -1$; then associate it to the Frey curve
\[E: Y^2 +XY = X^3 + \frac{\tilde{L} -1}{4} X^2  + (b^2+4c) 2^{2p-8} z^{2p}X \]
with conductor
\[N_E = 2 \cdot \prod_{\substack{\ell | b^2+4c \\ \ell \notdiv 2}} \ell \]
ie it is square free.

\item $F_n$ and $L_n$ are odd and $b^2+4c$ is odd

Then we have the usual Diophantine equation 
\[ (b^2+4c)y^{2p} \pm 4 = H_n^2. \]
For $c = 1$ it is $+4$ if $n$ is even and $-4$ if $n$ is odd and $H_n = \pm L_n$ such that $H_n \equiv -1\pmod{4}$ if $n$ is even and $-1 \pmod{4}$ if $n$ is odd.  If $c = -1$ then it is $+4$, so we choose $H_n \equiv -1 \pmod{4}$.


We associate this to the elliptic curve
\[ E: Y^3 = X^3 + H_n X^2 \pm X \]
where the $\pm$ corresponds to that in the original Diophantine equation.  This has conductor
\[N_E = 2^{\alpha} \cdot  \prod_{\substack{\ell | b^2+4c \\ \ell \notdiv 2}} \ell \cdot \prod_{\ell | y} \ell \]
with 
\[ \alpha = \begin{cases} 2 &: c = +1 \text{ and } n \text{ odd} \\ 3 &:   c= +1 \text{ and } n \text{ even} \text{ or } c = -1 \end{cases}. \]

\item It is not possible for $F_n$ and $L_n$ to be odd and $b^2+4c$ to be even (reduce the Diophantine equation mod $2$)

\item $F_n$ is odd and $L_n$ and $b^2+4c$ are even

Let $k$ be the order of $2$ dividing $b^2+4c$ such that $b^2 + 4c = 2^k \cdot A$.  Note that $k \geq 2$ as $b^2 + 4c \equiv 0,1 \pmod{4}$.  Then we again split into several cases:

\begin{enumerate}

\item $k = 2$:  We write our Diophantine equation as
\[ A y^{2p} \pm 1 = \tilde{L}^2 \]
where as above $\tilde{L} = \pm L_n/2$.  For $c=  +1$, we can also say that $A \equiv \mp 1 \pmod{4}$ and $y^{2p} \equiv 1 \pmod{4}$ by congruences reasons.  We then associate the following Frey curve

\[ E : \begin{cases} Y^2 = X^3 + 2 \tilde{L}X^2 + A X & : A \equiv -1 \pmod{4} \qquad (c = -1 \text{ and } n \text{ even or } c = -1) \\  Y^2 = X^3 + 2 \tilde{L}X^2 - X & : A \equiv 1 \pmod{4} \qquad (c = -1 \text{ and } n \text{ odd}) \end{cases} \]

These have conductor

\[ N_E =  2^5 \cdot \prod_{\ell |  A} \ell \cdot \prod_{\ell | y} \ell \]


\item $k =3$: we write our Diophantine equation as
\[ 2A y^{2p} \pm 1 = \tilde{L}^2 \]
which we associate the the Frey curve
\[ E : Y^2 = X^3 + 2\tilde{L}X^2 + 2Ay^{2p} X \]
which has conductor
\[ N_E = 2^7 \cdot \prod_{\ell | A} \ell \cdot \prod_{\ell | y} \ell. \]

\item $k = 4$: we write our Diophantine equation as
\[ 2^2A y^{2p} + 1  = \tilde{L}^2. \]
Note that in this case it is clear that $n$ is even as $\tilde{L}^2 \equiv 1 \pmod{4}$.  We choose $\tilde{L}$ such that it is congruent to $-A \pmod{4}$.  We can then associate this to the Frey curve given by
\[ E: Y^2 = X^3 + \tilde{L}X^2 + Ay^{2p}X \]
which has conductor
\[N_E = 2^\alpha \cdot \prod_{\ell | A} \ell \cdot \prod_{\ell | y} \ell \]
with 
\[ \alpha = \begin{cases} 2 & : F_n \equiv -A \pmod{4} \\ 3 & : F_n \equiv  A \pmod{4} \end{cases} .\]


\item $k=5, 6, 7$: we write our Diophantine equation as
\[ 2^{k-2}A y^{2p} \pm 1 = \tilde{L}^2 \]
note that in this case $\tilde{L} \equiv 1 \pmod{4}$.  We associate this to the Elliptic curve
\[ E: Y^2 = X^3 + \tilde{L}X^2 + 2^{k-4}A y^{2p} X \]
with conductor
\[ N_E = 2^\alpha \cdot \prod_{\ell | A} \ell \cdot \prod_{\ell | y} \ell \]
where
\[ \alpha = \begin{cases} 5 & : k = 5 \\ 3 & : k = 6,7 \end{cases} .\]

\item $k \geq 8$ : we write
\[ 2^{k-2} A y^{2p} \pm 1  = \tilde{L}^2 \]
where we take $\tilde{L} \equiv 1 \pmod{4}$.  We associate to this the elliptic curve
\[ E: Y^2 + XY = X^3 + \frac{\tilde{L} -1}{4}X^2 + 2^{k-8}Ay^{2p}X \]
with conductor
\[ N_E = 2^\alpha \cdot \prod_{\ell | A} \ell \cdot \prod_{\ell | y} \ell \]
where 
\[ \alpha = \begin{cases} 0 & : k = 8 \\ 1 & : k > 8 \end{cases} .\]



\end{enumerate}

All of this can be easily made into a sage algorithm see appendix \ref{findlevels}.
\end{enumerate}

\section{Some rough bounds using linear forms in logarithms}

\subsection{$b^2+4c = 1$}

In this case the Diophantine equation reduces to:
\[ x^n - (x-1)^n = y^p \]
for $x = \alpha = \beta + 1$ with $\alpha$ and $\beta$ the roots of the characteristic equation.  We do the following manipulations to get reasonable bounds using linear forms in logarithms.  Note that these are \emph{rough} bounds.  Better can be found.

First we simplify:
\[ 1 - \pfrac{x-1}{x}^n = \frac{y^p}{x^n}. \]
Thus letting
\[ \Lambda = p\log{y} - n\log{x} \]
we get
\[ |\Lambda| = \left| \log{ \left( 1 - \pfrac{x-1}{x}^n \right) } \right| < 2\pfrac{x-1}{x}^n < 2. \]
for $n$ sufficiently large (depending upon $x$).

This implies
\[ p \log{y} - 2 < n\log{x} < p\log{y} +2 \]
and thus
\[ \frac{p}{\log{x}} - \frac{2}{\log{x}\log{y}} < \frac{n}{\log{y}} < \frac{p}{\log{x}} + \frac{2}{\log{x}\log{y}}. \]

Now we consider 
\[ \log{|\Lambda|} < n \log{((x-1)/x)} +1 = -n\log{(x/x-1)}+1 = -n\log{x} + n\log{(x-1)} +1. \]
Simplifying the above, we use the lower bound for $n\log{x}$ here to get 
\[ \log{|\Lambda|} < -p\log{y} + n\log{(x-1)} +3. \]

Now we turn to a lower bound using the results of Mignotte, Laurent, and Nesterenko (1995) Cor 2, we set
\[ b' = \frac{n}{\log{y}} + \frac{p}{\log{x}} .\]
Then for $p$ large enough, we get the lower bound
\[ -24.34 \left( \log{b'}+.14 \right)^2 \cdot \log{x} \cdot \log{y} \leq \log|\Lambda|. \]
We can simplify $b'$ further using the above bounds.
\[ b' < \frac{2p}{\log{x}} + \frac{2}{\log{x}\log{y}} \]
so
\[ -24.34 \left( \log{\left( \frac{2p}{\log{x}} + \frac{2}{\log{x}\log{y}} \right)}+.14 \right)^2 \cdot \log{x} \cdot \log{y} < \log|\Lambda|. \]

Putting together the upper and lower bounds we get

\[ -24.34 \left( \log{\left( \frac{2p}{\log{x}} + \frac{2}{\log{x}\log{y}} \right)}+.14 \right)^2 \cdot \log{x} \cdot \log{y} < \log|\Lambda| < -p\log{y} + n\log{(x-1)} +3. \]
or simply
\[ -24.34 \left( \log{\left( \frac{2p}{\log{x}} + \frac{2}{\log{x}\log{y}} \right)}+.14 \right)^2 \cdot \log{x} \cdot \log{y} <  -p\log{y} + n\log{(x-1)} +3. \]
which implies
\[ -24.34 \left( \log{\left( \frac{2p}{\log{x}} + \frac{2}{\log{x}\log{y}} \right)}+.14 \right)^2<  -\frac{p}{\log{x}} + \frac{n}{\log{y}} \cdot \frac{\log{(x-1)}}{\log{x}} + \frac{3}{\log{x}\log{y}}. \]
Substituting
\[ -24.34 \left( \log{\left( \frac{2p}{\log{x}} + \frac{2}{\log{x}\log{y}} \right)}+.14 \right)^2<  -\frac{p}{\log{x}} + \left( \frac{p}{\log{x}} + \frac{2}{\log{x}\log{y}} \right) \cdot \frac{\log{(x-1)}}{\log{x}} + \frac{3}{\log{x}\log{y}}. \]

Or more compactly:
\[ -24.34 \left( \log{\left(\beta \right)}+.14 \right)^2 < - \left( \beta/2 - \frac{1}{\log{x}\log{y}} \right)\left( 1 - \frac{\log{(x-1)}}{\log{x}} \right) + \frac{1}{\log{x}\log{y}} \left( \frac{2\log{(x-1)}}{\log{x}} +3 \right). \]
Which is equivalent to 
\[ \left( \beta/2 - \frac{1}{\log{x}\log{y}} \right)\left( 1 - \frac{\log{(x-1)}}{\log{x}} \right) - \frac{1}{\log{x}\log{y}} \left( \frac{2\log{(x-1)}}{\log{x}} +3 \right) < 24.34 \left( \log{\left(\beta \right)}+.14 \right)^2. \]

We expect this to give a contradiction as $\beta$ (ie $p$) increases as the LHS is linear in $\beta$ and the RHS is $\log^2(\beta)$.  Note that as $x$ increases the bound will get worse.  Similarly, as $y$ increases the bound will get better.  Choosing the worst possible value of $y=2$ we get the following inequality:
\[ \left( \beta/2 - \frac{1}{\log{x}\log{2}} \right)\left( 1 - \frac{\log{(x-1)}}{\log{x}} \right) - \frac{1}{\log{x}\log{2}} \left( \frac{2\log{(x-1)}}{\log{x}} +3 \right) < 24.34 \left( \log{\left(\beta \right)}+.14 \right)^2. \]

Using this we get the following bounds:
\begin{center}
\begin{tabular}{c c}
$x$ & $p <$ \\ \hline \hline
$2$ & $1153$ \\
$3$ & $4167$ \\
$4$ & $8567$ \\
$5$ & $14092$ \\
$6$ & $20570$ 
\\ \hline \hline
\end{tabular}
\end{center}

We could derive better bounds on $p$ using the Theorem 1 of [LMN].  This will be necessary, as the bound on $n$ in terms of $p$ is of the order $p^{10p}$.



\pagebreak

\appendix


\section{Find Levels}\label{findlevels}

\begin{lstlisting}

def find_period_fib(b,c):
    R = Integers(4)
    f0 = 0; f1 = 1
    A1 = 0; A2 = 1
    for i in xrange(20):
        x = b*A2 + c*A1
        A1 = A2
        A2 = x
        if (i>= 1) and R(A1) == R(f0) and R(A2) == R(f1):
            return i+1
            
def find_period_luc(b,c):
    R = Integers(4)
    l0 = 2; l1 = b
    A1 = 2; A2 = b
    for i in xrange(20):
        x = b*A2 + c*A1
        A1 = A2
        A2 = x
        if (i>= 0) and R(A1) == R(l0) and R(A2) == R(l1):
            return i+1
            
def  get_congruences(b,c):
    R = Integers(4)
    f0 = 0; f1 = 1
    l0 = 2; l1 = b
    fib_per = find_period_fib(b,c)
    luc_per = find_period_luc(b,c)
    m = lcm(fib_per, luc_per)
    F = [R(0), R(1)]
    L = [R(2), R(b)]
    for i in xrange(m-2):
        x = b*f1 + c*f0
        f0 = f1
        f1 = x
        y = b*l1 + c*l0
        l0 = l1
        l1 = y
        F.append(R(x))
        L.append(R(y))
    return F,L
    
def find_levels(b,c):
    sign = (1-c)/2
    R = Integers(4)
    F,L = get_congruences(b,c)
    m = len(F)
    D = b^2+4*c
    print "D is", D
    for n in xrange(m):
        N = 0
        f = F[n]
        l = L[n]
        if R(f) == R(2):
            print "For term", n, "mod", m, "F is",f, "mod 4 and is not a perfect power"
            continue
        elif GF(2)(f) == GF(2)(0) and R(l) == R(2):
            j = (radical(D)).valuation(2)
            radD = radical(D)/2^j
            N = 2*radD
        elif GF(2)(f) == GF(2)(1) and GF(2)(l) == GF(2)(1) and GF(2)(D) == GF(2)(1):
            if (GF(2)(n) == GF(2)(0)) or sign:
                N = 2^3*radical(D)
            elif GF(2)(n) == GF(2)(1):
                N = 2^2*radical(D)
        elif GF(2)(f) == GF(2)(1) and GF(2)(l) == GF(2)(0) and GF(2)(D) == GF(2)(0):
            k = D.valuation(2)
            A = D/2^k
            if k == 2:
                N = 2^5*radical(A)
            if k == 3:
                N = 2^7*radical(A)
            if k == 4:
                if R(-A) == R(f):
                    N = 2^2*radical(A)
                if R(A) == R(f):
                    N = 2^3*radical(A)
            if k == 5:
                N = 2^5*radical(A)
            if k == 6 or k == 7:
                N = 2^3*radical(A)
            if k == 8:
                N = radical(A)
            if k >8:
                N = 2*radical(A)
        print "For term", n, "mod", m, "F is", f, "mod 4 and L is", l, "mod 4; it descends to level", N

\end{lstlisting}

\section{Analyze Levels}

\end{document}