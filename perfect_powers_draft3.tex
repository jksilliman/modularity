\documentclass[12pt]{amsart}
\usepackage{latexsym}
\usepackage{amssymb,amsmath}
\usepackage[pdftex]{graphicx}
\usepackage{enumerate}
\usepackage{endnotes}
%\usepackage{extpfeil}
\usepackage{hyperref}
\usepackage[usenames,dvipsnames]{xcolor}
\usepackage{stackrel}
\usepackage{bbm}
\usepackage{tikz}
\usepackage[margin=1.25in]{geometry}
\usepackage{hyperref}
\usepackage{listings}
\usepackage{courier}
\usepackage{color}
\usepackage{upgreek}

\lstset{
	basicstyle=\small\ttfamily,
	keywordstyle=\color{blue},
	language=python,
	xleftmargin=16pt,
}

\usetikzlibrary{arrows,chains,matrix,positioning,scopes}

\makeatletter
\tikzset{join/.code=\tikzset{after node path={%
\ifx\tikzchainprevious\pgfutil@empty\else(\tikzchainprevious)%
edge[every join]#1(\tikzchaincurrent)\fi}}}
\makeatother
%
\tikzset{>=stealth',every on chain/.append style={join},
         every join/.style={->}}
\tikzstyle{labeled}=[execute at begin node=$\scriptstyle,
   execute at end node=$]
\usetikzlibrary{patterns}

\usetikzlibrary{decorations.pathreplacing}

\DeclareSymbolFont{bbold}{U}{bbold}{m}{n}
\DeclareSymbolFontAlphabet{\mathbbold}{bbold}

\newtheorem{thm}{Theorem}[section]
\newtheorem{ithm}{Theorem}
\newtheorem{lem}[thm]{Lemma}
\newtheorem{conj}[thm]{Conjecture}
\newtheorem{prop}[thm]{Proposition}
\newtheorem{cor}[thm]{Corollary}

\theoremstyle{definition}
\newtheorem{defi}[thm]{Definition}
\newtheorem{example}[thm]{Example}
\newtheorem{exercise}[thm]{Exercise}
\newtheorem{rem}[thm]{Remark}


   
\def\B{{\mathbb B}}
\def\C{{\mathbb C}}
\def\D{{\mathbb D}}
\def\Fp{{\mathbb F}_p}
\def\Fell{{\mathbb F}_{\ell}}
\def\F{{\mathbb F}}
%\def\H{{\mathbb H}}
\def\M{{\mathbb M}}
\def\N{{\mathbb N}}
\def\O{{\mathcal O}}
\def\0{{\mathbb 0}}
\def\P{{{\mathbb P}}}
\def\Q{{\mathbb Q}}
\def\R{{\mathbb R}}
\def\T{{\mathbb T}}
\def\Z{{\mathbb Z}}

\newcommand{\sol}{_{a^p,b^p,c^p}}
\newcommand{\bound}{\partial}
\newcommand{\la}[1]{\mathfrak{#1}}
\newcommand{\im}{\text{Im} \hspace{0.1em} }
\newcommand{\ann}{\text{Ann} \hspace{0.1em} }
\newcommand{\rank}{\text{rank} \hspace{0.1em} }
\newcommand{\coker}[1]{\text{coker}\hspace{0.1em}{#1}}
\newcommand{\sgn}{\text{sgn}}
\newcommand{\lcm}{\text{lcm}}
\newcommand{\re}{\text{Re}  \hspace{0.1em} }
\newcommand{\ext}[1]{\text{Ext}(#1)}
\newcommand{\Hom}[1]{\text{Hom}(#1)}
\newcommand{\End}[1]{\text{End(#1)}}
\newcommand{\bs}{\setminus}
\newcommand{\rpp}[1]{\mathbb{R}\text{P}^{#1}}
\newcommand{\cpp}[1]{\mathbb{C}\text{P}^{#1}}
\newcommand{\tr}{\text{tr}\hspace{0.1em} }
\newcommand{\inner}[1]{\langle {#1}\rangle}
\newcommand{\tensor}{\otimes}
\newcommand{\Cl}{\text{Cl}}
\renewcommand{\sp}[1]{\text{Sp}_{#1}}
\newcommand{\GL}{\text{GL}}
\newcommand{\PGL}{\text{PGL}}
\renewcommand{\sl}[1]{\text{SL}_{#1}}
\newcommand{\so}[1]{\text{SO}_{#1}}
\newcommand{\SO}{\text{SO}}
\newcommand{\pso}[1]{\text{PSO}_{#1}}
\renewcommand{\o}[1]{\text{O}_{#1}}
\renewcommand{\sp}[1]{\text{Sp}_{#1}}
\newcommand{\psp}[1]{\text{PSp}_{#1}}
\newcommand{\Span}{\rm Span}
\newcommand{\Frob}{\rm Frob}
\newcommand{\tor}{\rm tor}
\newcommand{\rad}{\rm rad}
\newcommand{\denom}{\rm denom}
\renewcommand{\bar}{\overline}
\newcommand{\notdiv}{\nmid}
\newcommand{\pfrac}[2]{\left( \frac{#1}{#2} \right)}
\newcommand{\bfrac}[2]{\left| \frac{#1}{#2} \right|}
\newcommand{\Ell}{\rm Ell}
\newcommand{\AV}{\rm AV}
\newcommand{\Gal}{\rm Gal}

\newcommand{\kron}[2]{\bigl(\frac{#1}{#2}\bigr)}
\newcommand{\leg}[2]{\Biggl(\frac{#1}{#2}\Biggr)}

\DeclareSymbolFont{bbold}{U}{bbold}{m}{n}
\DeclareSymbolFontAlphabet{\mathbbold}{bbold}

\begin{document}

\title{Perfect Powers in Lucas Sequences via Galois Representations}
\author{Jesse Silliman and Isabel Vogt}

\maketitle


\section{Introduction}
The Fibonacci Sequence, perhaps the simplest linear recurrence relation, begins as \[0,\underline{1},\underline{1},2,3,5,\underline{8},13,21,34,55,89,\underline{144},233.\] Note that the underlined terms are perfect powers. It was a folklore conjecture that these terms are in fact the only perfect powers in the Fibonacci sequence. Various partial results in this direction, using classical methods, ruled out other $p$-th powers for small primes $p$. Finally, in a deep paper \cite{siksek06}, the conjecture was proven. However, in contrast with previous results, this proof relied upon the Modularity Theorem of Wiles et al., connecting the diophantine behavior of the Fibonacci series to the arithmetic properties of elliptic curves.

Here, we revisit this strategy with 2 aims. First, we would like to apply this modular method to other recurrence relations, listing explicitly the perfect powers occuring in them. In order to do this, we look at the generalizations of the Fibonacci sequence known as Lucas sequences, that is, recurrence relations of the forms \[ u_n = b u_{n-1} + c u_{n-2}, u_0 = 0, u_1 = 1, \] for b and c nonzero integers, and study integer solution $(n,y,p)$, $n > 0, p$ prime, to \begin{equation}\label{the_eqn}u_n = y^p\end{equation}

\begin{thm}\label{explicit_eg_thm}
For the following values of $b$ and $c$:
\begin{center}
\begin{tabular}{c | c }
b & c \\  \hline \hline
3 & -2 \\
5 & -6  \\
7 & -12 \\
17 & -72  \\
9 & -20 \\ \hline \hline
\end{tabular}
\end{center}
$u_n$ has no nontrivial $p$th powers. %except perhaps for 5th powers in $(9,-20)$.
\end{thm}
Note that for (b,c) = ???, this provides an alternate proof to a case of Catalan's Conjecture, recently proven by Mihaelescu \cite{mih04}.

The proof of \ref{explicit_eg_thm} relies upon modular methods for $p \geq 7$ and a combination of elementary and elliptic curve techniques for the remaining primes. 

These sorts of examples are easy, in the same way that Fermat's Last Theorem is easy: as we will see, any particular solution to \eqref{the_eqn} is associated to a weight 2 newforms of some level $N \sim \rad(c(b^2+4c))$, and for some small levels, there are no newforms.

As $N \to \infty$, many difficulties arise. The number of newforms grows linearly with $N$ (CHECK). For each newform corresponding to an elliptic curve, one might proceed as in \cite{siksek06}, deriving ``local conditions" on the index of the solution, which then allows linear forms in two logarithms to give a useful upper bound on $p$, below which sieve methods can feasibly check for solutions. When there are newforms with irrational coefficients, corresponding to abelian varieties of dimension $>$ 1, exceptional situations (IS THIS TRUE), such as CM abelian varieties, can be dealt with as in \cite{bennett04}. However, we do not know of any techniques, in general, to completely deal with abelian varieties.

This leads us to the second goal of our paper, which is to use modular methods to obtain general results about perfect powers in Lucas sequences. In particular, we attempt to derive explicit upper bounds on $p$ in terms of $(b,c)$, avoiding the use of the effectively computable, but inexplicit, bounds by Petho and Shorey \cite{petho82}\cite{shorey83}. We obtain the following result, conditional on the Frey-Mazur Conjecture - see \ref{FreyMazur}:

\begin{thm}\label{condbound}
Consider a solution (n,y,p) to \eqref{the_eqn}, with $n > 6$. Let $N = 2^8 \cdot \rad(c(b^2+4c))$. Then for all $\epsilon > 0$, there exist absolute effectively computable constants $C_{\epsilon}, D$, such that
\[ p \leq \max\{17, C_{\epsilon} \left( N \right)^{N + \epsilon}, \max\{30, N+1\} \cdot D\log{\alpha} \}, \]

\end{thm}

\tableofcontents

\section{Classical Facts about Lucas Sequences}


Let $(b,c) \in \Z \times \Z$ define the linear binary recurrence relation
\[ U_{n+2} = b\cdot U_{n+1}+ c\cdot U_n, \]
with characteristic polynomial and roots
\[ g(z) = z^2 - bz - c, \qquad \qquad \alpha, \beta = \frac{b \pm \sqrt{b^2+4c}}{2}.\]
Throughout this paper, we will refer to integral linear binary recurrence sequences as simply binary recurrence sequences.  In particular, let $u_n$ and $v_n$ denote the companion sequences specified by starting conditions
\[ u_0 = 0, u_1 = 1 \qquad \qquad v_0 = 2, v_1 = b .\]
The $n$th terms of these sequences are given by
\begin{equation}\label{binetform} u_n = \frac{\alpha^n - \beta^n}{\alpha - \beta} \qquad \qquad v_n = \alpha^n +\beta^n. \end{equation}
This formula easily implies the following two key facts
\begin{equation}\label{fib2} u_{2k} = u_kv_k \end{equation}
\begin{equation}\label{gen_diophan}(\alpha - \beta)^2u_n^2 = v_n^2 - 4(\alpha\beta)^n. \end{equation}

Using elementary methods alone, it is often possible to prove that there are no nontrivial perfect $p$th powers in a binary recurrence sequence for a specific fixed small value of $p$.  For example, let $(b,c)$ be one of the sequence in Table (FIX ME).  Then we have the following lemmas in the direction of Theorem (CITE).

\begin{lem}\label{relprime}
Let $(b,c)$ be any binary recurrence sequence such that $b^2+4c=1$.  For $n \geq 1$,  $u_n$, $v_n$, and $c$ are relatively prime.
\end{lem}

\begin{proof}
We prove this by induction on $n$.  First note that $b^2 +4c = 1$ forces $b$ and $c$ to be relatively prime and $c$ to be even.  Clearly as
\[ u_2 = b \cdot u_1 + c \cdot u_0 \]
and $u_1$ is relatively prime to $c$, $u_2$ is relatively prime to $c$.  Thus by induction $u_n$ is relatively prime to $c$.  Similarly for $v_n$.  And further as $u_n^2  = v_n^2 - 4(-c)^n$, no primes except perhaps those dividing $c$ can divide $u_n$ and $v_n$.  Thus they are pairwise relatively prime.
\end{proof}

\begin{lem}[Ruling out small primes]\label{smallp}
There are no nontrivial squares or cube terms in any of the examples in Table (CITE) except $u_2 = 9$ for the sequence $(9,-20)$.
\end{lem}

\begin{proof}

We first deal with the case of $p=2$; we will derive our contradiction from the relation \eqref{gen_diophan}, which in this case reduces to
\begin{equation}\label{D1dio} \alpha^n - (\alpha-1)^n = z^2.\end{equation} 
Assume that the index $n$, for which $u_n = z^2$, is odd; in this case we can absorb the sign and have a nontrivial integral solution to
\[ x^n +y^n = z^2 \]
But there are no nontrivial solutions to the $(n,n,2)$ Diophantine equation for $n \geq 4$  \cite{darmon97} .  We easily check that in our examples, there are no solutions for $n=3$.  In the case $n$ is even, write $n=2^rk$ for $k$ odd.  Then we can write
\begin{align*}
\alpha^{2^rk} - \beta^{2^rk} & = (\alpha^{2^{r-1}k} - \beta^{2^{r-1}k})(\alpha^{2^{r-1}k} + \beta^{2^{r-1}k}) \\
& = (\alpha^{2^{r-1}k} + \beta^{2^{r-1}k})(\alpha^{2^{r-2}k} + \beta^{2^{r-2}k}) \cdots (\alpha^{k} - \beta^{k}) (\alpha^{k} + \beta^{k})
\end{align*}
by \eqref{fib2}.  Further by Lemma \ref{relprime}, we know that that these terms are relatively prime.  Thus $u_n=z^2$ if and only if each of these terms is also a square.  In particular, $\alpha^k + \beta^k$ must be a square.  If $k \neq 1$, then this is identical to the previous condition of a nontrivial solution $(n,n2)$, which yields a contradiction.  If $k = 1$ (ie $n$ is powers of $2$), then this is the condition that $b$ is a perfect square.  It is easy to verify that this is not true in all of the above examples, except $(9,-20)$.  Further, $u_2 = 3^2$ is the only square in the sequence $(9,-20)$ as any other square term of index $n = 2^r$ would require that all of $u_m$ for $m = 2^{r-1},...,2^2$ also be squares.  As $u_4 = 369$, which is not a square, $u_2$ is the only perfect square in $(9,-20)$.  The proof for $p=3$ follows in exactly the same way from reduction to odd index and the solution of $(n,n,3)$ by \cite{darmon97}.

\end{proof}

In addition, in some specific cases it is possible to find all perfect powers in a binary recurrence sequence by elementary methods alone.  For example, in \cite{petho92}, Peth{\H{o}} proved that $u_7 = 13^2$ is the only nontrivial perfect power in the Pell sequence $(2,1)$.  This method is unsuccessful in general, so we introduce the heavy machinery of the modular method.


\section{The Modular Method and Theorem 1.1}

\subsection{The modular method applied to binary recurrence sequences}

The modular method is a modern approach to solving classically-intractable Diophantine equations that builds upon the theory of Galois representations associated to elliptic curves combined with the deep theorems of level-lowering and modularity of elliptic curves.  Most famously, the modular method was the key to the celebrated proof of Fermat's Last Theorem \cite{wiles95}, \cite{taylorwiles95}, as well as the determination of perfect powers in the Fibonacci sequence in \cite{siksek06}.

First, to a hypothetical solution to a Diophantine equation we associate a Frey elliptic curve with coefficients depending on the solution.  To these Frey curves we associate the mod $p$ Galois representation
\[ \rho_{E,p}: G_{\Q} \rightarrow \GL_2(\F_p) \]
corresponding to the $p$-torsion number field $\Q(E[p])$.



\section{Proof of Theorem 1.2}

Define $AV(N)$ to be the maximum prime $p \in \Z$ such that there exists an elliptic curve $E/\Q$ which level lowers, mod $p$, to a modular abelian variety of level $N$, dimension $> 1$. 

Similarly, define $Ell(b,c)$ to be the maximum prime $p \in \Z$ such that any Frey curve for the recurrence relation (b,c) level lowers, mod $p$, to an elliptic curve.

\begin{thm}\label{bound_ell}
Assuming the Frey-Mazur Conjecture, there exists an effectively computable constant $A_2$, depending on nothing, such that
\[Ell(b,c) \leq \max\{17, \max\{30, 2^{8} \cdot c(b^2+4c)+1\} \cdot A_2\log{\alpha} \} \]
\end{thm}
\begin{thm}\label{bound_av}
For all $\epsilon > 0$, there exists an effectively computable constants $A_1$, depending only on $\epsilon$, such that
\[AV(N) \leq A_1 \left( N \right)^{N + \epsilon}. \]
\end{thm}


We spend the rest of this section proving these theorems. Theorem \ref{condbound} immediately follows.


\subsection{Elliptic Curves Case and the Frey-Mazur Conjecture}

In the results that follow, we rely upon the following empirically-supported question of Mazur \cite{mazur78}, now referred to as the Frey-Mazur Conjecture, concerning the possibility of isomorphic Galois representations arising from non-isogenous elliptic curves.

\begin{conj}[Frey-Mazur]\label{FreyMazur}
Let $p > 17$, and $E_1$ and $E_2$ elliptic curves over $\Q$, with mod $p$ Galois representations $\rho_{E_1,p}$ and $\rho_{E_2,p}$ respectively.  If
\[ \rho_{E_1,p} \simeq \rho_{E_2,p} \]
then $E_1$ is isogenous to $E_2$.
\end{conj}

We prove our result \textbf{conditional on the Frey-Mazur Conjecture}. Our strategy is as follows: if the Frey curve $E$ corresponding to a solution $(n,y,p)$, $p > 17$, level lowers mod $p$ to an elliptic curve $F$, then Frey-Mazur implies that $N_E = N_F$. This forces $y$ to be the product of a fixed set of primes, depending only on $(b,c)$. Then, using general results on the smoothness of values in recurrence sequences \cite{gyory81}, \cite{gyory82}, \cite{gyory03}, we obtain a bound on $n$, in terms of $(b,c)$. We can then bound $p$ such that $y^p = u_n$ in terms of $n$.


\begin{thm}[\cite{gyory81}, \cite{gyory82}, \cite{gyory03}]\label{smoothterm}
Let $S$ be the set of all integers composed entirely of primes in some finite set $\{p_1,p_2,...,p_m\}$ with $p_m \geq p_i$ for all $i$.  Let $u_n$ be a binary recurrence sequence with starting conditions $u_0 = 0, u_1 = 1$.  For $n > 6$, if $u_n \in S$ then
\[ n \leq \max\{30, p_m +1 \}. \]
\end{thm}

\begin{lem}[Bound for $p$ in terms of $n$]\label{boundpintermsn}
There exists an absolute, effective constant $A$ such that for all solutions $u_n = y^p$, we have 
\[ p \leq A \cdot n \log(\alpha), \]
where $\alpha$ is the dominant root of the characteristic polynomial for $(b,c)$
\end{lem}

\begin{proof}
It is clear that
\[p\log{2} \leq \log \bfrac{\alpha^n - \beta^n}{\alpha-\beta}  = \log|\alpha^{n-1}+ \alpha^{n-2}\beta+...+\beta^{n-1}|. \]
Thus
\begin{align*}
p & \leq  \frac{1}{\log{2}} \cdot \log|n\alpha^{n-1}| = \frac{1}{\log{2}} \cdot (n-1)\log|\alpha| + \frac{1}{\log{2}} \log(n)  \\
& \leq A\cdot n \log|\alpha|.
\end{align*}
as $|\alpha| \geq 2$, so $\log_2|\alpha| \geq 1$ so $n\log|\alpha| \geq  \log(n)$.
\end{proof}

\begin{thm}\label{ell_bound_final}
Let $(b,c)$ relatively prime be a binary recurrence sequence with $u_0=0,u_1=1$. If $u_n = y^p$ for $n>6$, and if the associated Frey curve level lowers mod p to an elliptic curve of level N, then assuming the Frey Mazur conjecture, there exists an absolute, effectively computable constant $A_2$ such that
\[ p \leq \max\{17, \max\{30, 2^{8} \cdot c (b^2+4c)+1\} \cdot A_2\log{\alpha} \}. \]
\end{thm}

\begin{proof}
Assume there exists a solution $u_n = y^p$, $n > 6$.
We associate a Frey curve $E$ to the Diophantine equation
\[ y^{2p} +4(-c)^n = v_n^2 \]
as in \ref{freycurves}.  By proposition \ref{freycurves}, the conductor of $E$ is
\[ N_E = 2^{\alpha}  \cdot \prod_{\substack{ \ell \mid c(b^2+4c) \\ \ell \mid y \\ \ell \text{ odd}}} \ell \qquad \qquad \alpha \leq 8. \]
Thus we have a bound
\[ N_E \leq  2^{8} \cdot c \cdot (b^2+4c) \cdot y.\] 
By \ref{unram} the mod $p$ Galois representation $\rho_{E,p}$ is unramified outside $pN$, and flat at $p \notdiv c(b^2+4c)$.  Thus there exists a modular form $f$ of level 
\[N_f = 2^{\alpha} \cdot \prod_{\substack{ \ell \mid c(b^2+4c) \\ \ell \text{ odd}}} \ell \]
bounded by 
\[ 2^8 \cdot  \rad(c \cdot (b^2+4c)) \leq N \]
such that $\rho_{E,p} \simeq \bar{\rho_{f,p}} $.  

Now, assume that $f$ is rational, i.e. corresponds to an elliptic curve $E_f$ such that
\[ \rho_{E,p} \simeq \rho_{E_f,p}. \]  
Invoking the Frey-Mazur conjecture \ref{FreyMazur}, $E$ and $E_f$ are isogenous, and thus $N_E = N_{E_f}$.  But these differ exactly in the primes dividing $y$, thus
\[ \rad(y) | \rad(2c \cdot (b^2+4c)). \]
However, the largest prime factor of $2c \cdot (b^2+4c)$ is bounded by $N$.  Further by lemma \ref{boundpintermsn}, $p \leq n \cdot A\log{\alpha}$.  We conclude the theorem by contradiction.
\end{proof}








\subsection{Abelian Varieties Case}
First, we prove a general upper bound on the primes $p$ for which an irrational newform can arise as the mod $p$ level-lowering of an elliptic curve.

Let $E$ be an elliptic curve of level $M \cdot N$ with mod $p$ Galois representation $\rho_{E,p}$ irreducible, unramified outside $pN$, and flat at $p$. Let $a_\ell$ denote the coefficients of the $L$-function of $E$.  If $\rho_{E,p} \simeq \bar{\rho_{f,p}}$ for $f$ a newform of level $N$ with Fourier coeffients $c_\ell\in \O_f$ for $K = K_f = \Q(...,c_\ell,...)$ of degree $n_K = [K:\mathbb{Q}]$, then we have the following important lemma on necessary congruences.

\begin{lem}[\cite{cohen06}]\label{ircong1}
There exists a prime $\mathfrak{p} \mid p$ of $\mathcal{O}_f$ such that, for $\ell$ prime:
\begin{itemize}
\item $c_\ell \equiv a_\ell \mod \mathfrak{p}$, if $\ell \nmid pN$
\item $c_\ell^2 \equiv (\ell+1)^2 \mod \mathfrak{p}$, if $\ell \mid\mid N$
\end{itemize}
Further, as $|a_\ell| < 2\sqrt{\ell}$,
\[p \mid \gcd_{\ell \nmid N}(B(\ell)C(\ell)), \] where
\[B(\ell) = \ell \cdot N_{K_f / \mathbb{Q}}(c_\ell^2-(\ell+1)^2) \]
\[C(\ell) = \prod_{-2\sqrt{\ell} < r < 2\sqrt{\ell}}{N_{K_f / \mathbb{Q}}}(c_\ell - r).\]
\end{lem}

For $n_K > 1$, this gives a nontrivial bound on $p$, as there exists an $\ell$ such that $c_\ell \notin \mathbb{Z}$ and $\ell^2 \nmid N$. For such an $\ell$, the product above is nonzero. We can bound the first such $\ell$ using the following well-known theorem:
\begin{thm}[Sturm's Bound]\label{sturm}
Let $f,g \in M_k(\Gamma_0(N))$ have Fourier expansions $\sum_n a_nq^n$ and $\sum_n b_n q^n$ respectively.  Then $f = g$ iff $a_n = b_n$ for all
\[ n \leq \frac{k}{12} \cdot N \prod_{p|N} \left(1 + \frac{1}{p} \right) \]
\end{thm}

\begin{lem}\label{boundell}
Let $f$ be an irrational newform of level $N$.  If $\ell$ is the first prime such that $c_\ell \not\in \Z$, then
\[ \ell \leq A \cdot N^{1+\epsilon} .\]
\end{lem}

\begin{proof}
Since $f$ has some irrational coefficient, we know there exists $\sigma \in G_\Q$, such that $f^{\sigma}$ is a distinct newform, also of level $N$.  Then, by \ref{sturm}, they must differ in an coefficient $c_\ell$ for some 
\[ \ell \leq \frac{1}{6} N \cdot \prod_{p|N} \left(1 + \frac{1}{p} \right) = A \cdot N^{1+\epsilon}. \]
\end{proof}

\begin{prop}\label{irboundp}
Let $E$ be an elliptic curve and $f$ an irrational newform of level $N$ such that 
\[\rho_{E,p} \simeq \bar{\rho_{f,p}}.\]
Then, for all $\epsilon > 0$, there exists a constant $A_1$ such that
\[ p \leq  A_1 \cdot N^{N+\epsilon}. \]
(IS A EFFECTIVELY COMPUTABLE? WHAT DOES IT DEPEND ON?)
\end{prop}
\begin{proof}
If $\ell^2 \mid N$, then $c_l = 0$ (CHECK). Thus we can apply Lemma \ref{ircong1}, for $\ell$ a prime such that the $\ell$th Fourier coefficient $c_\ell$ of $f$ is not in $\Z$, 
\[ p \mid \ell \cdot N_{K / \mathbb{Q}}(c_\ell^2-(\ell+1)^2) \cdot \prod_{-2\sqrt{\ell} < r < 2\sqrt{\ell}}{N_{K / \mathbb{Q}}}(c_\ell - r).\]
Note that $c_\ell$, being the trace of $\Frob_\ell$, is the sum of two $\ell$-Weil numbers, hence satisfies $|c_\ell^{\sigma}| < 2\sqrt{l}$ for all $\sigma \in \Gal(K/\Q)$. Thus, for $k \leq \ell+1$, \[N(c_\ell - k) \leq (\ell+1 + 2\sqrt{\ell})^{n_{K}}.\] 
By Lemma \ref{boundell}, we may take $\ell \leq A \left( N \right)^{1+\epsilon}$, and $n_{K} \leq N^{1+\epsilon}$ (REF), thus
\[ p \leq A_1 \cdot N^{N+\epsilon}. \]
\end{proof}
 

\section{Examples}



















\bibliography{bib}{}
\bibliographystyle{amsalpha}


\end{document}