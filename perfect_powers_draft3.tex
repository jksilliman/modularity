\documentclass[12pt]{amsart}
\usepackage{latexsym}
\usepackage{amssymb,amsmath}
\usepackage[pdftex]{graphicx}
\usepackage{enumerate}
\usepackage{endnotes}
%\usepackage{extpfeil}
\usepackage{hyperref}
\usepackage[usenames,dvipsnames]{xcolor}
\usepackage{stackrel}
\usepackage{bbm}
\usepackage{tikz}
\usepackage[margin=1.25in]{geometry}
\usepackage{hyperref}
\usepackage{listings}
\usepackage{courier}
\usepackage{color}
\usepackage{upgreek}

\lstset{
	basicstyle=\small\ttfamily,
	keywordstyle=\color{blue},
	language=python,
	xleftmargin=16pt,
}

\usetikzlibrary{arrows,chains,matrix,positioning,scopes}

\makeatletter
\tikzset{join/.code=\tikzset{after node path={%
\ifx\tikzchainprevious\pgfutil@empty\else(\tikzchainprevious)%
edge[every join]#1(\tikzchaincurrent)\fi}}}
\makeatother
%
\tikzset{>=stealth',every on chain/.append style={join},
         every join/.style={->}}
\tikzstyle{labeled}=[execute at begin node=$\scriptstyle,
   execute at end node=$]
\usetikzlibrary{patterns}

\usetikzlibrary{decorations.pathreplacing}

\DeclareSymbolFont{bbold}{U}{bbold}{m}{n}
\DeclareSymbolFontAlphabet{\mathbbold}{bbold}

\newtheorem{thm}{Theorem}[section]
\newtheorem{ithm}{Theorem}
\newtheorem{lem}[thm]{Lemma}
\newtheorem{conj}[thm]{Conjecture}
\newtheorem{prop}[thm]{Proposition}
\newtheorem{cor}[thm]{Corollary}

\theoremstyle{definition}
\newtheorem{defi}[thm]{Definition}
\newtheorem{example}[thm]{Example}
\newtheorem{exercise}[thm]{Exercise}
\newtheorem{rem}[thm]{Remark}


   
\def\B{{\mathbb B}}
\def\C{{\mathbb C}}
\def\D{{\mathbb D}}
\def\Fp{{\mathbb F}_p}
\def\Fell{{\mathbb F}_{\ell}}
\def\F{{\mathbb F}}
%\def\H{{\mathbb H}}
\def\M{{\mathbb M}}
\def\N{{\mathbb N}}
\def\O{{\mathcal O}}
\def\0{{\mathbb 0}}
\def\P{{{\mathbb P}}}
\def\Q{{\mathbb Q}}
\def\R{{\mathbb R}}
\def\T{{\mathbb T}}
\def\Z{{\mathbb Z}}

\newcommand{\sol}{_{a^p,b^p,c^p}}
\newcommand{\bound}{\partial}
\newcommand{\la}[1]{\mathfrak{#1}}
\newcommand{\im}{\text{Im} \hspace{0.1em} }
\newcommand{\ann}{\text{Ann} \hspace{0.1em} }
\newcommand{\rank}{\text{rank} \hspace{0.1em} }
\newcommand{\coker}[1]{\text{coker}\hspace{0.1em}{#1}}
\newcommand{\sgn}{\text{sgn}}
\newcommand{\lcm}{\text{lcm}}
\newcommand{\re}{\text{Re}  \hspace{0.1em} }
\newcommand{\ext}[1]{\text{Ext}(#1)}
\newcommand{\Hom}[1]{\text{Hom}(#1)}
\newcommand{\End}[1]{\text{End(#1)}}
\newcommand{\bs}{\setminus}
\newcommand{\rpp}[1]{\mathbb{R}\text{P}^{#1}}
\newcommand{\cpp}[1]{\mathbb{C}\text{P}^{#1}}
\newcommand{\tr}{\text{tr}\hspace{0.1em} }
\newcommand{\inner}[1]{\langle {#1}\rangle}
\newcommand{\tensor}{\otimes}
\newcommand{\Cl}{\text{Cl}}
\renewcommand{\sp}[1]{\text{Sp}_{#1}}
\newcommand{\GL}{\text{GL}}
\newcommand{\PGL}{\text{PGL}}
\renewcommand{\sl}[1]{\text{SL}_{#1}}
\newcommand{\so}[1]{\text{SO}_{#1}}
\newcommand{\SO}{\text{SO}}
\newcommand{\pso}[1]{\text{PSO}_{#1}}
\renewcommand{\o}[1]{\text{O}_{#1}}
\renewcommand{\sp}[1]{\text{Sp}_{#1}}
\newcommand{\psp}[1]{\text{PSp}_{#1}}
\newcommand{\Span}{\rm Span}
\newcommand{\Frob}{\rm Frob}
\newcommand{\tor}{\rm tor}
\newcommand{\rad}{\rm rad}
\newcommand{\denom}{\rm denom}
\renewcommand{\bar}{\overline}
\newcommand{\notdiv}{\nmid}
\newcommand{\pfrac}[2]{\left( \frac{#1}{#2} \right)}
\newcommand{\bfrac}[2]{\left| \frac{#1}{#2} \right|}
\newcommand{\Ell}{\rm Ell}
\newcommand{\AV}{\rm AV}
\newcommand{\Gal}{\rm Gal}

\newcommand{\kron}[2]{\bigl(\frac{#1}{#2}\bigr)}
\newcommand{\leg}[2]{\Biggl(\frac{#1}{#2}\Biggr)}

\DeclareSymbolFont{bbold}{U}{bbold}{m}{n}
\DeclareSymbolFontAlphabet{\mathbbold}{bbold}

\begin{document}

\title{Perfect Powers in Lucas Sequences via Galois Representations}
\author{Jesse Silliman and Isabel Vogt}

\maketitle


\section{Introduction and Statement of Results}


\section{Classical Facts about Linear Binary Recurrence Sequences}

Let $(b,c) \in \Z \times \Z$ define the linear binary recurrence relation
\[ U_{n+2} = b\cdot U_{n+1}+ c\cdot U_n, \]
with characteristic polynomial and roots
\[ g(z) = z^2 - bz - c, \qquad \qquad \alpha, \beta = \frac{b \pm \sqrt{b^2+4c}}{2}.\]
Throughout this paper, we will refer to integral linear binary recurrence sequences as simply binary recurrence sequences.  In particular, let $u_n$ and $v_n$ denote the companion sequences specified by starting conditions
\[ u_0 = 0, u_1 = 1 \qquad \qquad v_0 = 2, v_1 = b .\]
The $n$th terms of these sequences are given by
\begin{equation}\label{binetform} u_n = \frac{\alpha^n - \beta^n}{\alpha - \beta} \qquad \qquad v_n = \alpha^n +\beta^n. \end{equation}
This formula easily implies the following two key facts
\begin{equation}\label{fib2} u_{2k} = u_kv_k \end{equation}
\begin{equation}\label{gen_diophan}(\alpha - \beta)^2u_n^2 = v_n^2 - 4(\alpha\beta)^n. \end{equation}

Using elementary methods alone, it is often possible to prove that there are no nontrivial perfect $p$th powers in a binary recurrence sequence for a specific fixed small value of $p$.  For example, let $(b,c)$ be one of the sequence in Table (FIX ME).  Then we have the following lemmas in the direction of Theorem (CITE).

\begin{lem}\label{relprime}
Let $(b,c)$ be any binary recurrence sequence such that $b^2+4c=1$.  For $n \geq 1$,  $u_n$, $v_n$, and $c$ are relatively prime.
\end{lem}

\begin{proof}
We prove this by induction on $n$.  First note that $b^2 +4c = 1$ forces $b$ and $c$ to be relatively prime and $c$ to be even.  Clearly as
\[ u_2 = b \cdot u_1 + c \cdot u_0 \]
and $u_1$ is relatively prime to $c$, $u_2$ is relatively prime to $c$.  Thus by induction $u_n$ is relatively prime to $c$.  Similarly for $v_n$.  And further as $u_n^2  = v_n^2 - 4(-c)^n$, no primes except perhaps those dividing $c$ can divide $u_n$ and $v_n$.  Thus they are pairwise relatively prime.
\end{proof}

\begin{lem}[Ruling out small primes]\label{smallp}
There are no nontrivial squares or cube terms in any of the examples in Table (CITE) except $u_2 = 9$ for the sequence $(9,-20)$.
\end{lem}

\begin{proof}

We first deal with the case of $p=2$; we will derive our contradiction from the relation \eqref{gen_diophan}, which in this case reduces to
\begin{equation}\label{D1dio} \alpha^n - (\alpha-1)^n = z^2.\end{equation} 
Assume that the index $n$, for which $u_n = z^2$, is odd; in this case we can absorb the sign and have a nontrivial integral solution to
\[ x^n +y^n = z^2 \]
But there are no nontrivial solutions to the $(n,n,2)$ Diophantine equation for $n \geq 4$  \cite{darmon97} .  We easily check that in our examples, there are no solutions for $n=3$.  In the case $n$ is even, write $n=2^rk$ for $k$ odd.  Then we can write
\begin{align*}
\alpha^{2^rk} - \beta^{2^rk} & = (\alpha^{2^{r-1}k} - \beta^{2^{r-1}k})(\alpha^{2^{r-1}k} + \beta^{2^{r-1}k}) \\
& = (\alpha^{2^{r-1}k} + \beta^{2^{r-1}k})(\alpha^{2^{r-2}k} + \beta^{2^{r-2}k}) \cdots (\alpha^{k} - \beta^{k}) (\alpha^{k} + \beta^{k})
\end{align*}
by \eqref{fib2}.  Further by Lemma \ref{relprime}, we know that that these terms are relatively prime.  Thus $u_n=z^2$ if and only if each of these terms is also a square.  In particular, $\alpha^k + \beta^k$ must be a square.  If $k \neq 1$, then this is identical to the previous condition of a nontrivial solution $(n,n2)$, which yields a contradiction.  If $k = 1$ (ie $n$ is powers of $2$), then this is the condition that $b$ is a perfect square.  It is easy to verify that this is not true in all of the above examples, except $(9,-20)$.  Further, $u_2 = 3^2$ is the only square in the sequence $(9,-20)$ as any other square term of index $n = 2^r$ would require that all of $u_m$ for $m = 2^{r-1},...,2^2$ also be squares.  As $u_4 = 369$, which is not a square, $u_2$ is the only perfect square in $(9,-20)$.  The proof for $p=3$ follows in exactly the same way from reduction to odd index and the solution of $(n,n,3)$ by \cite{darmon97}.

\end{proof}

In addition, in some specific cases it is possible to find all perfect powers in a binary recurrence sequence by elementary methods alone.  For example, in \cite{petho92}, Peth{\H{o}} proved that $u_7 = 13^2$ is the only nontrivial perfect power in the Pell sequence $(2,1)$.  This method is unsuccessful in general, so we introduce the heavy machinery of the modular method.


\section{The Modular Method and Theorem 1.1}

\subsection{The modular method applied to binary recurrence sequences}

The modular method is a modern approach to solving classically-intractable Diophantine equations that builds upon the theory of Galois representations associated to elliptic curves combined with the deep theorems of level-lowering and modularity of elliptic curves.  Most famously, the modular method was the key to the celebrated proof of Fermat's Last Theorem \cite{wiles95}, \cite{taylorwiles95}, as well as the determination of perfect powers in the Fibonacci sequence in \cite{siksek06}.

First, to a hypothetical solution to a Diophantine equation we associate a Frey elliptic curve with coefficients depending on the solution.  To these Frey curves we associate the mod $p$ Galois representation
\[ \rho_{E,p}: G_{\Q} \rightarrow \GL_2(\F_p) \]
corresponding to the $p$-torsion number field $\Q(E[p])$.





\vspace{100pt}

\section{The Frey-Mazur Conjecture and Theorem 1.2}


\section{Examples}



















\bibliography{bib}{}
\bibliographystyle{amsalpha}


\end{document}