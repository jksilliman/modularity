\documentclass[12pt]{article}
\usepackage{latexsym}
\usepackage{amssymb,amsmath}
\usepackage[pdftex]{graphicx}
\usepackage{listings}
\usepackage{courier}
\usepackage{color}
\usepackage[usenames,dvipsnames]{xcolor}
\usepackage{enumerate}
\usepackage{endnotes}
%\usepackage{extpfeil}
\usepackage{stackrel}
\usepackage{bbm}
\usepackage{tikz}
\usepackage[margin=2cm]{geometry}
\usepackage{hyperref}

\lstset{
	basicstyle=\small\ttfamily,
	keywordstyle=\color{blue},
	language=python,
	xleftmargin=16pt,
}

\newtheorem{thm}{Theorem}[section]
\newtheorem{ithm}{Theorem}
\newtheorem{lem}[thm]{Lemma}
\newtheorem{prop}[thm]{Proposition}
\newtheorem{cor}[thm]{Corollary}
\newenvironment{proof}[1][Proof.]{\begin{trivlist}
\item[\hskip \labelsep {\bfseries #1}]}{\end{trivlist}}

\newtheorem{defi}[thm]{Definition}
\newtheorem{example}[thm]{Example}
\newtheorem{exercise}[thm]{Exercise}
\newtheorem{rem}[thm]{Remark}

   
\def\B{{\mathbb B}}
\def\C{{\mathbb C}}
\def\D{{\mathbb D}}
\def\Fp{{\mathbb F}_p}
\def\Fell{{\mathbb F}_{\ell}}
\def\F{{\mathbb F}}
\def\H{{\mathbb H}}
\def\M{{\mathbb M}}
\def\N{{\mathbb N}}
\def\O{{\mathcal O}}
\def\0{{\mathbb 0}}
\def\P{{{\mathbb P}}}
\def\Q{{\mathbb Q}}
\def\R{{\mathbb R}}
\def\T{{\mathbb T}}
\def\Z{{\mathbb Z}}

\newcommand{\sol}{_{a^p,b^p,c^p}}
\newcommand{\bound}{\partial}
\newcommand{\la}[1]{\mathfrak{#1}}
\newcommand{\im}{\text{Im} \hspace{0.1em} }
\newcommand{\ann}{\text{Ann} \hspace{0.1em} }
\newcommand{\rank}{\text{rank} \hspace{0.1em} }
\newcommand{\coker}[1]{\text{coker}\hspace{0.1em}{#1}}
\newcommand{\sgn}{\text{sgn}}
\newcommand{\lcm}{\text{lcm}}
\newcommand{\re}{\text{Re}  \hspace{0.1em} }
\newcommand{\ext}[1]{\text{Ext}(#1)}
\newcommand{\Hom}[1]{\text{Hom}(#1)}
\newcommand{\End}[1]{\text{End(#1)}}
\newcommand{\bs}{\setminus}
\newcommand{\rpp}[1]{\mathbb{R}\text{P}^{#1}}
\newcommand{\cpp}[1]{\mathbb{C}\text{P}^{#1}}
\newcommand{\tr}{\text{tr}\hspace{0.1em} }
\newcommand{\inner}[1]{\langle {#1}\rangle}
\newcommand{\tensor}{\otimes}
\newcommand{\Cl}{\text{Cl}}
\renewcommand{\sp}[1]{\text{Sp}_{#1}}
\newcommand{\GL}{\text{GL}}
\newcommand{\pgl}[1]{\text{PGL}_{#1}}
\renewcommand{\sl}[1]{\text{SL}_{#1}}
\newcommand{\so}[1]{\text{SO}_{#1}}
\newcommand{\SO}{\text{SO}}
\newcommand{\pso}[1]{\text{PSO}_{#1}}
\renewcommand{\o}[1]{\text{O}_{#1}}
\renewcommand{\sp}[1]{\text{Sp}_{#1}}
\newcommand{\psp}[1]{\text{PSp}_{#1}}
\newcommand{\Span}{\rm Span}
\newcommand{\Frob}{\rm Frob}
\newcommand{\tor}{\rm tor}
\newcommand{\rad}{\rm rad}
\newcommand{\denom}{\rm denom}
\renewcommand{\bar}{\overline}
\newcommand{\notdiv}{\nshortmid}
\newcommand{\pfrac}[2]{\left( \frac{#1}{#2} \right)}

\newcommand{\kron}[2]{\bigl(\frac{#1}{#2}\bigr)}
\newcommand{\leg}[2]{\Biggl(\frac{#1}{#2}\Biggr)}

\DeclareSymbolFont{bbold}{U}{bbold}{m}{n}
\DeclareSymbolFontAlphabet{\mathbbold}{bbold}

\begin{document}


\begin{center}
{\bf \large{Bounds on Abelian Varieties}} \\
\smallskip
Isabel Vogt, Jesse Silliman\\
Last Edited: \today \\
\end{center}

\section{Bounds on p for Abelian Varieties}
We state a standard lemma: [Cohen]

If an elliptic curve $E$ arises mod p from a modular form $f$ of level N with field of coefficients $K_f = \mathbb{Q}(...,c_l,...)$, $d = [K_f,\mathbb{Q}]$, there exists a prime $\mathfrak{p} \mid p$ of $\mathcal{O}_f$ such that, for $l$ prime:
\begin{itemize}
\item $c_l \equiv a_l(E) \mod \mathfrak{p}$, if $l \nmid pN$
\item $c_l^2 \equiv (l+1)^2 \mod \mathfrak{p}$, if $l \mid N$
\end{itemize}

Further, $|a_l| < 2\sqrt{l}$ (Weil Bound), so we have
\[p \mid \gcd_{l \nmid N}(B(l)C(l)), \] where
\[B(l) = lN_{K_f / \mathbb{Q}}(c_l^2-(l+1)^2) \]
\[C(l) = \prod_{-2\sqrt{l} < r < 2\sqrt{l}}{N_{K_f / \mathbb{Q}}}(c_l - r)\]


For $d > 1$, this gives us a nontrivial bound on $p$: since there exists an $l$ such that $c_l \notin \mathbb{Z}$, and then the product is nonzero.

\begin{rem}
If $q^2 \mid N$, we believe that $\gcd_{l \nmid N}(c_l - (l+1)) = 1$, for if it were divisible by $p$, the Galois representation would be reducible, isomorphic to $\chi_p \oplus 1$, and this should not happen, by looking at the action of the $q$-inertia subgroup. It is unclear if this can be used to help our bound. We would at very least need to rule out $c_l + (l+1)$ as well.
\end{rem}

Let $l$ be the smallest prime number such that $c_l \notin \mathbb{Z}$. Then, since $c_l$ is an $l$-Weil number, we see that for $k \leq l+1$, $N(c_l - k) \leq (l+1 + 2\sqrt{l})^d$

Using an effective version of Chebotarev's Density Theorem, we can obtain a computable bound. A specialization of a theorem of Lagarias:

\begin{thm}
There exists an absolute, effectively computable const $A_1$ such that for every Galois extension $K/\mathbb{Q}$ of degree d, every conjugacy class $C \subset G(K/\mathbb{Q})$, there exists a prime $p \in Z$, unramified in $K$, such that $Frob_p = C$, $p \leq d^{A_1}$. 

**CHECK**
%Algebraic number fields : (L-functions and Galois properties) : proceedings of a symposium / organised by the London Mathematical Society with the support of the Science Research Council and the Royal Society ; edited by A. Frohlich.%
Further, under the assumption of GRH, we have the bound \[p \leq A_2 (\log d)^2 \].
\end{thm}

Assuming GRH, then, we see that $l$, as above, is less than $A_2 (\log d)^2$, so we conclude that
\[ p \leq (A_2 (\log d)^2 + 1 + 2\sqrt{A_2}\log(d))^d \]


\end{document}